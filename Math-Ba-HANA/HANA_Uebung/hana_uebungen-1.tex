\begin{exercisePage}[Topologische Räume][15/15]
%
% Aufgabe 1
\begin{exercise}
	Untersuchen Sie die topologischen Räume $(\Rn, \tau)$ aus dem Beispiel der Vorlesung ($\tau$ ist euklidische Topologie, triviale Topologie, diskrete Topologie $\pows{\Rn}$ oder Streifentopologie):
	\begin{enumerate}[leftmargin=*, nolistsep, topsep=-\parskip]
		\item Ist $(\Rn , \tau)$ separiert?
		\item  Ist $\tau$ die von einer Metrik auf $\Rn$ erzeugte Standardtopologie?
	\end{enumerate}
\end{exercise}

\begin{enumerate}[leftmargin=\zulength, label=(zu \alph*)]
	\item Seien $x,y \in \Rn$ beliebig. Definiere $\epsilon \defeq \frac{1}{2} \dist{x}{y}$. Dann gilt für die Umgebungen $U = B_\epsilon(x)$ von $u$ und $V = B_\epsilon(y)$ von $v$ nach Konstruktion stets $U \cap V = \emptyset$, d.h. $\Rn$ ist separiert bzgl. der euklidischen Topologie.
	
	Sei $\tau = \menge{\emptyset, \Rn}$ und seien $u,v \in \Rn$ beliebig. Angenommen $\Rn$ sei separiert, dann existieren Umgebungen $U$ von $u$ und $V$ von $v$. Da $U$ und $V$ Umgebungen sind, existieren $M,N \in \tau$ mit $v \in M \subseteq U$ und $v \in N \subseteq V$. Dann muss aber schon $M = N = \Rn$ gelten, da kein Element in $\emptyset$ enthalten ist. Somit folgt daraus auch $U = V = \Rn$. Da $U \cap V = \Rn \neq \emptyset$ ist $\Rn$ nicht separiert.
	
	Sei $\tau = \pows{\Rn}$ und $u,v \in \Rn$ beliebig. Verwende stets Umgebungen $U = \menge{u} \in \tau$ von $u$ und $V = \menge{v} \in \tau$ von $v$. Für $u \neq v$ ist $U \cap V = \emptyset$, also ist $\Rn$ separiert.
	
	Sei $\tau = \menge{ \menge{(x_1,x_2, \dots, x_n) \in \Rn : x_1 \in R} : R \subseteq \R \text{ euklidisch offen}}$ die Streifentopologie. Angenommen $\Rn$ sei separiert.
	Seien $u = (u_1 , u_2, \dots, u_n), v = (v_1, v_2, \dots, v_n) \in \Rn$ mit $u_1 = v_1$. Die kleinste Menge $M \in \tau$ mit $u \in M$ ist $\menge{(x_1,x_2, \dots, x_n) \in \Rn : x_1 = u_1}$. Somit muss jede Umgebung $U$ von $u$ auch $M$ enthalten, also $M \subseteq U$. Anderseits ist $M$ auch in jeder Umgebung $V$ von $v$ enthalten, da $u_1 = v_1$ gilt. Somit ist $\emptyset \neq M \subseteq U \cap V$, also $U \cap V \neq \emptyset$ und damit $\Rn$ nicht separiert.
	
	\item Die euklidische Topologie $\tau$ ist offensichtlich metrisierbar mit dem euklidischen Abstandsbegriff 
	\begin{equation*}
		\dist{x}{y} = \abs{x - y} = \sqrt{\sum_{i=1}^{n} (x_i - y_i)^2}
	\end{equation*}
	
	Die triviale Topologie $\tau = \menge{\emptyset, \Rn}$ ist nicht metrisierbar. Angenommen es gäbe eine Metrik $d$ und $\kappa$ die davon erzeugte Standardtopologie. Seien $x,y \in \Rn$ mit $x \neq y$ beliebig. Definiere $\epsilon \defeq \dist{x}{y} > 0$. Setze außerdem $\delta \defeq \frac{\epsilon}{2}$. Dann ist $B_\delta(x)$ offen bezüglich $\kappa$, d.h. $B_\delta \in \kappa$. Jedoch ist $x \in B_\delta(x)$, aber $y \notin B_\delta(x)$ und somit $B_\delta(x) \neq \emptyset$ und $B_\delta(x) \neq \Rn$, also muss $\tau \neq \kappa$ sein.
	
	Die diskrete Topologie $\tau = \pows{\Rn}$ ist metrisierbar mit 
	\begin{equation*}
		\dist{x}{y} = \begin{cases} 0 & x = y \\ 1 & x \neq y \end{cases}
	\end{equation*}
	Damit haben die $\epsilon$-Kugeln folgende Gestalt: $B_\epsilon(x) = \menge{y \in \Rn: \dist{x}{y} < \epsilon} = \menge{x}$ für $0 < \epsilon < 1$.
	Schließlich können wir die Zugehörigkeit eines Elementes $x$ zu eine Menge $M \in \tau$ mit einer Kugel vom Radius $0 < \epsilon < 1$ beschreiben, was genau der Definition einer induzierten Topologie entspricht.
	
	Die Streifentopologie ist nicht metrisierbar. Mithilfe der Standardkonstruktion der von einer Metrik $d$ erzeugten Topologie können wir zeigen, dass ein metrischer Raum $(X,d)$ stets separiert ist. Nehmen wir an, dass die Streifentopolgie metrisierbar wäre, dann wäre sie auch separiert, was wir oben bereits widerlegt haben. Somit existiert also keine Metrik, die die Streifentopologie induziert.
\end{enumerate}


\begin{exercise}
	Es sei $(X, d)$ ein metrischer Raum. Zeigen Sie:
	\begin{enumerate}[leftmargin=*, nolistsep, topsep=-\parskip]
		\item Für alle Punkte $x, x', y, y' \in X$ gilt die sogenannte Vierecksungleichung
		\begin{equation*}
			\abs{\dist{x}{x'} - \dist{y}{y'}} \le \dist{x}{y} + \dist{x'}{y'}
		\end{equation*}
		\item Ist $(X, d)$ separabel, so gilt das zweite Abzählbarkeitsaxiom.
	\end{enumerate}
\end{exercise}

\begin{enumerate}[leftmargin=\zulength, label=(zu \alph*)]
	\item Seien $x,x',y,y' \in X$. Dann gilt mit zweifacher Anwendung der Dreiecksungleichung für eine Metrik $d$
	\begin{equation*}
		\begin{alignedat}{2}
		\dist{x}{x'} &\le \dist{x}{y}  + \dist{y}{y'} + \dist{y'}{x'} &&= \dist{x}{y} + \dist{y}{y'} + \dist{x'}{y'} \\
		\dist{y}{y'} &\le \dist{y}{x} + \dist{x}{x'} + \dist{x'}{y'} &&= \dist{x}{y} + \dist{y}{y'} + \dist{x'}{y'}
		\end{alignedat}
	\end{equation*}
	Durch Subtraktion von $\dist{x}{x'}$ bzw. $\dist{y}{y'}$ erhält man schließlich die beiden Ungleichungen
	\begin{equation*}
		\begin{alignedat}{2}
		\dist{x}{x'} -  \dist{y}{y'} &\le \dist{x}{y} + \dist{x'}{y'} \\
		- \brackets{\dist{x}{x'} - \dist{y}{y'}} &\le \dist{x}{y} + \dist{x'}{y'}
		\end{alignedat}
	\end{equation*}
	woraus schlussendlich die behauptete Gleichung
	\begin{equation*}
		\abs{\dist{x}{x'} - \dist{y}{y'}} \le \dist{x}{y} + \dist{x'}{y'}
	\end{equation*}
	folgt.
	
	\item Sei $(X,d)$ ein separabler, metrischer Raum, d.h. es existiert eine höchstens abzählbare, dichte Menge $D$ in $X$. Da $D$ dicht ist bezüglich der von $d$ induzierten Standardtopologie $\tau$, d.h. für alle $x \in D$ und für alle $r > 0$ existiert ein $y \in D$ mit $\dist{x}{y} < r$. Somit können wir als Basis $\mathcal{B} = \menge{B_\epsilon(x) : x \in D, 0 < \epsilon \in \Q}$ wählen.
	Da sowohl $D$ als auch $\Q$ abzählbar ist, ist auch $\mathcal{B}$ wieder abzählbar. Nun müssen wir noch zeigen, dass $\mathcal{B}$ auch eine Basis von $\tau$ ist. Dies ist jedoch klar, da wir für alle $x \in M \in \tau$ stets die Kugeln $B_\epsilon(x)$ wählen können und deren Vereinigung bereits $M$ beschreibt, also für hinreichend kleine $0 < \epsilon_x \in \Q$ gilt
	\begin{equation*}
		M = \bigcup_{x \in M} B_{\epsilon_x}(x)
	\end{equation*}
\end{enumerate}

\begin{exercise}
	Es sei $(X, \tau)$ ein topologischer Raum. Beweisen Sie:
	\begin{enumerate}[leftmargin=*, nolistsep, topsep=-\parskip]
		\item Ist $\tau$ die triviale Topologie, so konvergiert jede Folge gegen jeden Punkt.
		\item Ist $\tau$ die diskrete Topologie, so konvergiert eine Folge genau dann gegen $u \in X$, wenn alle bis auf endlich viele Folgenglieder mit $u$ übereinstimmen.
		\item  Ist $X = \Rn$ ($n \ge 2$) $\tau$ die Streifentopologie, so hat keine konvergente Folge einen eindeutigen Grenzwert.
	\end{enumerate}
	Hinweis: \textit{Eine Folge $\folge{x_k}{k \in \N} \subseteq X$ konvergiert per Definition gegen $u \in X$, wenn für jede Umgebung $U$ von $u$ höchstens endlich viele Glieder der Folge nicht in $U$ liegen.}
\end{exercise}

\begin{enumerate}[label=(zu \alph*)]
	\item Seien $\folge{x_n}{n \in \N} \subseteq X$ und $x \in X$ beliebig. Da $\tau = \menge{\emptyset, X}$ ist und $\emptyset$ nie eine Umgebung sein kann, existiert nur eine Umgebung $X$ von $u$. Da $\folge{x_n}{n \in \N} \subseteq X$, also $x_n \in X$ für alle $n \in \N$, liegen alle Glieder der Folge in jeder Umgebung von $u$ und somit $x_n \to x$ für alle Folgen $\folge{x_n}{n \in \N} \subseteq X$ und alle $x \in X$.
	%
	\item Sei $\folge{x_n}{n \in \N} \subseteq X$ eine gegen $x \in X$ konvergente Folge, d.h. für jede Umgebung $U$ von $x$ liegen nur endlich viele Folgenglieder außerhalb von $U$. Da $\menge{x} \in \tau$ eine Umgebung von $x$ ist, liegen auch fast alle $x_n \in \menge{x}$, was gleichbedeutend ist mit $x_n = x$ für fast alle $x_n$.
	
	Angenommen es stimmen fast alle (alle bis auf endlich viele) Folgenglieder $x_n$ mit $x$ überein, dann ist stets $\menge{x} \subseteq U \in \tau$ für eine Umgebung $U$ von $x$. Somit liegen stets auch nur endliche viele $x_n \notin \menge{x} \subseteq U$.
	%
	\item Sei $\folge{u_k}{k \in \N} \subseteq \Rn$ eine beliebige gegen $u = (u^1, u^2 , \dots , u^n) \in \Rn$ konvergente Folge. %Notiere $x_k = (x_k^1 , x_k^2 , \dots , x_k^n)$.
	Wähle $v = (v^1, v^2, \dots , v^n) \in \Rn$ mit $v^1 = u^1$, aber $v \neq u$. Dann ist $u$ in jeder Umgebung von $v$ enthalten und vice versa. Sei dazu $V$ eine Umgebung von $v$, d.h. es existiert eine Menge $M \in \tau$ mit $v \in M \subseteq U$. Die kleinste Umgebung ist genau ein Streifen der Breite Null, d.h. $M^\ast \defeq \menge{(x^1, x^2, \dots, x^n): x^1 = v^1} \subseteq V$ für jede Umgebung $V$ von $v$. Jedoch ist auch $u \in M^\ast$, sodass $u \in V$ für jede Umgebung $V$ von $v$. Mit vertauschten Rollen folgt auch, dass $v \in U$ für jede Umgebung $U$ von $u$. Erfüllt also $U$ die Konvergenzbedingung, so wird sie auch $V$ erfüllt.
\end{enumerate}
\end{exercisePage}