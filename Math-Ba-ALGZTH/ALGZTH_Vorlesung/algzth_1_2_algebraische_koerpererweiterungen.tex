\section{Algebraische Körpererweiterungen}
Sei $L | K$ eine Körpererweiterung.

\begin{definition}[!]
    Sei $\alpha \in L$. Gibt es ein $0 \neq f \in \polynom[K]$ mit $f(\alpha) = 0$, so heißt $\alpha$ \begriff{algebraisch} über $K$, andernfalls \begriff{transzendent} über $K$.
\end{definition}

\begin{beispiel}
    \begin{enumerate}[nolistsep, leftmargin=*, label=(\alph*)]
        \item $\alpha \in K \follows \alpha$ ist algebraisch über $K$ (denn $f(\alpha) = 0$ für $f = X - \alpha \in \polynom[K]$)
        \item $\sqrt{-1} \in \Q(\sqrt{-1})$ ist algebraisch über $\Q$ (denn $f(\sqrt{-1})=0$ für $f = X^2 + 1 \in \polynom[\Q]$) \\
        $\sqrt{-1} \in \C$ ist algebraisch über $\R$        
    \end{enumerate}
\end{beispiel}

\begin{bemerkung}
    Sind $K \subseteq L \subseteq M$ Körper und $\alpha \in M$ algebraisch über $K$, so auch über $L$.
\end{bemerkung}

\begin{lemma} \label{lemma: 1_2.4}
    Genau dann ist $\alpha \in L$ algebraisch über $K$, wenn $1, \alpha, \alpha^2 , \dots$ $K$-linear abhängig sind.
\end{lemma}
\begin{proof}
    Für $\lambda_0 , \lambda_1 , \dots \in K$, fast alle gleich Null, ist
    $\sum_{i=0}^\infty \lambda_i \alpha^i \equivalent f(\alpha) = 0$ für $f = \sum_{i=0}^\infty \lambda_i X^i \in \polynom[K]$.
\end{proof}

\begin{lemma}
    Betrachte den Epimorphismus
    \begin{equation*}
        \bigabb{\phi_\alpha}{\polynom[K]}{\polynom[K][\alpha]}{f}{f(\alpha)}
    \end{equation*}
    Genau dann ist $\alpha$ algebraisch über $K$, wenn $\Ker(\phi_\alpha) \neq (0)$. In diesem Fall ist $\Ker(\phi_\alpha) = (f_\alpha)$ mit einem eindeutig bestimmten irreduziblen, normierten $f_\alpha \in \polynom[K]$.
\end{lemma}
\begin{proof}
    $\polynom[K]$ Hauptidealring $\follows \Ker(\phi_\alpha) = (f_\alpha)$, $f_\alpha \in \polynom[K]$, o.E. sei $f_\alpha$ normiert. Aus $\polynom[K][\alpha] \subseteq L$ nullteilerfrei folgt, dass $\Ker(\phi_\alpha)$ prim ist. Somit ist $f_\alpha$ prim und im Hauptidealring also auch irreduzibel.
\end{proof}

\begin{definition}
    Sei $\alpha \in L$ algebraisch über $K$, $\Ker(\phi_\alpha) = (f_\alpha)$ mit $f_\alpha \in \polynom[K]$ normiert und irreduzibel.
    \begin{enumerate}[nolistsep, topsep=-\baselineskip, leftmargin=*]
        \item $\MinPol{\alpha}{K} \defeq f_\alpha$, das \begriff{Minimalpolynom} von $\alpha$ über $K$.
        \item $\deg(\alpha \, | \, K) \defeq \deg(f_\alpha)$, der \begriff{Grad} von $\alpha$ über $K$.
    \end{enumerate}
\end{definition}

\begin{satz}
    Sei $\alpha \in L$.
    \begin{enumerate}[label=(\alph*), topsep=-\baselineskip, leftmargin=*]
        \item $\alpha$ transzendent über $K$ \\
        \follows $\polynom[K][\alpha] \isomorph \polynom[K]$ $\quad$, $\quad$ $K(\alpha) \isomorph_K K(X)$ $\quad$ , $\quad$ $[K(\alpha) \colon K] = \infty$.
        \item $\alpha$ algebraisch über $K$ \\
        \follows $\polynom[K][\alpha] = K(\alpha) \isomorph \lfrac{\polynom[K]}{\MinPol{\alpha}{K}}$ $\quad$ , $\quad$ $[ K(\alpha) \colon K)]  = \deg(\alpha | K) < \infty$ $\quad$ und \\
         $1, \alpha, \dots , \alpha^{\deg(\alpha | K) -1}$ ist $K$-Basis von $K(\alpha)$. 
     \end{enumerate}
\end{satz}
\begin{proof}
    \begin{enumerate}[leftmargin=*, label=(\alph*)]
        \item $\Ker(\phi_\alpha) = (0)$ \follows $\phi_\alpha$ ist Isomorphismus (da zusätzlich injektiv) \\
        \follows $K(\alpha) \isomorph_K \Quot(\polynom[K][\alpha]) \isomorph_K \Quot(\polynom[K]) = K(X)$ \\
        \follows $[K(\alpha) \colon K] = [K(X) \colon K] = \infty$
        \item Sei $f = f_\alpha = \MinPol{\alpha}{K}$, $n = \deg(\alpha | K) = \deg(f)$.
        \begin{itemize}
            \item $f$ irreduzibel $\follows (f) \neq (0)$ prim $\overset{\text{GEO II.4.7}}{\Longrightarrow} (f)$ ist maximal \\
            $\follows \polynom[K][\alpha] \isomorph \lfrac{\polynom[K]}{(f)}$ ist Körper $\follows \polynom[K][\alpha] = K(\alpha)$
            \item $1, \alpha, \dots , \alpha^{n-1}$ sind $K$-linear unabhängig: 
            \begin{equation*}
                \sum_{i=0}^{n-1} \lambda_i \alpha^i = 0 \follows \sum_{i=0}^{n-1} \lambda_i X^i \in (f) \quad \overset{\deg f = n}{\Longrightarrow} \quad \lambda_i = 0 \enskip \forall i
            \end{equation*}
            $1, \alpha, \dots , \alpha^{n-1}$ ist Erzeugendensystem: Für $g \in \polynom[K]$ ist 
            \begin{equation*}
                g = q*f + r \text{ mit } q,r \in \polynom[K] \text{ und } \deg(r) < \deg(f) = n
            \end{equation*}
            und  
            \begin{equation*}
                g(\alpha) = q(\alpha) \underbrace{f(\alpha)}_{=0} + r(\alpha) = r(\alpha)
            \end{equation*}
            somit $\polynom[K] = \Image(\phi_\alpha) = \menge{g(\alpha) \colon g \in \polynom[K]} = \menge{r(\alpha) \colon r \in \polynom[K], \deg(r) < n} = \sum_{i=0}^{n-1} K * \alpha^i$
        \end{itemize}
    \end{enumerate}
\end{proof}

\begin{beispiel}
    \begin{enumerate}[leftmargin=*, label=(\alph*)]
        \item $p \in \Z$ prim $\follows$ $\sqrt{p} \in \C$ ist algebraisch über $\Q$. \\
        Da $f(X) = X^2 - p$ irreduzibel in $\polynom[\Q]$ ist (GEO II.7.3), ist $\MinPol{\sqrt{p}}{\Q} = X^2 - p$, $[\Q(\sqrt{p}) \colon \Q] = 2$.
        \item Sei $\zeta_p = e^{\lfrac{2\pi i}{p}} \in \C$ ($p \in \N$ prim). Da $\Phi_p =  \frac{X^p-1}{X-1} = X^{p-1} + X^{p-2} + \cdots + X + 1 \in \polynom[\Q]$ irreduzibel in $\polynom[\Q]$ ist (GEO II.7.9), ist $\MinPol{\zeta_p}{\Q} = \Phi_p$, $[\Q(\zeta_p) \colon \Q] = p-1$. Daraus folgt schließlich $[\C \colon \Q] \geq [\Q(\zeta_p) \colon \Q] = p-1 \enskip \forall p \follows [\C \colon \Q] = \infty \follows [\R \colon \Q] = \infty$.
        \item $e \in \R$ ist transzendent über $\Q$ (Hermite 1873), 
        $\pi \in \R$ ist transendent über $\Q$ (Lindemann 1882). \\
        Daraus folgt: $[\R \colon \Q] \geq [\Q(\pi) \colon \Q] = \infty$. Jedoch ist unbekannt, ob z.B. $\pi + e$ transzendent ist.
    \end{enumerate}
\end{beispiel}

\begin{definition}
    $L | K$ ist \begriff{algebraisch} $\defequiv$ jedes $\alpha \in L$ ist algebraisch über $K$.
\end{definition}

\begin{satz}
    $L | K$ endlich $\follows$ $L | K$ algebraisch.
\end{satz}
\begin{proof}
    $\alpha \in L$, $[L \colon K] = n$ $\follows 1, \alpha, \dots , \alpha^n$ $K$-linear abhängig $\overset{\ref{lemma: 1_2.4}}{\follows} \alpha$ algebraisch über $K$.
\end{proof}

\begin{korollar}
    Ist $L = K(\alpha_1, \dots, \alpha_n)$ mit $\alpha_1, \dots, \alpha_n$ algebraisch über $K$, so ist $L | K$ endlich, insbesondere algebraisch.
\end{korollar}
\begin{proof_induction}[$n$]
    \ianfang[$n=0$] $\checkmark$
    \ischritt[$n > 0$] $K_1 \defeq K(\alpha_1, \dots, \alpha_{n-1})$ \\
        $\follows L=K_1(\alpha_n)$, $\alpha_n$ algebraisch über $K_1$ (2.3) \\
        $\follows [L \colon K] = \underbrace{[K_1(\alpha_n) \colon K_1]}_{< \infty \text{ nach 2.7}}* \underbrace{[K_1 \colon K]}_{< \infty \text{ nach IH}}$
\end{proof_induction} 

\begin{korollar}
    Es sind äquivalent:
    \begin{enumerate}[nolistsep, leftmargin=*, topsep=-\parskip]
        \item $L | K$ ist endlich.
        \item $L | K$ ist endlich erzeugt und algebraisch.
        \item $L = K(\alpha_1, \dots , \alpha_n)$ mit $\alpha_1, \dots, \alpha_n$ algebraisch über $K$.
    \end{enumerate}
\end{korollar}
\begin{proof}
    \begin{description}[nolistsep, leftmargin=*]
        \item[(1) $\Rightarrow$ (2):] 1.15 und 2.10
        \item[(2) $\Rightarrow$ (3):] trivial
        \item[(3) $\Rightarrow$ (1):] 2.11
    \end{description}
\end{proof}

\begin{bemerkung}
    Nach 2.7 ist
    \begin{equation*}
        \alpha \text{ algebraisch über } K \equivalent \polynom[K][\alpha] = K(\alpha)
    \end{equation*}
    Direkter Beweis für $(\Rightarrow)$: \\
    Sei $0 \neq \beta \in \polynom[K][\alpha]$. Daraus folgt, dass $f(\beta) = 0$ für ein irreduzibles $0 \neq f = \sum_{i=0}^n a_i X^i \in \polynom[K]$. Durch Einsetzen von $\beta$ und Division durch $\beta$ erhält man (auch wegen der aus der Irreduzibilität folgenden Bedingung $a_0 \neq 0$)
    \begin{equation*}
        \beta^{-1} = -a_0^{-1} ( a_1 + a_2 \beta + \dots + a_n \beta^{n-1}) \in \polynom[K][\beta] \subseteq \polynom[K][\alpha]
    \end{equation*}
\end{bemerkung}