\section{Spaltengenerierung}

Wir betrachten die stetige Relaxation des Bin-Packing-Problems (vgl. \cref{chapter_1_einfuehrung}). Zur Erinnerung: Es sind $b_i$ Teile der Länge $\ell_i$ ($i = 1, \dots, m$) in möglichst wenige Behälter der Kapazität $L$ zu packen. 
\begin{itemize}[nolistsep, topsep=-\parskip]
	\item Packungsvarianten: $a^j = \transpose{a_1^j , \dots , a_m^j} \in \Z_+^m$ mit $\trans{\ell}a^j \le L$ ($j \in J$)
	\item Variablen: $x_j$ beschreibt Häufigkeit, wie oft Variante $a^j$ genutzt wird.
\end{itemize}
\begin{equation*}
	z = \sum_{j \in  J} x_j \to \min \quad \bei \quad \sum_{j \in J} a_i^j * x_j = b_i \quad (i \in I) \quad \und \quad x_j \ge 0 \quad (j \in J)
\end{equation*}
Grundsätzlich ist diese Aufgabe mit dem Simplexverfahren lösbar, jedoch gibt es im Allgemeinen exponentiell viele Variablen, sodass pro Austauschschritt ein großer Aufwand entstünde.

Wir können uns hierbei zu Nutze machen, dass alle Spalten der Systemmatrix $A$ eine gemeinsame Struktur aufweisen:
\begin{equation*}
	a^j \text{ ist Spalte von } A \equivalent a^j \in \Z_+^m \und \trans{\ell} a^j \le L
\end{equation*}

Offenbar gilt hie $c = e = \transpose{1, \dots, 1}$, sodass für eine gewählte Basismatrix $A_B$ in Schritt 2 des Simplexalgorithmus folgendes zu bestimmen wäre:
\begin{equation*}
	\quer{q} \defeq \min_{j \in J_N} q_j \quad \mit \quad q_j = c_j - \trans{d} a^j = 1 - \trans{d}a^j, \quad a^j \in \Z_+^m, \trans{\ell} a^j \le L
\end{equation*}
wobei $\trans{d} \defeq \trans{c_B} A_B^{-1}$. In Schritt 2 wäre folglich die Aufgabe 
\begin{equation*}
	1 - \trans{d} a^j = q_j \to \min \bei \trans{\ell} a^j \le L \und a^j \in \Z_+^m
\end{equation*}
bzw. 
\begin{equation*}
	\trans{d} a^j \to \max \bei \trans{\ell} a^j \le L, a^j \in \Z_+^m
\end{equation*}
zu lösen.
Gilt $q_j^\ast < 0$, so liegt keine Optimalität vor und eine zugehörige Lösung $a^{j, \ast}$ wäre in die Basismatrix aufzunehmen. Gilt $g_j^\ast \ge 0$, so sind wir fertig.