\section{Das Minimalgerüst-Problem / Minimum Spanning Tree Problem}

Gegeben sei ein ungerichteter, zusammenhängender und gewichteter Graph $G = (V,E,c)$, wobei $\abb{c}{E}{\R}$ die Gewichtsfunktion ist.

Gesucht ist eine Teilmenge $T \subseteq E$ mit $\card{T} = \card{V} - 1$, sodass der induzierte Teilgraph $G_T = (V,T)$ kreisfrei ist.

\begin{definition}
	Jede Teilmenge $T$ mit diesen Eigenschaften wird \begriff{Gerüst} oder \begriff{Spannbaum} genannt.
\end{definition}

\begin{bemerkung}
	Der zu einem Spannbaum $T$ gehörende Teilgraph $G_T$ ist zusammenhängend.
\end{bemerkung}

Das Minimum-Spanning-Tree (MST) Problem besteht nun darin, einen Spannbaum zu finden, dessen Gesamtgewicht $\sum_{e \in T} c(e)$ kleinstmöglich ist. Es lässt sich mithilfe eines sogenannten \begriff{Greedy-Algorithmus} effizient lösen.