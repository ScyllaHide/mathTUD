\section{Beispiele zur diskreten Optimierung}

\subsection{Das Rucksackproblem}

Gegeben seien ein Behälter (''Rucksack``) mit Kapazität $b \in \Z_+ \defeq \menge{0, 1, \dots}$ sowie $m$ Teile, die jeweils durch ein Gewicht $a_i \in \Z_+$ und einen Nutzen $c_i \in \Z_+$ beschrieben werden ($i = 1, \dots , m$). Aus dieser Menge von Objekten ist eine nutzenmaximale Teilmenge auszuwählen.


\begin{description}
	\item[Variablen] \begin{equation*}
		x_i \defeq \begin{cases}
		1 & \text{wenn Teil $i$ eingepackt wird} \\ 0 & \text{sonst}
		\end{cases} \quad (i = 1, \dots , m)
	\end{equation*}
	\item[Zielfunktion] $f(x) = \sum\limits_{i=1}^{m} c_i x_i \to \max$
	\item[Nebenbedingungen] Kapazitätsbedingung: $\sum\limits_{i=1}^{m} a_i x_i \le b$
\end{description}

Als Modell können wir somit formulieren:
\begin{equation*}
	\begin{aligned}
	f(x) = \sum_{i=1}^{m} c_i x_i \to \max \quad \bei \sum_{i=1}^m a_i x_i \le b \und x_i \in \menge{0,1} \enskip (i = 1, \dots , m)
	\end{aligned}
\end{equation*}

Aufgrund der binären Gestalt der Variablen wird das Problem auch als $0/1$-Rucksackproblem bezeichnet. Im Gegensatz dazu ist beim klassischen Rucksackproblem jedes Teil mehrfach nutzbar. In diesem Fall ist $x_i \in \Z_+$ zu fordern.

\subsection{Das Bin-Packing-Problem}
\label{subsec: 1.3.2}

Gegeben seien (sehr große) Anzahl an Behältern der Kapazität $L$ sowie $b_i$ Teile des Gewichts oder Volumens $\ell_i$ mit $i \in I = \menge{1, \dots , m}$. Man ermittle die minimale Anzahl an Behältern, die benötigt wird, um alle Objekte zu verstauen.
Jede Packung (eines Behälters) kann als Vektor $a = (a_1 , \dots , a_m) \in \Z_+^m$ geschrieben werden, wobei $a_i$ angibt, wie oft das Teil $i$ benutzt wird. Ein solcher Vektor ist eine zulässige Packung, wenn 
\begin{equation*}
	\sum_{i=1}^m \ell_i a_i \le L
\end{equation*} 
ist.

\begin{description}
	\item[Modell nach \person{Kantorovich}] Wir benötigen 
	\begin{itemize}[nolistsep]
		\item eine obere Schranke $u \in \Z_+$ für die maximal benötigte Anzahl an Behältern
		\item $y_k = \begin{cases}
		1 & \text{wenn Rucksack } k \text{ benutzt wird} \\ 0 & \text{sonst}
		\end{cases} \quad (k = 1 , \dots , u)$
		\item $x_{ik} \in \Z_+$, die angeben, wieviele Objekte vom Typ $i$ in Rucksack $k$ gepackt werden ($(i,k) \in \menge{1, \dots , m} \times \menge{1, \dots , u}$)
	\end{itemize}
	Daraus ergibt sich nun folgendes Modell:
	\begin{equation*}
		\begin{aligned}
		f^\text{Kant}(x,y) = \sum_{k=1}^u y_k \to \min \bei \quad & \sum_{k=1}^u x_{ik} = b_i \quad &&(i = 1, \dots , m) \\
		& \sum_{i=1}^m x_{ik} \ell_i \le L * y_k \quad &&(k = 1 , \dots , u) \\
		& y_k \in \menge{0,1} \quad &&(k = 1 , \dots , u) \\
		& x_{ik} \in \Z_+ \quad &&((i,k) \in \menge{1, \dots , m} \times \menge{1, \dots , u})
		\end{aligned}
	\end{equation*}
	Die erste Nebenbedingung sorgt dafür, dass alle Teile gepackt werden; die zweite Nebenbedingung liefert die Einhaltung der Kapazität unter Berücksichtigung, dass nur bepackte Behälter gezählt werden.

	Es kann stets $u = \sum_{i=1}^m b_i$ gewählt werden. Das Auffinden besserer Schranken ist im Allgemeinen schwierig.
	Eine Relaxation kann z.B. durch $y_k \in [0,1]$ und $x_{ik} \in \R_+$ erhalten werden. Diese liefert jedoch keine guten Näherungen.
	%
	\item[Modell von Gilmore \& Gomory] Es seien $J$ eine Indexmenge aller zulässigen Packungen und $x_j \in \Z_+$ ($j \in J$) die Häufigkeit, wie oft ein Behälter nach dem durch $j$ angegebenen Schema $a^j = (a_1^j , \dots , a_m^j)$ mit $\trans{\ell} a^j \le L$ gefüllt wird.
	Daraus ergibt sich folgendes Modell:
	\begin{equation*}
		\begin{aligned}
		f^{GG}(x) = \sum_{j \in J} x_j \to \min \quad \bei \quad 
		& \sum_{j \in J} a_i^j * x_j = b_i \quad && (i = 1, \dots , m) \\
		& x_j \in \Z_+ && (j \in J)
		\end{aligned}
	\end{equation*}
	Die Nebenbedingung sorgt dafür, dass alle Teile gepackt werden.
	
	Es gibt im Allgemeinen exponentiell viele zulässige Packungen $a^j$ ($j \in J$), deren Koeffizienten allesamt in den Nebenbedingungen benötigt werden.
	
	Eine Relaxation erhält man zum Beispiel durch $x_j \in \R_+$. Diese stetige Relaxation ist sehr gut; man vermutet, dass folgende Bedingung gilt:
	\begin{equation*}
		f^{GG, \ast} - f^{GG, \ast}_\text{relax} < 2
	\end{equation*}
\end{description}

Erfreulicherweise gibt es zum Gilmore-Gomory-Modell äquivalente Formulierungen, die mit einer polynomiellen Zahl von Variablen arbeiten und eine ebenso gute stetige Relaxation besitzen (z.B. Flussmodelle).


\subsection{Standortplanung}
Ein Dienstleister möchte neue Filialen aufbauen, um seine Kunden $k \in K \defeq \menge{1, \dots , m}$ zu versorgen. Dabei sind aus der Menge $S \defeq \menge{1, \dots , n}$ mögliche Standorte, die neuen Standorte so auszuwählen, dass der Bedarf aller Kunden befriedigt wird und die Gesamtkosten minimial sind.

Wir benötigen
\begin{itemize}[nolistsep, topsep=-\parskip]
	\item $c_s > 0$ \dots Fixkosten für den Aufbau von Standort $s \in S$
	\item $d_{ks} > 0$ \dots Kosten, um den Kunden $k \in K$ (vollständig) von Standort $s \in S$ zu beliefern.
\end{itemize}

Variablen:
\begin{itemize}[nolistsep, topsep=-\parskip]
	\item $x_s = \begin{cases} 1 & \text{wenn Standort } s \in S \text{ gebaut wird} \\ 0 & \text{sonst} \end{cases}$ 
	\item $y_{ks} \ge 0$ \dots Anteil des Bedarfs des Kunden $k \in K$, der vom Standort $s \in S$ bedient wird (implizit: $y_{ks} \in [0,1]$)
\end{itemize}

Modell zur Standortplanung:
\begin{equation*}
	f(x,y) = \underbrace{\sum_{s \in S} x_s c_s}_{\text{Fixkosten}}  + \underbrace{\sum_{s \in S} \sum_{k \in K} y_{ks} d_{ks}}_{\text{variable Kosten}} \to \min
\end{equation*}
bei
\begin{equation*}
\begin{alignedat}{2}
	\sum_{s \in S} y_{ks} &= 1 &&(k \in K) \\
	y_{ks} &\le x_s &&(s \in S, k \in K) \\
	x_s &\in \menge{0,1}  &&(s \in S) \\
	y_{ks} &\ge 0\qquad &&(k \in K, s \in S)
\end{alignedat}
\end{equation*}

\subsection{Quadratisches Zuordnungsproblem}

Es sollen $n$ Personen auf $n$ Räume verteilt werden. Person $i$ muss Person $j$ $c_{ij}$ mal am Tag treffen. Außerdem habe Büro $k$ von Büro $\ell$ die Entfernung $d_{k \ell}>0$. Wird Person $i$ das Büro $k$ zugewiesen und Person $j$ das Büro $\ell$, so ergibt sich eine Gesamtwegstrecke von $2 c_{ij} d_{kl}$ (beachte Hin- und Rückweg). Gesucht ist die wegminimale Belegung der Büros.

Variablen: $x_{ik} = \begin{cases}
	1 & \text{wenn Person $i$ das Büro $k$ zugewiesen wird} \\ 0 & \text{sonst}
	\end{cases} \quad (i,k) \in \menge{1, \dots , n} \times \menge{1, \dots , n}$
	
Zielfunktion:
\begin{equation*}
	f(x) = \sum_{i=1}^n \sum_{k=1}^n \sum\limits_{\substack{j=1 \\ j \neq i}}^n \sum\limits_{\substack{\ell =1 \\ \ell \neq k}}^n x_{j \ell} x_{i k} * 2c_{ij} d_{kl} \to \min
\end{equation*}
bei
\begin{equation*}
	\begin{alignedat}{3}
	\sum_{i=1}^n x_{ik} &= 1 \quad &&(k = 1, \dots, n) \quad &&\text{Büro $k$ bekommt genau einen Einwohner} \\
	\sum_{k=1}^n x_{ik} &= 1 &&(i = 1, \dots, n) &&\text{Person $i$ bekommt genau ein Büro} \\
	x_{ik} \in &\menge{0,1} \quad &&(i,k) \in \menge{1, \dots , n} \times \menge{1, \dots , n} &&
	\end{alignedat}
\end{equation*}

Weitere Beispiele sind in der Vorlesung ''Diskrete Optimierung`` (Master Mathe) zu finden..