\section{Das Lemma von \person{Farkas}}

Das folgende Resultat besitzt vielfältige Anwendungen in der Optimierung ($\nearrow$ Dualität).

\begin{lemma}[Farkas] %2.7
	Es seien $A \in \R^{m \times n}$ und $a \in \Rm$.  Von den Systemen
	\begin{enumerate}[label=(\roman*), nolistsep, topsep=-\parskip]
		\item $Az \le 0$, $\trans{a} z > 0$
		\item $\trans{A} u = a$, $u \ge 0$.
	\end{enumerate}
	ist \textit{genau} eines lösbar.
\end{lemma}
\begin{proof}
	\begin{itemize}[leftmargin=*, nolistsep]
		\item \textit{höchstens} eines der Systeme ist lösbar: Seien (i) und (ii) lösbar. Dann gilt
		\begin{equation*}
			0 < \trans{a} z = \trans{\trans{A}u} z = \underbrace{\trans{u}}_{\ge 0} \underbrace{A z}_{\le 0} \le 0 \quad \lightning
		\end{equation*}
		\item  \textit{mindestens} eines der Systeme ist lösbar --- die Unlösbarkeit von (ii) impliziert die Lösbarkeit von (i):  Sei (ii) nicht lösbar. Dann gilt
		$a \notin K \defeq \menge{x = \trans{A} u \colon u \ge 0}$, wobei $K$ ein konvexer, abgeschlossener Kegel ist. Wir betrachten die Optimierungsaufgabe
		\begin{equation*}
			f(x) = \frac{1}{2} \norm{a-x}_2^2 = \frac{1}{2} \transpose{a-x}(a-x) \to \min \bei x \in K
		\end{equation*}
		Dann existiert eine eindeutige (und globale) Lösung $\quer{x} \in K$ mit
		\begin{equation*}
			(1) \quad \nabla \trans{f(\quer{x})} \quer{x} = 0 
			\qquad \qquad 
			(2) \quad \nabla \trans{f(\quer{x})} x \ge 0 \enskip \forall x \in K
		\end{equation*}
		Zunächst folgt gemäß \cref{aussage: 2.4}, dass $\nabla \trans{f(\quer{x})} (x - \quer{x}) \ge 0$ für alle $x \in K$. Durch Einsetzen von $x = \frac{1}{2} \quer{x} \in K$ und $x = 2*\quer{x} \in K$ (beachte: $K$ ist Kegel) erhält man (1). Dies wiederum lässt sich zur notwendigen Bedingung ddazu addieren und man erhält (2). 
		Nun zeigen wir, dass $z \defeq a - \quer{x} \neq 0$ (wegen $a \notin K$) das System (i) löst. Es gilt $\nabla f(\quer) = -z$ und damit folgt
		\begin{equation*}
			0 = \nabla \trans{f(\quer{x})} \quer{x} = -\trans{z} * (\quer{x} - a + a) = \trans{z} (z-a)  \follows \trans{a} z = \trans{z} z \overset{z \neq 0}{>} 0
		\end{equation*}
		Weiter gilt $x \in K$ genau dann, wenn ein $u \ge 0$ existiert mit $x = \trans{A} u$. Aus (2) folgt dann 
		\begin{equation*}
			\begin{aligned}
			\nabla \trans{f(\quer{x})} x \ge 0 \enskip \forall x \in K &\follows& - \trans{z} \trans{A} u &\ge 0 \enskip &&\forall u \ge 0 \\
			&\follows& \transpose{Az} u &\le 0 \enskip &&\forall u \ge 0 \\
			&\follows& Az &\le 0 && (\text{wähle z.B. wieder } u = e^1, e^2, \dots)
			\end{aligned}
		\end{equation*}
		Damit löst $z$ das System (1).
	\end{itemize}
\end{proof}

Damit können die notwendigen Optimalitätsbedingungen \eqref{eq: 2.6} bzw. äquivalent dazu
\begin{equation}
	\forall i \in I_0(x) \colon \trans{a_i} * d \le 0 \follows \nabla \trans{f(\schlange{x})} * d \ge 0 \label{eq: 2.7}
\end{equation}
wie folgt umformuliert werden: Offenbar ist \eqref{eq: 2.7} gleichbedeutend mit der Unlösbarkeit von 
\begin{equation*}
	\nabla \trans{f(\schlange{x})} * d < 0, \qquad \trans{a_i} * d \le 0 \enskip \forall i \in I_0(\schlange{x})
\end{equation*}
Wählt man also im Lemma von Farkas $a = - \nabla f(\schlange{x})$ und $A$ bestehend aus den Zeilen $\trans{a_i}$, so folgt die Lösbarkeit des Systems
\begin{equation}
	\nabla f(\schlange{x}) + \sum_{i \in I_0(\schlange{x})} u_i a_i = 0 \qquad (u \ge 0) \label{eq: 2.8}
\end{equation}
Für konvexe Optimierungsaufgaben ist die Lösbarkeit von \eqref{eq: 2.8} sogar äquivalent dazu, dass $\schlange{x}$ Lösung der betrachteten Aufgabe ist.

Gerade im Hinblick auf die praktische Anwendbarkeit ist \eqref{eq: 2.8} in der jetzigen Form wenig hilfreich, da $\schlange{x}$ und damit $I_0(\schlange{x})$ unbekannt sind. Man betrachtet daher oftmals die folgende äquivalente Umformulierung:

\begin{lemma} %2.8
	Sei $G \defeq \menge{x \in \Rn \colon \trans{a_i}x \le b_i , i \in I}$ und $\abb{f}{G}{\R}$ stetig differenzierbar. Wenn $x \in \Rn$ Lösung von
	\begin{equation*}
		f(x) \to \min \bei x \in  G
	\end{equation*}
	ist, dann existiert ein Vektor $u$, sodass das Paar $(x,u)$ das folgende System löst:
	\begin{align}
		\begin{split}
		\nabla f(x) + \sum_{i \in I} u_i a_i &= 0 \qquad u_i \ge 0 , \quad \trans{a_i} - b_i \le 0 \quad (i \in I) \\
		u_i \brackets{\trans{a_i}x - b_i} &= 0 \qquad (i \in I)
		\end{split} \label{eq: 2.9}
	\end{align}
	Dabei beschreibt $\trans{a_i} - b_i \le 0$ die Zulässigkeit von $x \in G$ und $u_i \brackets{\trans{a_i}x - b_i} = 0$ gleicht die zu große Indexmenge der Summe wieder aus, d.h. für inaktive Restriktionen folgt $u_i = 0$.
	Ist $f$ konvex, so gilt auch die Umkehrung. Man nennt \eqref{eq: 2.9} auch ein \begriff{KKT-System}.
\end{lemma}

\begin{bemerkung} %2.6
	\begin{enumerate}[nolistsep]
		\item KKT steht für \person{Karush}-\person{Kuhn}-\person{Tucker}.
		\item Die Variablen $u$ heißen \begriff{\person{Lagrange}-Multiplikatoren}.
		\item Gibt es neben den Ungleichungen auch Gleichungsrestriktionen $\trans{a_i}x = b_i$ für $i = m+1, \dots , \quer{m}$ und $\quer{m} > m$, dann erhält man das KKT-System
		\begin{align}
			\begin{split}
			\nabla f(x) + \sum_{i=1}^m u_i a_i + \sum_{i=m+1}^{\quer{m}} u_i a_i &= 0 \\
			u_i \ge 0 , \trans{a_i}x - b_i &\le 0 \qquad (i = 1, \dots , m) \\
			\trans{a_i}x - b_i &= 0 \qquad (i = m+1, \dots , \quer{m}) \\
			u_i \brackets{\trans{a_i}x - b_i} &= 0 \qquad (i = 1 , \dots , m)
			\end{split}
		\end{align}
	\end{enumerate}
\end{bemerkung}