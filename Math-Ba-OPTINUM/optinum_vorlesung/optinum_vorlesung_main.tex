% This work is licensed under the Creative Commons
% Attribution-NonCommercial-ShareAlike 4.0 International License. To view a copy
% of this license, visit http://creativecommons.org/licenses/by-nc-sa/4.0/ or
% send a letter to Creative Commons, PO Box 1866, Mountain View, CA 94042, USA.

% (c) Eric Kunze, 2019

%%%%%%%%%%%%%%%%%%%%%%%%%%%%%%%%%%%%%%%%%%%%%%%%%%%%%%%%%%%%%%%%%%%%%%%%%%%%
% Template for lecture notes and exercises at TU Dresden.
%%%%%%%%%%%%%%%%%%%%%%%%%%%%%%%%%%%%%%%%%%%%%%%%%%%%%%%%%%%%%%%%%%%%%%%%%%%%

\documentclass[ngerman, a4paper, 11pt]{report}

\usepackage[ngerman]{babel}
\usepackage{../../texmf/tex/latex/layoutMathTUD}
\usepackage[smallequationskip]{../../texmf/tex/latex/mathworkMathTUD}

\usepackage{../../texmf/tex/latex/mathoperatorsMathTUD}
\usepackage{optinum_theorem}

\usepackage{../../texmf/tex/latex/titlepageMathTUD}
\usepackage{../../texmf/tex/latex/graphicsMathTUD}

%%%%%%%%%%%%%%%%%%%%%%%%%%%%%%%%%%%%%%%%%%%%%%%%%%%%%%%%%%%%%%%%%%%
%                        TITLE STYLES                             %
%%%%%%%%%%%%%%%%%%%%%%%%%%%%%%%%%%%%%%%%%%%%%%%%%%%%%%%%%%%%%%%%%%%

\usepackage{titlesec}   % change title headings look
\usepackage{chngcntr}   % modify counters
\usepackage{relsize}    % relative font size (smaller[i], larger[i], ...)

\makeatletter
\@ifpackageloaded{opensans}{}{\usepackage[scale=1]{opensans}}
\ifx\osfamily\undefined
    \newcommand*{\osfamily}{\fontfamily{fos}\selectfont}
    \DeclareTextFontCommand{\textos}{\osfamily}
\fi
\makeatother

\newcommand{\titlefont}{\osfamily}
\newcommand{\chaptersize}{\huge}
\newcommand{\sectionsize}{\LARGE}

\renewcommand{\thepart}{\Alph{part}}

% \titleformat{<command>}[<shape>]{<format>}{<label>}{<sep>}{<before-code>}[<after-code>]
% \titlespacing*{<command>}{<left>}{<before-sep>}{<after-sep>}[<right-sep>]

%%%%%%%%% Kapitel * \\ Titel
%\titleformat{\chapter}[display]{\bfseries\titlefont\color{cddarkblue}}{\chaptersize\smaller \chaptername\;\thechapter}{-10pt}{\chaptersize\MakeUppercase}%
%\titlespacing{\chapter}{10pt}{0pt}{10pt}%

%%%%%%%%% like break but additionally framed
\titleformat{\chapter}[frame]{\bfseries\titlefont\color{cddarkblue}}{\enskip \chaptersize \smaller \chaptername\;\thechapter \enskip}{10pt}{\chaptersize\centering\MakeUppercase}%
\titlespacing{\chapter}{0pt}{0pt}{10pt}%


%%%%%%%%% chapter.section
\counterwithin{section}{chapter}%
\titleformat*{\section}{\bfseries\titlefont\sectionsize}%{\thesection}{8pt}{}%
\titlespacing{\section}{0pt}{10pt}{5pt}
\titleformat*{\subsection}{\bfseries\titlefont\sectionsize\smaller}

%%%%%%%%% section.
%\renewcommand{\thechapter}{\Roman{chapter}}
%\titlelabel{\thetitle.\quad} % "." behind section/sub... (3. instead of 3)
%\counterwithout{section}{chapter}%
%\titleformat*{\section}{\bfseries\titlefont\sectionsize}%{\thesection}{8pt}{}%
%\titleformat*{\subsection}{\bfseries\titlefont\sectionsize\smaller}

%%%%%%%%% section
%\titlelabel{\thetitle \quad} % no "." behind section/sub... (3 instead of 3.)
%\titleformat{\section}[hang]{\bfseries\titlefont\sectionsize}{\thesection}{8pt}{}%
%\titleformat*{\section}{\bfseries\titlefont\sectionsize}
%\titleformat*{\subsection}{\bfseries\titlefont\sectionsize\smaller}

%%%%%%%%%%%%%%%%%%%%%%%%%%%%%%%%%%%%%%%%%%%%%%%%%%%%%%%%%%%%%%%%%%%
%                          HIGHLIGHTING                           %
%%%%%%%%%%%%%%%%%%%%%%%%%%%%%%%%%%%%%%%%%%%%%%%%%%%%%%%%%%%%%%%%%%%
\newcommand{\begriff}[1]{\textbf{#1}}
\newcommand{\person}[1]{\textsc{#1}}

%%%%%%%%%%%%%%%%%%%%%%%%%%%%%%%%%%%%%%%%%%%%%%%%%%%%%%%%%%%%%%%%%%%
%                             COUNTER                             %
%%%%%%%%%%%%%%%%%%%%%%%%%%%%%%%%%%%%%%%%%%%%%%%%%%%%%%%%%%%%%%%%%%%
\usepackage{chngcntr}

% automatic reset of section after chapter ended 
\pretocmd{\chapter}{\setcounter{section}{0}}{}{}

% automatic reset of equation counter in each section
\pretocmd{\chapter}{\setcounter{equation}{0}}{}{}

%\counterwithin{theorem}{chapter}
%\counterwithin{definition}{chapter}
%\counterwithin{satz}{chapter}
%\counterwithin{lemma}{chapter}
%\counterwithin{proposition}{chapter}
%\counterwithin{folgerung}{chapter}
%\counterwithin{korollar}{chapter}
%\counterwithin{beispiel}{chapter}
%\counterwithin{erinnerung}{chapter}
%\counterwithin{wiederholung}{chapter}
%\counterwithin{bemerkung}{chapter}
%\counterwithin{anmerkung}{chapter}
%\counterwithin{algorithmus}{chapter}

%%%%%%%%%%%%%%%%%%%%%%%%%%%%%%%%%%%%%%%%%%%%%%%%%%%%%%%%%%%%%%%%%%%
%                          ENUMERATIONS                           %
%%%%%%%%%%%%%%%%%%%%%%%%%%%%%%%%%%%%%%%%%%%%%%%%%%%%%%%%%%%%%%%%%%%
\usepackage{enumerate}
\usepackage[inline]{enumitem} 		%customize label

\renewcommand{\labelitemi}{\raisebox{1pt}{\scalebox{.6}{$\blacksquare$}}}
\renewcommand{\labelitemii}{$\vartriangleright$}
\renewcommand{\labelitemiii}{--}
% Variantionen des Dreiecks als Aufzählungszeichen $\blacktriangleright$ / $\vartriangleright$ / $\triangleright$

\renewcommand{\labelenumi}{(\arabic{enumi})}
\renewcommand{\labelenumii}{\alph{enumii}.}
\renewcommand{\labelenumiii}{\roman{enumiii}.}
%%%%%%%%%%%%%%%%%%%%%%%%%%%%%%%%%%%%%%%%%%%%%%%%%%%%%%%%%%%%%%%%%%%


%%%%%%%%%%%%%%%%%%%%%%%%%%%%%%%%%%%%%%%%%%%%%%%%%%%%%%%%%%%%%%%%%%%
%                         HEADER & FOOTER                         %
%%%%%%%%%%%%%%%%%%%%%%%%%%%%%%%%%%%%%%%%%%%%%%%%%%%%%%%%%%%%%%%%%%%
\newcommand*{\rightinfo}{Vorlesung ''Optimierung`` bei Dr. Martinovic im Wintersemester 2019/20}

\usepackage{tikz}       % needed for right info
\usetikzlibrary{calc}

\usepackage{fancyhdr} 	% customize header / footer
% Add new page-style (just footer), patch \chapter command to use this page style

\fancypagestyle{myStyle}{%
    \fancyhf{} %
    \fancyfoot[C]{\thepage} %
    \renewcommand{\headrulewidth}{0pt}     % Line at the header invisible
    \renewcommand{\footrulewidth}{0pt}     % Line at the footer visible
    \fancyhead[C]{\textcolor{gray}\leftmark} %
    \fancyhead[R]{%
        \begin{tikzpicture}[overlay,remember picture]
        \node [
        fill=none,  % Farbe des Randstreifens
        text=gray,  % Textfarbe
        font=\osfamily\normalsize,  % Einstellungen für die Schrift
        inner xsep=\footskip,       % Abstand des Textes von unten
        % maximale Textbreite = Papierhöhe - 2*Abstand des Textes von unten:
        text width={\dimexpr\paperheight-2\footskip\relax},
        align=center,
        minimum height=7mm,% Breite des Randstreifens
        anchor=south west,
        rotate=90
        ]
        at ($(current page.south east)+(-10mm,0mm)$)
        {\rightinfo};
        \end{tikzpicture}%
     }
}

\fancypagestyle{rightinfo}{%
    \fancyhf{} %
    \fancyfoot[C]{\thepage} %
    \renewcommand{\headrulewidth}{0pt}     % Line at the header invisible
    \renewcommand{\footrulewidth}{0pt}     % Line at the footer visible
    \fancyhead[R]{%
        \begin{tikzpicture}[overlay,remember picture]
        \node [
        fill=none,  % Farbe des Randstreifens
        text=gray,  % Textfarbe
        font=\sffamily\normalsize,  % Einstellungen für die Schrift
        inner xsep=\footskip,       % Abstand des Textes von unten
        % maximale Textbreite = Papierhöhe - 2*Abstand des Textes von unten:
        text width={\dimexpr\paperheight-2\footskip\relax},
        align=center,
        minimum height=7mm,% Breite des Randstreifens
        anchor=south west,
        rotate=90
        ]
        at ($(current page.south east)+(-10mm,0mm)$)
        {\rightinfo};
        \end{tikzpicture}%
     }
}

%% changes pagestyle on first page of each chapter; instead of empty page the normal footer is printed
\patchcmd{\chapter}{\thispagestyle{plain}}{\thispagestyle{rightinfo}}{}{}

\pagestyle{myStyle}
\pagenumbering{arabic}

%% remember chapter-title in \leftmark and \rightmark
%\renewcommand{\chaptermark}[1]{%
%    \markboth{\chaptername
%        \ \thechapter:\ #1}{}}
%
%% remember section title in \leftmark
%\renewcommand{\sectionmark}[1]{%
%    \markright{\thesection.\ #1}{}}
%
%%change header:
%\renewcommand{\headrulewidth}{0.75pt}
%\renewcommand{\footrulewidth}{0.3pt}
%\lhead{\rightmark}%left: section-number. section-title
%\rhead{\leftmark}%right: chapter chapternumber: chapter-title

% remove page number from part{}-pages
%\let\sv@endpart\@endpart
%\def\@endpart{\thispagestyle{empty}\sv@endpart}
%%%%%%%%%%%%%%%%%%%%%%%%%%%%%%%%%%%%%%%%%%%%%%%%%%%%%%%%%%%%%%%%%%%


%%%%%%%%%%%%%%%%%%%%%%%%%%%%%%%%%%%%%%%%%%%%%%%%%%%%%%%%%%%%%%%%%%%
%                        TABLE OF CONTENTS                        %
%%%%%%%%%%%%%%%%%%%%%%%%%%%%%%%%%%%%%%%%%%%%%%%%%%%%%%%%%%%%%%%%%%%
\usepackage{tocloft}

\renewcommand{\cfttoctitlefont}{\titlefont\Huge\bfseries}
%%%%%%%%%%%%%%%%%%%%%%%%%%%%%%%%%%%%%%%%%%%%%%%%%%%%%%%%%%%%%%%%%%%

%%%%%%%%%%%%%%%%%%%%%%%%%%%%%%%%%%%%%%%%%%%%%%%%%%%%%%%%%%%%%%%%%%%
%                            LISTINGS                             %
%%%%%%%%%%%%%%%%%%%%%%%%%%%%%%%%%%%%%%%%%%%%%%%%%%%%%%%%%%%%%%%%%%%
\usepackage{listings}

%%%%%%%%%%%%%%%%%%%%%%%%%%%%%%%%%%%%%%%%%%%%%%%%%%%%%%%%%%%%%%%%%%%
%                           REFERENCES                            %
%%%%%%%%%%%%%%%%%%%%%%%%%%%%%%%%%%%%%%%%%%%%%%%%%%%%%%%%%%%%%%%%%%%
\RequirePackage[unicode,bookmarks=true]{hyperref}
\hypersetup{
    % pdfborder={0 0 0}			% no boxed around links
    pdfborderstyle={/S/U/W 1},	% underlining insteas of boxes
    linkbordercolor=cdblue,
    urlbordercolor=cdblue
%	colorlinks,
%	citecolor=black,
%	filecolor=cddarkblue!80,
%	linkcolor=black,
%	urlcolor=cddarkblue!80
}

\RequirePackage{cleveref}
\crefname{theorem}{Theorem}{Theoreme}
\crefname{satz}{Satz}{Sätze}
\crefname{lemma}{Lemma}{Lemmata}
\crefname{aussage}{Aussage}{Aussagen}
\crefname{proposition}{Proposition}{Propositionen}
\crefname{folgerung}{Folgerung}{Folgerungen}
\crefname{korollar}{Korollar}{Korollare}
\crefname{definition}{Definition}{Definitionen}
\crefname{bemerkung}{Bemerkung}{Bemerkungen}
\crefname{beispiel}{Beispiel}{Beispiele}
\crefname{erinnerung}{Erinnerung}{Erinnerungen}
\crefname{algorithmus}{Algorithmus}{Algorithmen}

\RequirePackage{bookmark}		% pdf-bookmarks


%\usepackage{../../texmf/tex/latex/referencesMathTUD}

%%%%%%%%%%%%%%%%%%%%%%%%%%%%%%%%%%%%%%%%%%%%%%%%%%%%%%%%%%%%%%%%%%%%%%%%%%%%

%---------------------------------------
% additional packages
%---------------------------------------

% none

%---------------------------------------
% general settings
%---------------------------------------

\name{Eric Kunze}
\matnr{Nummer}
\email{\href{mailto:eric.kunze@mailbox.tu-dresden.de}{\ttfamily eric.kunze@mailbox.tu-dresden.de}}

\modul{Optimierung und Numerik}
\period{Wintersemester 2019/20}

%\renewcommand{\tutor}{Dr. Legrand}
%\renewcommand{\group}{Tag x. DS, (un)gerade Woche}

\lecturer{Dr. John Martinovic}
\faculty{Mathematik}
\institute{Numerik}
\professorship{Numerik der Optimalen Steuerung}

%%%%%%%%%%%%%%%%%%%%%%%%%%%%%%%%%%%%%%%%%%%%%%%%%%%%%%%%%%%%%%%%%%%%%%%%%%%%



\undef\folge
\NewDocumentCommand{\folge}{m m}{\left\{ #1 \right\}_{#2}}
\renewcommand{\complement}{\mathsf{C}}

\newcommand{\widesim}[1][2.5]{
	\mathrel{\scalebox{#1}[1]{\ensuremath{\sim}}}
}

\newenvironment{indentpar}{\vspace{\parskip} \par \setlength{\parindent}{1cm}}{\vspace{\parskip} \par}

\newcolumntype{R}[1]{>{\raggedleft\arraybackslash}p{#1}}

\makeatletter
\newcommand{\leqnomode}{\tagsleft@true\let\veqno\@@leqno}
\newcommand{\reqnomode}{\tagsleft@false\let\veqno\@@eqno}
\makeatother

%%%%%%%%%%%%%%%%%%%%%%%%%%%%%%%%%%%%%%%%%%%%%%%%%%%%%%%%%%%%%%%%%%%%%%%%%%%%


\begin{document}

\makeTUtitle
    
\tableofcontents

\chapter{Einführung}
\label{chapter_1_einfuehrung}
\section{Aufgabenstellung und Grundbegriffe}

Es seien $G \subseteq \Rn$ und $\abb{f}{G}{\R}$ gegeben. In dieser Vorlesung betrachten wir Optimierungsaufgaben (OA) der Form
\begin{equation}\label{eq: oa}
	f(x) \to \min \quad \bei x \in G
\end{equation}
Man nennt
\begin{itemize}[nolistsep, topsep=-\parskip]
	\item $f$ die \begriff{Zielfunktion},
	\item $G$ den \begriff{zulässigen Bereich} und
	\item ein $x \in  G$ \begriff{zulässigen Punkt} (oder zulässige Lösung).
\end{itemize}
Ein zulässiger Punkt $x^\ast \in G$ heißt \begriff{optimal} (oder Lösung oder optimale Lösung), wenn für alle $x \in G$ die Ungleichung
\begin{equation}
	f(x^\ast) \le f(x)
\end{equation}
gilt. Falls das Problem \eqref{eq: oa} lösbar ist, so wird mit $f^\ast = f(x^\ast)$ der \begriff{Optimalwert} bezeichnet. Das Problem \eqref{eq: oa} ist ein
\begin{itemize}[nolistsep, topsep=-\parskip]
	\item \begriff{unrestringiertes} (oder freies) Optimierungsproblem, wenn $G = \Rn$ gilt,
	\item andernfalls (d.h. für $G \neq \Rn$) ein \begriff{restringiertes} Problem
\end{itemize}
und außerdem eine
\begin{itemize}[nolistsep, topsep=-\parskip]
	\item \begriff{diskrete} (oder ganzzahlige) OA (engl. integer program), falls jede Variable eine diskreten Menge angehört
	\item \begriff{kontinuierliche} (oder stetige) OA, falls alle Variablen stetige Werte annehmen
	\item gemischt ganzzahlige OA, wenn sowohl stetige als auch diskrete Variablen vorkommen.
\end{itemize}

Gilt in \eqref{eq: oa} $f(x) = \trans{c} x$ für ein $c \in \Rn$ und ist $G$ durch lineare Bedingungen beschreibbar, so heißt \eqref{eq: oa} \begriff{linear}. In diesem Fall lässt sich \eqref{eq: oa} schreiben als
\begin{equation}
	\trans{c} x \to \min \quad \bei Ax = a, Bx \le b
\end{equation}
mit geeigneten Matrizen $A$ und $B$ sowie Vektoren $a$ und $b$.

Gerade für (gemischt) ganzzahlige OA kann die Lösung der Originalaufgabe schwierig sein. Eine verwandte, jedoch im Allgemeinen leichter zu lösende Aufgabe kann  in diesen Fällen wie folgt erhalten werden:

\begin{definition}
	Wir betrachten die Optimierungsaufgaben
	\begin{itemize}[nolistsep, topsep=-\parskip]
		\item[(P)] $f(x) \to \min \quad \bei x \in D \cap E$
		\item[(Q)] $g(x) \to \min \quad \bei x \in E$
	\end{itemize}
	(Q) heißt \begriff{Relaxation} zu (P) falls $g(x) \le f(x)$ für alle $x \in D \cap E$ gilt. In vielen Fällen wird dabei $g = f$ gewählt.
\end{definition}

Der Optimalwert der Relaxation kann als Näherung (bzw. untere Schranke) für den tatsächlichen Optimalwert von (P) genutzt werden. Meistens liefert die Lösung von (Q) jedoch keinen zulässigen Puntk für (P).

\begin{satz}
	Ist $\quer{x}$ eine Lösung von (Q) und gilt $\quer{x} \in D$ sowie $f(\quer{x}) = g(\quer{x})$, dann löst $\quer{x}$ auch (P).
\end{satz}
\begin{proof}
	siehe Übung
\end{proof}

\begin{definition}
	Seien (Q1) und (Q2) Relaxationen zu (P). (Q1) heißt \begriff{stärker} (oder strenger) als (Q2), wenn die Schranke (d.h. der Optimalwert) von (Q1) größer oder gleich der Schranke (Optimalwert) von (Q2) für jede Instanz von (P) ist.
\end{definition}

\begin{*anmerkung}
	Zur Erklärung des Begriffes ''Instanz`` betrachte das folgende Beispiel.
	\begin{itemize}[nolistsep, topsep=-\parskip]
		\item Problemklasse: $\trans{c} x \to \min$
		\item Instanz der Problemklasse: $x_1 + 2x_2 - 3x_3 \to \min$
	\end{itemize}
	Eine Instanz ist also eine konkrete Belegung.
\end{*anmerkung}
\section{Beispiele zur kontinuierlichen Optimierung}

\subsection{Transportoptimierung}

Es gebe Erzeuger $i \in I = \menge{0, \dots , n}$ und Verbraucher $j \in J = \menge{1, \dots , n}$. Weiterhin seien die Kosten $c_{ij}$ für den Transport einer Einheit von $i$ nach $j$ sowie der Vorrat $a_i > 0$ und der Bedarf $b_j > 0$ für alle $i$ und $j$ gegeben. Wie muss der Transport organisiert werden, damit die Gesamtkosten minimal sind?

Für jedes mathematische Modell einer OA braucht man
\begin{itemize}[nolistsep, topsep=-\parskip]
	\item geeignete Variablen ($\to x$)
	\item Zielfuntkion ($\to f$)
	\item Nebenbedingungen ($\to G$)
\end{itemize}

\begin{description}
	\item[Variablen] $x_{ij} \ge 0$ für alle $i \in I$ und $j \in J$ beschreibe die Einheiten, die von $i$ nach $j$ transportiert werden.
	\item[Zielfunktion] $f(x) = \sum\limits_{i \in I} \sum\limits_{j \in J} c_{ij} x_{ij} \to \min$
	\item[Nebenbedingungen] \leavevmode
	\begin{itemize}[nolistsep, topsep=-\parskip]
		\item Kapazitätsbeschränkung der Erzeuger $i \in I$: $\sum\limits_{j \in J} x_{ij} \le a_i \quad (i \in I)$
		\item Bedarfserfüllung von Verbrauchern $j \in J$: $\sum\limits_{i \in I} x_{ij} \ge b_j \quad (j \in J) $
	\end{itemize}
\end{description}

Somit können wir als Modell formulieren:
\begin{equation*}
	\begin{aligned}
	f(x) = \sum_{i \in I} \sum_{j \in J} c_{ij} x_{ij} \to \min \quad \bei &\sum_{j \in J} x_{ij} \le a_i \enskip (i \in I), \\
	&\sum_{i \in I} x_{ij} \ge b_j \enskip (j \in J), \\
	& x_{ij} \ge 0 \enskip ((i,j) \in I \times J)
	\end{aligned}
\end{equation*}
\section{Beispiele zur diskreten Optimierung}

\subsection{Das Rucksackproblem}

Gegeben seien ein Behälter (''Rucksack``) mit Kapazität $b \in \Z_+ \defeq \menge{0, 1, \dots}$ sowie $m$ Teile, die jeweils durch ein Gewicht $a_i \in \Z_+$ und einen Nutzen $c_i \in \Z_+$ beschrieben werden ($i = 1, \dots , m$). Aus dieser Menge von Objekten ist eine nutzenmaximale Teilmenge auszuwählen.


\begin{description}
	\item[Variablen] \begin{equation*}
		x_i \defeq \begin{cases}
		1 & \text{wenn Teil $i$ eingepackt wird} \\ 0 & \text{sonst}
		\end{cases} \quad (i = 1, \dots , m)
	\end{equation*}
	\item[Zielfunktion] $f(x) = \sum\limits_{i=1}^{m} c_i x_i \to \max$
	\item[Nebenbedingungen] Kapazitätsbedingung: $\sum\limits_{i=1}^{m} a_i x_i \le b$
\end{description}

Als Modell können wir somit formulieren:
\begin{equation*}
	\begin{aligned}
	f(x) = \sum_{i=1}^{m} c_i x_i \to \max \quad \bei \sum_{i=1}^m a_i x_i \le b \und x_i \in \menge{0,1} \enskip (i = 1, \dots , m)
	\end{aligned}
\end{equation*}

Aufgrund der binären Gestalt der Variablen wird das Problem auch als $0/1$-Rucksackproblem bezeichnet. Im Gegensatz dazu ist beim klassischen Rucksackproblem jedes Teil mehrfach nutzbar. In diesem Fall ist $x_i \in \Z_+$ zu fordern.

\subsection{Das Bin-Packing-Problem}

Gegeben seien (sehr große) Anzahl an Behältern der Kapazität $L$ sowie $b_i$ Teile des Gewichts oder Volumens $\ell_i$ mit $i \in I = \menge{1, \dots , m}$. Man ermittle die minimale Anzahl an Behältern, die benötigt wird, um alle Objekte zu verstauen.
Jede Packung (eines Behälters) kann als Vektor $a = (a_1 , \dots , a_m) \in \Z_+^m$ geschrieben werden, wobei $a_i$ angibt, wie oft das Teil $i$ benutzt wird. Ein solcher Vektor ist eine zulässige Packung, wenn 
\begin{equation*}
	\sum_{i=1}^m \ell_i a_i \le L
\end{equation*} 
ist.

\begin{description}
	\item[Modell nach \person{Kantorovich}] Wir benötigen 
	\begin{itemize}[nolistsep]
		\item eine obere Schranke $u \in \Z_+$ für die maximal benötigte Anzahl an Behältern
		\item $y_k = \begin{cases}
		1 & \text{wenn Rucksack } k \text{ benutzt wird} \\ 0 & \text{sonst}
		\end{cases} \quad (k = 1 , \dots , u)$
		\item $x_{ik} \in \Z_+$, die angeben, wieviele Objekte vom Typ $i$ in Rucksack $k$ gepackt werden ($(i,k) \in \menge{1, \dots , m} \times \menge{1, \dots , u}$)
	\end{itemize}
	Daraus ergibt sich nun folgendes Modell:
	\begin{equation*}
		\begin{aligned}
		f^\text{Kant}(x,y) = \sum_{k=1}^u y_k \to \min \bei \quad & \sum_{k=1}^u x_{ik} = b_i \quad &&(i = 1, \dots , m) \\
		& \sum_{i=1}^m x_{ik} \ell_i \le L * y_k \quad &&(k = 1 , \dots , u) \\
		& y_k \in \menge{0,1} \quad &&(k = 1 , \dots , u) \\
		& x_{ik} \in \Z_+ \quad &&((i,k) \in \menge{1, \dots , m} \times \menge{1, \dots , u})
		\end{aligned}
	\end{equation*}
	Die erste Nebenbedingung sorgt dafür, dass alle Teile gepackt werden; die zweite Nebenbedingung liefert die Einhaltung der Kapazität unter Berücksichtigung, dass nur bepackte Behälter gezählt werden.

	Es kann stets $u = \sum_{i=1}^m b_i$ gewählt werden. Das Auffinden besserer Schranken ist im Allgemeinen schwierig.
	Eine Relaxation kann z.B. durch $y_k \in [0,1]$ und $x_{ik} \in \R_+$ erhalten werden. Diese liefert jedoch keine guten Näherungen.
	%
	\item[Modell von Gilmore \& Gomory] Es seien $J$ eine Indexmenge aller zulässigen Packungen und $x_j \in \Z_+$ ($j \in J$) die Häufigkeit, wie oft ein Behälter nach dem durch $j$ angegebenen Schema $a^j = (a_1^j , \dots , a_m^j)$ mit $\trans{\ell} a^j \le L$ gefüllt wird.
	Daraus ergibt sich folgendes Modell:
	\begin{equation*}
		\begin{aligned}
		f^{GG}(x) = \sum_{j \in J} x_j \to \min \quad \bei \quad 
		& \sum_{j \in J} a_i^j * x_j = b_i \quad && (i = 1, \dots , m) \\
		& x_j \in \Z_+ && (j \in J)
		\end{aligned}
	\end{equation*}
	Die Nebenbedingung sorgt dafür, dass alle Teile gepackt werden.
	
	Es gibt im Allgemeinen exponentiell viele zulässige Packungen $a^j$ ($j \in J$), deren Koeffizienten allesamt in den Nebenbedingungen benötigt werden.
	
	Eine Relaxation erhält man zum Beispiel durch $x_j \in \R_+$. Diese stetige Relaxation ist sehr gut; man vermutet, dass 
	\begin{equation*}
		f^{GG, \ast} - f^{GG, \ast}_\text{relax} < 2
	\end{equation*}
	gilt.
\end{description}

Erfreulicherweise gibt es zum Gilmore-Gomory-Modell äquivalente Formulierungen, die mit einer polynomiellen Zahl von Variablen arbeiten und eine ebenso gute stetige Relaxation besitzen (z.B. Flussmodelle).

\chapter{Grundlagen}
\label{chapter_2_grundlagen}
\section{Existenz von Lösungen}

Wir betrachten die Optimierungsaufgabe
\begin{equation}
	f(x) \to \min \quad \bei x \in G \label{eq: 2.1}
\end{equation}
wobei folgende Bedingungen erfüllt seinen:
\begin{itemize}[nolistsep, topsep=-\parskip]
	\item $f$ ist stetig (zumindest auf $G$)
	\item $G$ ist kompakt
	\item $G \neq \emptyset$
\end{itemize}
\vspace{\parskip}

\begin{satz}[Weierstrass] \label{satz: 2.1_weierstrass}
	Unter diesen Voraussetzungen existiert ein $\quer{x} \in G$ mit 
	\begin{equation*}
		f^\ast \defeq f(\quer{x}) \le f(x) \quad \forall x \in G
	\end{equation*}
\end{satz}
\begin{proof}
	Sei $f^\ast \defeq \inf_{x \in G} f(x)$. Wegen $G \neq \emptyset$, finden wir eine Folge $\folge{f_k}{k \in \N} \subseteq \R$ mit $f_k = f(x_k) \ge f^\ast$ mit $x_k \in G$ für alle $k \in \N$ und $\lim_{k \to \infty} f_k = f^\ast$. Die daraus resultierende Folge $\folge{x_k}{k \in \N}$ besitzt wegen der Kompaktheit von $G$ eine konvergente Teilfolge $\folge{\schlange{x_k}}{k \in \N} \subseteq \folge{x_k}{k \in \N}$ mit $\lim_{k \to \infty} \schlange{x_k} = \quer{x} \in G$ (Abgeschlossenheit von $G$). Die Stetigkeit von $f$ ergibt nun $\lim_{k \to \infty} f(\schlange{x_k}) = f(\quer{x}) = f^\ast$ (insbesondere $f^\ast \in \R$)
\end{proof}

\begin{beispiel}
	\begin{enumerate}[nolistsep, leftmargin=*, topsep=-\parskip]
		\item \cref{satz: 2.1_weierstrass} anwendbar ($G$ kompakt, Minimum existiert):
		\begin{equation*}
			f(x_1,x_2) = x_1 - x_2 \to \min \bei x_1^2 + 4x_2^2 \le 1
		\end{equation*}
		Der zulässige Bereich ist eine Ellipse mit Rand.
		\item \cref{satz: 2.1_weierstrass} nicht anwendbar ($G$ unbeschränkt, kein Minimum, $f^\ast = -\infty$):
		\begin{equation*}
			f(x_1,x_2) = x_1 - x_2 \to \min \bei x_1^2 + 4x_2^2 \ge 1
		\end{equation*}
		\item \cref{satz: 2.1_weierstrass} nicht anwendbar ($G$ unbeschränkt, kein Minimum, $f^\ast = 0$)
		\begin{equation*}
			f(x_1,x_2) = \frac{1}{x_1} \to \min \bei x_2 \le \frac{1}{x_1}, x_1 \ge 1, x_2 \ge 0
		\end{equation*}
		\item \cref{satz: 2.1_weierstrass} nicht anwendbar ($G$ unbeschränkt, Minimum existiert, $f^\ast = -1$)
		\begin{equation*}
			f(x_1,x_2) = - \frac{1}{x_1} \to \min \bei x_2 \le \frac{1}{x_1}, x_1 \ge 1, x_2 \ge 0
		\end{equation*}
	\end{enumerate}
\end{beispiel}

Offensichtlich besitzen also nicht alle Optimierungsaufgaben eine (globale) Lösung, insbesondere deshalb, weil Bedingung \eqref{eq: 1.2_optimal} ziemlich stark ist. Stattdessen hat sich in der Literatur auch der folgende ''schwächere`` Lösungsbegriff etabliert.

\begin{definition}
	Ein zulässiger Punkt $\quer{x} \in G$ heißt lokale Lösung von \eqref{eq: 2.1}, falls ein $\rho > 0$ existiert mit 
	\begin{equation*}
		f(\quer{x}) \le f(x) \quad \forall x \in G \cap B_\rho(\quer{x})
	\end{equation*}
	wobei $B_\rho(\quer{x}) \defeq \menge{x \in \Rn : \norm{x - \quer{x}}_2 \le \rho}$ die offene Kugel vom Radius $\rho$ um $\quer{x}$ ist.
\end{definition}

\begin{bemerkung}
	Jede globale Lösung ist auch lokale Lösung. Die Umkehrung ist im Allgemeinen nicht korrekt.
\end{bemerkung}

Sofern eine globale Lösung existiert, ist diese in der Menge der lokalen Lösungen enthalten. Die Betrachtung lokaler Lösungen ist damit im Allgemeinen ausreichend. Für eine spezielle Klasse von Optimierungsaufgaben sind beide Lösungskonzepte sogar äquivalent. Dazu betrachten wir die folgenden Definitionen:

\begin{definition}[Konvexität] %2.2
	\begin{enumerate}[nolistsep]
		\item $G \subseteq \Rn$ ist konvex, falls für alle $x,y \in G$ gilt
		\begin{equation*}
			[x,y] \defeq \menge{x(\lambda) \in \Rn : x(\lambda) = (1-\lambda)x + \lambda y, \lambda \in [0,1]} \subseteq G
		\end{equation*}
		% TODO Abbildung konvexe Menge, nichtkonvexe Menge
		\item Sei $G$ konvex. Die Funktion $\abb{f}{G}{\R}$ heißt konvex, wenn gilt
		\begin{equation*}
			f(x + \lambda (y-x)) \le f(x) + \lambda \brackets{f(y) - f(x)}
		\end{equation*}
		für alle $x,y \in G$ und $\lambda \in [0,1]$.
		\item  Sei $G$ konvex. Eine Funktion $\abb{f}{G}{\R}$ heißt streng konvex, wenn gilt
		\begin{equation*}
			f(x + \lambda (y-x)) < f(x) + \lambda \brackets{f(y) - f(x)}
		\end{equation*}
		für alle $x,y \in G$ und $\lambda \in [0,1]$.
	\end{enumerate}
\end{definition}

%TODO Abbildung konvexe Funktion

Ausgehend von diesen Begrifflichkeiten erhalten wir das folgende Resultat:

\begin{satz} %2.2
	Sei $G \subseteq \Rn$ eine konvexe Menge und $\abb{f}{G}{\R}$ eine konvexe Funktion.
	\begin{enumerate}
		\item Jede lokale Lösung ist gleichzeitig auch globale Lösung von \eqref{eq: 2.1}.
		\item Falls $f$ sogar streng konvex ist, dann existiert höchstens eine Lösung.
	\end{enumerate}
\end{satz}
\begin{proof}
	\begin{enumerate}
		\item Sei $\schlange{x} \in G$ eine lokale Lösung von \eqref{eq: 2.1}. Wir nehmen an, dass dies jedoch keine globale Lösung ist, d.h. es existiert ein $\quer{x} \in G$ mit $f(\quer{x}) < f(\schlange{x})$. Wegen der Konvexität von $G$ gilt dann $x(\lambda) = \schlange{x} + \lambda(\quer{x} - \schlange{x}) \in G$ für alle $\lambda \in [0,1]$. Mit der Konvexität von $f$ folgt letzlich
		\begin{equation*}
			f(x(\lambda)) = f(\schlange{x} + \lambda (\quer{x} - \schlange{x})) 
			\overset{f \text{ konvex}}{\le} f(\schlange{x}) + \underbrace{\lambda}_{> 0} \underbrace{\brackets{f(\quer{x}) - f(\schlange{x})}}_{< 0}
			< f(\schlange{x}) \qquad \forall \lambda \in (0,1]
		\end{equation*}
		Somit ist $\schlange{x}$ keine lokale Lösung im Widerspruch zur Annahme.
		%
		\item Seien $x,y$ zwei voneinander verschiedene Lösungen., d.h. $f(x) = f(y) = f^\ast$. Wir erhalten $x(\lambda) \in G$ für alle $\lambda \in [0,1]$ und
		\begin{equation*}
			f(x(\lambda)) = f(x + \lambda(y-x)) \overset{f \text{ streng konvex}}{<} f(x) + \lambda \underbrace{\brackets{f(y) - f(x)}}_{= 0}
		\end{equation*}
		Somit ist $x$ keine Lösung.
	\end{enumerate}
\end{proof}

Für konvexe Optimierungsaufgaben sind lokale und globale Lösungen also äquivalent. Als wichtigen Spezialfall konvexer Mengen halten wir die folgende Darstellung fest.

\begin{aussage} %2.3
	Sei $G$ gegeben durch
	\begin{equation*}
		G \defeq \menge{x \in \Rn : g_i(x) \le 0, i \in I, h_j(x) = 0, j \in J}
	\end{equation*}
	Dann gilt: falls alle Funktionen $g_i$ ($i \in i$) konvex und alle Funktionen $h_j$ ($j \in J$) affin-linear sind, dann ist $G$ konvex.
\end{aussage}
\begin{proof}
	Seien $x,y \in G$ und $\lambda \in (0,1)$. Zur Klärung der Konvexität, stellt sich die Frage, ob $x(\lambda) \in G$?
	\begin{equation*}
		\begin{aligned}
		g_i(x(\lambda)) &= g_i(x + \lambda (y-x)) \le g_i(x) + \lambda \brackets{g_i(y) - g_i(x)} = \underbrace{1-\lambda}_{>0} \underbrace{g_i(x)}_{\le 0} + \underbrace{\lambda}_{>0} \underbrace{g_i(y)}_{\le 0} \\
		&\le 0 \\
		%
		h_j(x(\lambda)) &= h_j(x + \lambda(y-x)) = A_j(x+\lambda(y-x)) + b_j = (1-\lambda) A_j x + \lambda A_j y + b_j \\
		&= (1-\lambda) [\underbrace{A_j x + b_j}_{h_j(x) = 0}] + \lambda [\underbrace{A_j y + b_j}_{h_j(y) = 0}] \\
		&= 0
		\end{aligned}
	\end{equation*}
	Somit ist $x(\lambda) \in G$ und $G$ also konvex.
\end{proof}

Jeder zulässige Bereich einer linearen Optimierungsaufgabe ($\nearrow$ Kapitel 3) hat diese Gestalt. %TODO ref
\section{Optimalitätsbedingungen}

\begin{definition} %2.3
	\label{definition: 2.3_kegel}
	Eine Menge $K \subseteq \Rn$ heißt \begriff{Kegel}, falls gilt:
	\begin{equation*}
	x \in K \follows \lambda x \in K \quad \forall \lambda \ge 0
	\end{equation*}
	Ein Kegel $K$ ist ein \begriff{konvexer Kegel}, falls $K$ eine konvexe Menge bzw. falls gilt
	\begin{equation*}
	x,y \in K \follows x+y \in K
	\end{equation*}
	für alle $x,y \in K$. Der \begriff{Kegel der zulässigen Richtungen} $Z(\schlange{x})$ ist definiert durch
	\begin{equation*}
	Z(\schlange{x}) \defeq \menge{d \in \Rn \mid \exists \quer{t} \defeq \quer{t}(\schlange{x},d) > 0 \text{ sodass } \schlange{x} + td \in G \enskip \forall t \in [0,\quer{t}]}
	\end{equation*}
\end{definition}

% TODO Kegelbilder einfügen (Punkt - konvex, eine Gerade-konvex, zwei Geraden-mit zwischenteil konvex)

% TODO zulässige Richtungen einfügen

Für Optimierungsaufgaben ist der Kegel der zulässigen Richtungen von großer Bedeutung.

\begin{aussage}[notwendiges Optimalitätskriterium] %2.4
	\label{aussage: 2.4}
	Ist $f$ auf $G$ stetig differenzierbar und $\schlange{x} \in G$ ein lokales Minimum. Dann gilt
	\begin{equation} \label{eq: 2.2}
	\nabla \trans{f(\schlange{x})} * d \ge 0 \qquad \forall d \in Z(\schlange{x})
	\end{equation}
	Ist $G$ konvex, dann erhält man die Bedingung
	\begin{equation} \label{eq: 2.3}
	\nabla \trans{f(\schlange{x})} \brackets{x - \schlange{x}} \ge 0 \qquad \forall x \in G
	\end{equation}
\end{aussage}
\begin{proof}
	Sei $\schlange{x}$ ein lokales Minimum und $d \in Z(\schlange{x})$ eine zulässige Richtung. Dann existiert gemäß Definition ein $\quer{t}$, sodass $\schlange{x} + td \in G$ für alle $t \in [0,\quer{t}]$ gilt. Weil außerdem $\quer{x}$ eine lokale Lösung ist, gibt es $\rho > 0$ mit $\rho < \quer{t}$ sodass $f(\schlange{x} + td) \ge f(\schlange{x})$ für $t \in (0,\rho)$ gilt. Aus dieser Ungleichung folgt
	\begin{equation*}
	\frac{f(\schlange{x} - td) - f(\schlange{x})}{t} \ge 0 \qquad \forall t \in (0,\rho)
	\end{equation*}
	Durch Grenzwertbildung $t \to 0$ auf beiden Seiten erhält man mithilfe der Definition der Richtungsableitung und der Stetigkeit von $f$ die Behauptung \eqref{eq: 2.2}.
	Für konvexe Mengen gilt stets $x - \schlange{x} \in Z(\schlange{x})$ für $x \in G$, also folgt \eqref{eq: 2.3}.
\end{proof}

Dieses Kriterium sagt aus, dass im Punkt $\schlange{x}$ alle Richtungsableitungen (bezüglich zulässiger Richtungen) nicht-negativ sind, d.h. es keine zulässige Abstiegsrichtung gibt.

\begin{bemerkung} %2.2
	Ein Punkt, der die Bedingung \eqref{eq: 2.2} erfüllt, heißt \begriff{stationärer Punkt}.
\end{bemerkung}

\begin{bemerkung} %2.3
	Bei der freien Minimierung (d.h. für $G = \Rn$) ergibt sich wegen $Z(\schlange{x}) = \Rn$ für alle $\schlange{x} \in G$ die notwendige Bedingung
	\begin{equation*}
	\schlange{x} \text{ ist lokales Minimum} \follows \nabla f(\schlange{x}) = 0
	\end{equation*}
	Wähle dafür $d \in \menge{\pm e^i}_{i=1}^n$.
\end{bemerkung}

Für konvexe Optimierungsaufgaben gilt auch die Umkehrung des Resultats der vorherigen Aussage.

\begin{aussage}[hinreichendes Optimalitätskriterium] %2.5
	Es seien $G \subseteq \Rn$ sowie $\abb{f}{G}{\R}$ konvex und stetig differenzierbar. Falls ein $\schlange{x} \in G$ existiert, welches der Bedingung \eqref{eq: 2.3} genügt, dann ist $\schlange{x}$ (globales) Minimum von \eqref{eq: 2.1}.
\end{aussage}
\begin{proof}
	Wenn $f$ konvex und stetig differenzierbar ist und gilt
	\begin{equation*}
	f(x) \ge f(\schlange{x}) + \nabla \trans{f(\schlange{x})} \brackets{x - \schlange{x}} \qquad \forall x \in G
	\end{equation*}
	Wegen \eqref{eq: 2.3} folgt unmittelbar die (globale) Optimalität. Ausführlicher: siehe Übung.
\end{proof}

Im Fall polyedrischer zulässiger Mengen $G \subseteq \Rn$ (wie z.B. in der linearen Optimierung) kann die Bedingung \eqref{eq: 2.2} präzisiert werden, da dann $Z(x)$ eine einfache Struktur besitzt.

\begin{definition} %2.4
	\label{definition: 2.4}
	$G \subseteq \Rn$ heißt \begriff{polyedrisch}, falls eine Darstellung $G = \menge{x \in \Rn \colon Ax \le b}$ für eine geeignete Matrix $A$ und einen geeigneten Vektor $b$ existiert. Hierbei gilt
	\begin{equation*}
	Ax \le b \defequiv \forall i \in I = \menge{1,\dots,n}: \trans{a_i}x = \sum_{j=1}^n a_{ij}x_j \le b_i
	\end{equation*}
\end{definition}

%TODO Bilder Polyeder

\begin{bemerkung} %2.4
	Eine polyedrische Menge $G$ ist konvex und abgeschlossen, aber im Allgemeinen nicht beschränkt. Implizit können in der Beschreibung von $G$ aus \cref{definition: 2.4} auch Gleichungsrestriktionen enthalten sein.
\end{bemerkung}

\begin{definition} %2.5
	Für $x \in G$ ist die \begriff{Indexmenge der aktiven Restriktionen} definiert durch 
	\begin{equation*}
	I_0(x) \defeq \menge{i \in I \colon \trans{a_i} x = b_i}
	\end{equation*}
\end{definition}

%TODO Abbildung aktive Restriktionen

Sei nun ein zulässiger Punkt $x \in G$ gegeben. Damit eine beliebige Richtung $d \in \Rn$ zulässig ist, nuss ein $\quer{t} > 0$ existieren, sodass $x + td \in G$ für alle $t \in [0,\quer{t}]$ gilt.
Für einen polyedrischen Bereich $G$ ist dies äquivalent zu
\begin{equation*}
\forall i \in I \colon \trans{a_i} \brackets{x + td} \le b_i \equivalent \forall i \in I \colon t \trans{a_i} d \le b_i - \trans{a_i} x
\end{equation*}
für alle $t \in [0,\quer{t}]$.

\begin{itemize}[nolistsep]%, topsep=-\parskip]
	\item Für alle inaktiven Restriktionen (also solche $\trans{a_i} x < b_i$) wäre $t \trans{a_i} d \le b_i - \trans{a_i} x$ zu erfüllen. Egal, welchen Wert $\trans{a_i}d$ annimmt, es kann stets eine hinreichend kleine Schrittweite (im Sinne der Definition einer zulässigen Richtung) gefunden werden. Somit schränken inaktive Restriktionen die möglichen Richtungen $d \in \Rn$ \textit{nicht} ein.
	\item Für aktive Restriktionen (also $\trans{a_i} x = b_i$) erhält man $t \trans{a_i} d \le 0$, also (wegen $t > 0$) $\trans{a_i} d \le 0$.
\end{itemize}

Diese Bedingung lässt sich geometrisch interpretieren: 
das Skalarprodukt der zulässigen Richtungen und des Normalenvektors (nach außen gerichtet) $a_i$ der begrenzenden Hyperebene muss kleiner oder gleich Null sein, d.h. der Schnittwinkel beider Vektoren liegt im Bereich $[\frac{\pi}{2},\pi]$. Folglich zeigt die zulässige Richtung $d \in \Rn$ tatsächlich in das Innere von $G$.
%TODO Bild

Für einen zulässigen Punkt $x \in G$ kann somit folgende Beobachtung angegeben werden:
\begin{equation}
d \in Z(x) \equivalent \forall u \in I_0(x) \colon \trans{a_i}d \le 0 \label{eq: 2.4}
\end{equation}
Außerdem ist die Größe $\schlange{t}$ (maximale Schrittweite) wohldefiniert.
\begin{equation}
\schlange{t} \defeq \schlange{t} \defeq \min\menge{\frac{b_i - \trans{a_i}x}{\trans{a_i}d} \colon i \in I(x,d)} \label{eq: 2.5}
\end{equation}
wobei $I(x,d) \defeq \menge{ i \in I \colon \trans{a_i}d > 0}$.

\begin{bemerkung} %2.5
	Falls $I(x,d) = \emptyset$, setzen wir $\schlange{t} \defeq \infty$.
\end{bemerkung}

\begin{beispiel} %2.2
	Wir betrachten $x \defeq \transpose{1,1,1}$ und die polyedrische Menge
	\begin{equation*}
		 G \defeq \menge{\transpose{x_1,x_2,x_3} \in \R^3 \colon x_1 + 2x_2 + x_3 \le 4, 3x_1 + x_2 + x_3 \le 6, x_i \ge 0, i=1,2,3}
	\end{equation*}
	Offenbar gilt $x \in G$. Wir betrachten die Richtungen
	\begin{equation*}
		d^1 = \transpose{1,1,1} \und d^2 = \transpose{-1,-2,-1}
	\end{equation*}
	Als aktive Restriktionen erkennen wir $I_0(x) = \menge{1}$ (da nur die erste Nebenbedingung von $G$ mit Gleichheit erfüllt ist).
	\begin{itemize}
		\item Für $d = d^1$ gilt
		\begin{equation*}
			\trans{a_i} d = \transpose{\begin{smallmatrix} 1 \\ 2 \\ 1	\end{smallmatrix}} \left(\begin{smallmatrix} 1 \\ 1 \\ 1 \end{smallmatrix}\right) = 4 > 0
		\end{equation*}
		Somit ist $d^1$ keine zulässige Richtung wegen \eqref{eq: 2.4}.
		\item Für $d = d^2$ gilt
		\begin{equation*}
		\trans{a_i} d = \trans{\begin{pmatrix} 1 \\ 2 \\ 1	\end{pmatrix}} \begin{pmatrix} -1 \\ -2 \\ -1 \end{pmatrix} = -6 \le 0
		\end{equation*}
		Somit ist $d^2$ eine zulässige Richtung wegen \eqref{eq: 2.4}. Zur maximalen Schrittweite: Die Ungleichung $\transpose{3,1,1}(x + td) \le 6$ liefert die Bedingung $t \in [-\frac{1}{6}, \infty)$. Aus $x+td \ge 0$ folgt die Bedingung $t \le \frac{1}{2}$. Insgesamt gilt $\schlange{t} = \frac{1}{2}$.
	\end{itemize} 
\end{beispiel}

Zusammengefasst erhalten wir das folgende Resultat:

\begin{folgerung} %2.6
	Sei $G$ polyedrisch, d.h. $G = \menge{x \in \Rn \colon Ax \le b}$ und $\abb{f}{G}{\R}$ stetig differenzierbar. Ist $\schlange{x}$ eine lokale Lösung von \eqref{eq: 2.1}, so gilt 
	\begin{equation}
		\nabla \trans{f(\schlange{x})} * d \ge 0 \qquad \forall d \in \Rn  \mit \trans{a_i} d \le 0 \enskip \forall i \in I_0(\schlange{x}) \label{eq: 2.6}
	\end{equation}
	Ist $f$ zusätzlich konvex, dann gilt auch die Umkehrung.
\end{folgerung}
\section{Das Lemma von \person{Farkas}}

Das folgende Resultat besitzt vielfältige Anwendungen in der Optimierung ($\nearrow$ Dualität).

\begin{lemma}[Farkas] %2.7
	Es seien $A \in \R^{m \times n}$ und $a \in \Rm$.  Von den Systemen
	\begin{enumerate}[label=(\roman*), nolistsep, topsep=-\parskip]
		\item $Az \le 0$, $\trans{a} z > 0$
		\item $\trans{A} u = a$, $u \ge 0$.
	\end{enumerate}
	ist \textit{genau} eines lösbar.
\end{lemma}
\begin{proof}
	\begin{itemize}[leftmargin=*, nolistsep]
		\item \textit{höchstens} eines der Systeme ist lösbar: Seien (i) und (ii) lösbar. Dann gilt
		\begin{equation*}
			0 < \trans{a} z = \trans{\trans{A}u} z = \underbrace{\trans{u}}_{\ge 0} \underbrace{A z}_{\le 0} \le 0 \quad \lightning
		\end{equation*}
		\item  \textit{mindestens} eines der Systeme ist lösbar --- die Unlösbarkeit von (ii) impliziert die Lösbarkeit von (i):  Sei (ii) nicht lösbar. Dann gilt
		$a \notin K \defeq \menge{x = \trans{A} u \colon u \ge 0}$, wobei $K$ ein konvexer, abgeschlossener Kegel ist. Wir betrachten die Optimierungsaufgabe
		\begin{equation*}
			f(x) = \frac{1}{2} \norm{a-x}_2^2 = \frac{1}{2} \transpose{a-x}(a-x) \to \min \bei x \in K
		\end{equation*}
		Dann existiert eine eindeutige (und globale) Lösung $\quer{x} \in K$ mit
		\begin{equation*}
			(1) \quad \nabla \trans{f(\quer{x})} \quer{x} = 0 
			\qquad \qquad 
			(2) \quad \nabla \trans{f(\quer{x})} x \ge 0 \enskip \forall x \in K
		\end{equation*}
		Zunächst folgt gemäß \cref{aussage: 2.4}, dass $\nabla \trans{f(\quer{x})} (x - \quer{x}) \ge 0$ für alle $x \in K$. Durch Einsetzen von $x = \frac{1}{2} \quer{x} \in K$ und $x = 2*\quer{x} \in K$ (beachte: $K$ ist Kegel) erhält man (1). Dies wiederum lässt sich zur notwendigen Bedingung ddazu addieren und man erhält (2). 
		Nun zeigen wir, dass $z \defeq a - \quer{x} \neq 0$ (wegen $a \notin K$) das System (i) löst. Es gilt $\nabla f(\quer) = -z$ und damit folgt
		\begin{equation*}
			0 = \nabla \trans{f(\quer{x})} \quer{x} = -\trans{z} * (\quer{x} - a + a) = \trans{z} (z-a)  \follows \trans{a} z = \trans{z} z \overset{z \neq 0}{>} 0
		\end{equation*}
		Weiter gilt $x \in K$ genau dann, wenn ein $u \ge 0$ existiert mit $x = \trans{A} u$. Aus (2) folgt dann 
		\begin{equation*}
			\begin{aligned}
			\nabla \trans{f(\quer{x})} x \ge 0 \enskip \forall x \in K &\follows& - \trans{z} \trans{A} u &\ge 0 \enskip &&\forall u \ge 0 \\
			&\follows& \transpose{Az} u &\le 0 \enskip &&\forall u \ge 0 \\
			&\follows& Az &\le 0 && (\text{wähle z.B. wieder } u = e^1, e^2, \dots)
			\end{aligned}
		\end{equation*}
		Damit löst $z$ das System (1).
	\end{itemize}
\end{proof}

Damit können die notwendigen Optimalitätsbedingungen \eqref{eq: 2.6} bzw. äquivalent dazu
\begin{equation}
	\forall i \in I_0(x) \colon \trans{a_i} * d \le 0 \follows \nabla \trans{f(\schlange{x})} * d \ge 0 \label{eq: 2.7}
\end{equation}
wie folgt umformuliert werden: Offenbar ist \eqref{eq: 2.7} gleichbedeutend mit der Unlösbarkeit von 
\begin{equation*}
	\nabla \trans{f(\schlange{x})} * d < 0, \qquad \trans{a_i} * d \le 0 \enskip \forall i \in I_0(\schlange{x})
\end{equation*}
Wählt man also im Lemma von Farkas $a = - \nabla f(\schlange{x})$ und $A$ bestehend aus den Zeilen $\trans{a_i}$, so folgt die Lösbarkeit des Systems
\begin{equation}
	\nabla f(\schlange{x}) + \sum_{i \in I_0(\schlange{x})} u_i a_i = 0 \qquad (u \ge 0) \label{eq: 2.8}
\end{equation}
Für konvexe Optimierungsaufgaben ist die Lösbarkeit von \eqref{eq: 2.8} sogar äquivalent dazu, dass $\schlange{x}$ Lösung der betrachteten Aufgabe ist.

Gerade im Hinblick auf die praktische Anwendbarkeit ist \eqref{eq: 2.8} in der jetzigen Form wenig hilfreich, da $\schlange{x}$ und damit $I_0(\schlange{x})$ unbekannt sind. Man betrachtet daher oftmals die folgende äquivalente Umformulierung:

\begin{lemma} %2.8
	\label{lemma: 2.8}
	Sei $G \defeq \menge{x \in \Rn \colon \trans{a_i}x \le b_i , i \in I}$ und $\abb{f}{G}{\R}$ stetig differenzierbar. Wenn $x \in \Rn$ Lösung von
	\begin{equation*}
		f(x) \to \min \bei x \in  G
	\end{equation*}
	ist, dann existiert ein Vektor $u$, sodass das Paar $(x,u)$ das folgende System löst:
	\begin{align}
		\begin{split}
		\nabla f(x) + \sum_{i \in I} u_i a_i &= 0 \qquad u_i \ge 0 , \quad \trans{a_i} - b_i \le 0 \quad (i \in I) \\
		u_i \brackets{\trans{a_i}x - b_i} &= 0 \qquad (i \in I)
		\end{split} \label{eq: 2.9}
	\end{align}
	Dabei beschreibt $\trans{a_i} - b_i \le 0$ die Zulässigkeit von $x \in G$ und $u_i \brackets{\trans{a_i}x - b_i} = 0$ gleicht die zu große Indexmenge der Summe wieder aus, d.h. für inaktive Restriktionen folgt $u_i = 0$.
	Ist $f$ konvex, so gilt auch die Umkehrung. Man nennt \eqref{eq: 2.9} auch ein \begriff{KKT-System}.
\end{lemma}

\begin{bemerkung} %2.6
	\begin{enumerate}[nolistsep]
		\item KKT steht für \person{Karush}-\person{Kuhn}-\person{Tucker}.
		\item Die Variablen $u$ heißen \begriff{\person{Lagrange}-Multiplikatoren}.
		\item Gibt es neben den Ungleichungen auch Gleichungsrestriktionen $\trans{a_i}x = b_i$ für $i = m+1, \dots , \quer{m}$ und $\quer{m} > m$, dann erhält man das KKT-System
		\begin{align}
			\begin{split}
			\nabla f(x) + \sum_{i=1}^m u_i a_i + \sum_{i=m+1}^{\quer{m}} u_i a_i &= 0 \\
			u_i \ge 0 , \trans{a_i}x - b_i &\le 0 \qquad (i = 1, \dots , m) \\
			\trans{a_i}x - b_i &= 0 \qquad (i = m+1, \dots , \quer{m}) \\
			u_i \brackets{\trans{a_i}x - b_i} &= 0 \qquad (i = 1 , \dots , m)
			\end{split}
		\end{align}
	\end{enumerate}
\end{bemerkung}

\chapter{Lineare Optimierung}
\label{chapter_3_lineareOptimierung}
Wir betrachten die Optimierungsaufgabe
\begin{equation}
	z = \trans{c} x \to \min \bei x \in G \defeq \menge{x \in \Rn \colon Ax=b, x \ge 0} \label{eq: 3.1}
\end{equation}
mit $A \in \R^{m \times n}$, $c \in \Rn$, $b \in \Rm$. Außerdem nehmen wir an, dass $\rg(A) = m$ gilt und dass $m \le n$ erfüllt ist.

\begin{bemerkung} %3.1
	\begin{enumerate}[nolistsep]
		\item $G$ ist eine polyedrische Menge.
		\item Alle endlich-dimensionalen linearen Optimierungsaufgaben lassen sich \begriff{Standardform} \eqref{eq: 3.1} überführen ($\nearrow$ Übung).
	\end{enumerate}
\end{bemerkung}
\section{Basislösungen und Ecken}

Sei $I \defeq \menge{1, \dots , n}$. Da $\rg(A) = m$, existiert eine Indexmenge $I_B \subseteq I$ mit $\card{I_B} = m$ derart, dass alle Spalten $A^i$ ($i \in I_B$) linear unabhängig sind. $I_B$ wird \begriff{Basis-Indexmenge} genannt. Mit $I_N \defeq I \setminus I_B$ (Nichtbasis) definieren wir
\begin{align*}
A_B &= (A^i)_{i \in I_B} & A_N &= (A^i)_{i \in I_N} \\
c_B &= (c_i)_{i \in I_B} & c_N &= (c_i)_{i \in I_N} \\
x_B &= (x_i)_{i \in I_B} & x_N &= (x_i)_{i \in I_N}
\end{align*}
Dann lässt sich \eqref{eq: 3.1} schreiben als
\begin{equation}
	z = \trans{c_B} x_B + \trans{c_N} x_N \to \min \bei A_B x_B + A_N x_N = b, x_b \ge 0, x_N \ge 0 \label{eq: 3.2}
\end{equation}
bzw. durch Auflösen der Gleichung nach $x_B$ (beachte: $A_B$ hat Vollrang) als
\begin{equation}
	z = \brackets{\trans{c_N} - \trans{c_B} A_B^{-1}A_N} x_N + \trans{c_B} A_B^{-1} b \to \min \bei x_B = -A_B^{-1} A_N x_N + A_B^{-1}b, x_B \ge 0, x_N \ge 0 \label{eq: 3.3}
\end{equation}

\begin{definition} %3.1
	Der Punkt 
	\begin{equation*}
		x = \left(\begin{matrix} x_B \\ x_N \end{matrix} \right) = \left( \begin{matrix} A_B^{-1} b \\ 0 \end{matrix} \right)
	\end{equation*}
	heißt \begriff{Basislösung} zu  $I_B$. Gilt zusätzlich $A_B^{-1}b \ge 0$, dann heißt $x=(x_B,x_N)$ \begriff{zulässige Basislösung}.
\end{definition}

\begin{definition} %3.2
	Der Punkt $x \in G$ heißt \begriff{Ecke} (von $G$), falls aus $x = \frac{1}{2} \brackets{x^1 + x^2}$ mit $x^1,x^2 \in G$ stets $x = x^1 = x^2$ folgt.
\end{definition}

Ecken des zulässigen Bereichs können also nicht durch andere zulässige Punkte linear kombiniert werden.

Zur Wiederholung benennen wir im Folgenden (ohne Beweis) einige Eigenschaften von Ecken und zulässigen Basislösungen.

\begin{satz} %3.1
	Sei $\rg(A) = m$. Dann ist jede zulässige Basislösung auch Ecke von $G$. Umgekehrt gibt es zu jeder Ecke mindestens eine zulässige Basislösung.
\end{satz}

Häufig unterscheidet man zwischen 
\begin{itemize}[nolistsep, topsep=-\parskip]
	\item \begriff{degenerierten} (oder entarteten) Ecken, die mehrere zulässige Basislösungen besitzen
	\item \begriff{nicht-degenerierten} (oder nicht-entarteten) Ecken, die genau eine zulässige Basislösung besitzen.
\end{itemize}
Dabei gilt: Eine Ecke $x \in G$ ist genau dann degeneriert, wenn ein $i \in I_B$ mit $x_i=0$ existiert.

\begin{beispiel} %3.1
	Sei
	\begin{equation*}
		G \defeq \menge{x \in \Rn \colon x_1+x_2+x_3=1, \enskip 2x_1+x_2+x_4=2, \enskip x_1,\dots , x_4 \ge 0}
	\end{equation*}
	Hierbei ist die Ecke $E_1 = \transpose{0,1,0,1}$ nicht degeneriert, da sie nur die Zerlegung $I_B = \menge{2,4}$ und $I_N = \menge{1,3}$ gestattet.
	Die Ecke $E_2 = \transpose{1,0,0,0}$ ist degeneriert, weil ein $i \in I_B$ zwangsläufig $x_i = 0$ erfüllen muss.
\end{beispiel}

\begin{satz} %3.2
	Seo $G \neq \emptyset$. Dann besitzt $G$
	\begin{enumerate}[nolistsep, topsep=-\parskip]
		\item mindestens eine Ecke
		\item höchstens endlich viele Ecken.
	\end{enumerate}
\end{satz}
\begin{proof}
	siehe Übung
\end{proof}

\begin{satz} %3.3
	Ist \eqref{eq: 3.1} lösbar, dann gibt es eine Ecke von $G$, die \eqref{eq: 3.1} löst.
\end{satz}

Bei linearen Optimierungsaufgaben genügt es daher die Ecken von $G$ zu betrachten. Ist die Aufgabe lösbar, so findet man durch systematisches Abschreiten der Ecken eine Lösung.
Um dabei zu erkennen, ob Optimalität vorliegt, hilft folgendes Resultat:

\begin{aussage}[Optimalitätskriterium] %3.4
	\label{aussage: 3.4}
	Gilt für die zuässige Basislösung $x = (x_B, x_N) = (A_B^{-1}b, 0)$ die Bedingung 
	\begin{equation}
		\trans{c_N} - \trans{c_B} A_B^{-1} A_N \ge 0 \label{eq: 3.4}
	\end{equation}
	dann ist $x$ Lösung von \eqref{eq: 3.1}.
\end{aussage}
\begin{proof}
	Sei $x = (x_B, x_N)$ eine zulässige Basislösung. Wir zeigen zunächst:
	\begin{equation*}
		Z(x) \subseteq \menge{d \in \Rn \colon Ad = 0, d_N \ge 0}
	\end{equation*}
	Sei $d \in Z(x)$. Dann existiert $t > 0$ mit $A(x+td) \overset{!}{=} b$ (beachte die Definition von $G$ mit Gleichheitsrestriktionen). Es gilt 
	\begin{equation*}
		Ax + t Ad = b  \equivalent b + t Ad = b \overset{t > 0}{\equivalent} Ad = 0
	\end{equation*}
	Wegen $x_N = 0$ ergibt sich aus $x + td \overset{!}{\ge} 0$ (nach Definition von $G$) sofort $d_N \ge 0$. Insbesondere gilt
	\begin{equation*}
		Ad = 0 \equivalent A_B d_B + A_N d_N = 0 \equivalent d_B = -A_B^{-1} A_N d_N \qquad \forall d \in Z(x)
	\end{equation*}
	Damit folgt unter Berücksichtigung von \eqref{eq: 3.4}
	\begin{align*}
		\nabla \trans{f(x)} d 
		&= \trans{c} d \\
		&= \trans{c_B} d_B + \trans{c_N} d_N \\
		&= -\trans{c_B} A_B^{-1} A_N d_N + \trans{c_N} d_N \\
		&= \underbrace{\brackets{\trans{c_N} - \trans{c_B} A_B^{-1} A_N}}_{\ge 0} \underbrace{d_N}_{\ge 0} \ge 0 \qquad \forall d \in Z(x)
	\end{align*}
	d.h. $x$ genügt der notwendigen Optimalitätsbedingung \eqref{eq: 2.2}, die hier (im konvexen Fall) auch hinreichend ist.
\end{proof}

Eine entsprechende Systematik zum Abschreiten der Ecken wird im Folgenden Abschnitt behandelt.
\section{Das primale Simplex-Verfahren}

Das primale Simplexverfahren durchläuft zwei Phasen (falls nötig):
\begin{itemize}
	\item Phase 1 besteht aus der Ermittlung einer ersten Ecke (zulässige Basislösung), 
	\item Phase 2 aus der darauf aufbauenden Bestimmung einer optimalen Ecke.
\end{itemize} 

\subsection{Phase 2 des Simplex-Verfahrens}

Wir betrachten die (erste) zulässige Basislösung (Ecke) $x = (x_B, x_N)$ und schreiben \eqref{eq: 3.3} als Simplex-Tableau:

\begin{center}
	\begin{tabular}{r|c|c}
		$T_0$ & $x_N$ & $1$ \\ \hline
		$x_B = $ & $P$ & $p$ \\ \hline
		$z =$ & $\trans{q}$ & $q_0$
	\end{tabular}
\end{center}

\begin{equation}
	\begin{aligned}
		P &= -A_B^{-1} A_N  &
		p &= A_B^{-1} b \\
		\trans{q} &= \trans{c_N} - \trans{c_B} A_B^{-1} A_N &
		q_0 &= \trans{c_B} A_B^{-1} b 
	\end{aligned}
	\label{eq: 3.5}	
\end{equation}

%Optimalität erkenn man nun an der Bedingung $\trans{q} \ge 0$.

Wir nehmen zunächst an, dass $x = (x_B, x_N)$ eine nicht-entartete Ecke mit $x_B = \transpose{x_1, \dots, x_m}$ und $x_N = \transpose{x_{m+1}, \dots, x_n}$ ist. Die hierzu gehörige Basislösung ist $x = (x_B, x_N) = (p,0)$ und es gilt $p \ge 0$ (da zulässig). Folglich ist $x \in G$.

\textbf{Frage:} Wenn $x$ nicht optimal ist -- wie kann eine bessere zulässige Basislösung (Ecke) gefunden werden?

\textbf{Antwort:} Wahl einer zulässigen Richtung $d \in Z(x)$ mit maximaler Schrittweite, die eine Verkleinerung des Zielfunktionswerts ermöglicht.

Nach \cref{aussage: 3.4} ist $x$ optimal, falls $q \ge 0$ gilt. Sei nun $q_\tau < 0$ für $\tau \in I_N$. Zur Konstruktion einer neuen Ecke setzen wir $x_\tau = t$ (bisher war $x_\tau = 0$). Dann folgt zunächst $x_N(t) = t * e_\tau$ und wegen der Forderung $x_N(t) \ge 0$ auch $t \ge 0$. Ferner ergibt sich aus Tableau $T_0$ der Zusammenhang $x_i(t) = P_{i \tau} * x_\tau + p_i = P_{i \tau} * t + p_i$ für alle $i \in I_B$.

Insgesamt verfolgen wir ausgehend von $x =(p,0)$ die zulässige Richtung $d \in \Rn$
\begin{equation*}
	d_i = \begin{cases}
	P_{i \tau} & i \in I_B \\
	1 & i = \tau \\
	0 & i \in I_N \setminus \menge{\tau}
\end{cases}
\end{equation*}

Die maximale Schrittweite $\quer{t}$ erhält man wie folgt: Für jedes $i \in I_B$ ist $x_i(t) \ge 0$ zu gewährleisten. Gilt $P_{i \tau} \ge 0$, so ergibt dies keine Einschränkung für die Schrittweite (weil $p_i \ge 0$, $t \ge 0$, $P_{i \tau} \ge 0$ $\follows x_i(t) \ge 0$ für alle $t \ge 0$). Für $P_{i \tau} < 0$ muss hingegen $t \le - \frac{p_i}{P_{i \tau}}$ (aus Tableauzusammenhang) gewählt werden. Die maximal mögliche Schrittweite ergibt sich folglich zu
\begin{equation}
t \le \quer{t} = \quer{t}(x,d) \defeq \min\menge{- \frac{p_i}{P_i \tau} \colon P_{i \tau} < 0, i \in I_B}
\label{eq: 3.6}
\end{equation}
bzw. $\quer{t} = \infty$, falls $P_{i \tau} \ge 0$ für alle $i \in I_B$.

\begin{aussage} %3.5
	Im Fall $\quer{t} = \infty$ besitzt \eqref{eq: 3.1} keine Lösung, da die Zielfunktion nach unten unbeschränkt ist.
\end{aussage}
\begin{proof}
	Wegen $\quer{t} = \infty$ gilt $x(t) \in G$ für alle $t \ge 0$. Dann liefert $q_\tau < 0$ sogleich $Z(t) = \quer{q} * x_N(t) + q_0 \overset{x_N(t) = t * e_\tau}{=} q_\tau * t + q_0 \to - \infty$ für $t \to \infty$.
\end{proof}

\begin{bemerkung} %3.2
	Die beiden Fälle 
	\begin{enumerate}[nolistsep, topsep=-\parskip]
		\item $q_i \ge 0$ für alle $i \in I_N \qquad \leadsto$ Optimalität
		\item es existiert ein $\tau \in I_N$ mit $q_\tau < 0$ und $P_{i \tau} \ge 0$ für alle $i \in I_B \qquad \leadsto$ Unbeschränktheit
	\end{enumerate}
	werden primal entscheidbar genannt.
\end{bemerkung}

Im sogenannten nicht-entscheidbaren Fall, d.h. falls
\begin{equation*}
	\brackets{\exists \ \tau \in I_N \colon a_\tau < 0} \land \brackets{ \exists \ \sigma \in I_B \colon \quer{t} = -\frac{p_\sigma}{P_{\sigma \tau}} = \min \menge{-\frac{p_i}{P_{i \tau}} \colon P_{i \tau < 0, i \in I_B}} < \infty}
\end{equation*}
ergibt die (maximale) Schrittweite $\quer{t}$ den Punkt
\begin{equation*}
	\quer{x} = x + \quer{t} d \in G \mit f(\quer{x}) = f(x) + \quer{t} q_\tau = q_0 + \quer{t} q_\tau
\end{equation*}
Für entartete Ecken kann man die Schrittweite $\quer{t} = 0$ erhalten. In diesem Fall ändert sich der Punkt $\quer{x}$ nicht, aber die Menge $I_N$ und $I_B$. Zum Verlassen einer (noch nicht optimalen) entarteten Ecke können mehrere Schritte nötig sein. 

\begin{satz} %3.6
	\label{satz: 3.6}
	$\quer{x}$ ist eine Ecke von $G$ mit Basis-Indexmenge
	\begin{equation*}
		\begin{aligned}
			\quer{I_B} &= \quer{I_B}(\quer{x}) \defeq \brackets{I_B \setminus \menge{\sigma}} \cup \menge{\tau} \\
			\quer{I_N} &= \quer{I_N}(\quer{x}) \defeq \brackets{I_N \setminus \menge{\tau}} \cup \menge{\sigma}
		\end{aligned}
	\end{equation*}
\end{satz}

Um zu zeigen, dass die Matrix $\quer{A_B} \defeq (A')_{i \in \quer{I_B}}$ regulär ist, nutzen wir das folgende Resultat.

\begin{lemma}[Sherman / Morrison] %3.7
	\label{lemma: 3.7}
	Es seien $B \in \R^{m \times m}$ regulär und $u,v \in \Rm$. Die Matrix $\quer{B} \defeq B + u \trans{v}$ ist genau dann regulär, wenn $1 + \trans{v} B^{-1} u \neq 0$ erfüllt ist und dann gilt
	\begin{equation*}
		\quer{B}^{-1} = B^{-1} - \frac{B^{-1} u \trans{v} B^{-1}}{1 + \trans{v} B^{-1} u}
	\end{equation*}
\end{lemma}
\begin{proof}
	Übung oder Selbststudium (aber wird nie gefragt werden)
\end{proof}

\begin{proof}[\cref{satz: 3.6}]
	Der ''Austausch`` der Spalten $A^\tau$ und $A^\sigma$ kann durch ein dyadisches Produkt $u \trans{v}$ beschrieben werden:
	\begin{equation*}
		\quer{A_B} = A_B + u \trans{v} \mit u = A^\tau - A^\sigma \und v = e^\sigma 
	\end{equation*}
	Wegen
	\begin{align*}
		1 + \trans{v} B^{-1} u &= 1 + \transpose{e^\sigma} A_B^{-1} \brackets{A^\tau - A^\sigma} \\ 
		&= 1 + \transpose{e^\sigma} A_B^{-1} A^\tau - 1 \tag{''$A^\sigma \in A_B$``} \\
		&= - P_{\sigma \tau} \neq 0
		\tag{$P = -A_B^{-1}A_N$ und ''$A^\tau \in A_N$``}
	\end{align*}
	folgt aus \cref{lemma: 3.7} die Regularität von $\quer{A_B}$.
\end{proof}

\begin{beispiel} %3.2
	\label{beispiel: 3.2}
	Wir betrachten $z = -x_1 - x_2 \to \min$ bei $x_1 + 2 x_2 \le 6$, $4x_1 + x_2 \le 10$, $x_1, x_2 \ge 0$. Um aus den Ungleichungen Gleichungsnebenbedingungen zu machen, führen wir sogenannte Schlupfvariablen $x_3, x_4 \ge 0$ ein und erhalten
	\begin{equation*}
		x_1 + 2 x_2 + x_3 = 6, \quad  4x_1 + x_2 + x_4 = 10, \quad x_1, x_2, x_3, x_4 \ge 0 \\
		\follows x_3 = 6 - x_1 - 2x_2 \und x_4 = 10 - 4x_1 + x_2
	\end{equation*}
	Damit liegt nun eine Optimierungsaufgabe in Standardform \eqref{eq: 3.1} vor. Notieren wir dies nun in Tableauform mit $I_N = \menge{1,2}$ und $I_B = \menge{3,4}$ (betrachte dazu $x_B = Px_N + p$):	
	\begin{center}
		\begin{tabular}{r|cc|c l}
			$T_0$ & $x_1$ & $x_2$ & $1$ \\ \cline{1-4}
			$x_3 = $ & $-1$ & $-2$ & $6$ & {\footnotesize $\quer{t} = \frac{6}{1}$} \\ 
			$x_4 = $ & $-4$ & $-1$ & $10$ & {\footnotesize $\quer{t} = \frac{10}{4}$} \\ \cline{1-4}
			$z =$ & $-1$ & $-1$    & $0$
		\end{tabular}
	\end{center}
	Wir können nun $\tau \in \menge{1,2}$ wählen, oBdA wählen wir hier $\tau = 1$.
	Für $x_3$ ergibt sich eine maximale Schittweite $\quer{t} = -\frac{p_i}{P_{i \tau}} = \frac{6}{1}$. Für $x_4$ ergibt sich $\quer{t} = \frac{10}{4}$.  Damit wird $\sigma = 4$ gewählt.
	
	Mit $\tau = 1$ und $\sigma = 4$ sowie $\quer{t} = \frac{5}{2}$ erhält man
	\begin{equation*}
		\quer{x} = x + \quer{t} d = \begin{pmatrix} 0 \\ 0 \\ 6 \\ 10 \end{pmatrix} + \frac{5}{2} \begin{pmatrix} 1 \\ 0 \\ -1 \\-4 \end{pmatrix}
		= \begin{pmatrix} \frac{5}{2} \\ 0 \\ \frac{7}{2} \\ 0 \end{pmatrix}
	\end{equation*}
	und 
	\begin{equation*}
		z(\quer{x}) = -\frac{5}{2}
	\end{equation*}
\end{beispiel}

Der Austausch von $x_\tau$ und $x_\sigma$ im Simplextableau kann formal durch die sogenannten Austauschregeln erfolgen.

\begin{center}
	\begin{tabular}{r|c|c}
		$T_0$ & $x_N$ & $1$ \\ \hline
		$x_B = $ & $P$ & $p$ \\ \hline
		$z =$ & $\trans{q}$ & $q_0$
	\end{tabular}
	$\quad \implies \quad$
	\begin{tabular}{r|c|c}
		$T_1$ & $x_{\schlange{N}}$ & $1$ \\ \hline
		$x_{\schlange{B}} = $ & $\schlange{P}$ & $\schlange{p}$ \\ \hline
		$z =$ & $\trans{\schlange{q}}$ & $\schlange{q_0}$
	\end{tabular}
	\begin{align*}
		\schlange{I_B} &= \brackets{I_B \cup \menge{\tau}} \setminus \menge{\sigma} \\
		\schlange{I_N} &= \brackets{I_B \cup \menge{\sigma}} \setminus \menge{\tau}
	\end{align*}
\end{center}

Austauschregeln:

\begin{align*}
	\schlange{P}_{\sigma , \tau} &\defeq \frac{1}{P_{\sigma,  \tau}} 
	\tag{Pivotelement} \\
	%
	\schlange{P}_{\sigma, j} &\defeq - \frac{P_{\sigma, j}}{P_{\sigma, \tau}} \quad (j \in I_N \setminus \menge{\tau}) & \schlange{p}_\sigma &= - \frac{p_\sigma}{P_{\sigma, \tau}} 
	\tag{Pivotzeile} \\
	%
	\schlange{P}_{i, \tau} &\defeq - \frac{P_{i, \tau}}{P_{\sigma, \tau}} \quad (i \in I_B \setminus \menge{\sigma}) & \schlange{q}_\tau &\defeq \frac{q_\tau}{P_{\sigma, \tau}}
	\tag{Pivotspalte} \\
	%
	\schlange{P}_{i,j} &\defeq P_{i,j} - \frac{P_{\sigma, j}}{P_{\sigma, \tau}} P_{i,\tau} \quad (i \in I_B \setminus \menge{\sigma} , j \in I_N \setminus \menge{\tau}) 
	\tag{sonstige Elemente}\\
	%
	\schlange{p}_i &\defeq p_i - \frac{p_\sigma}{P_{\sigma, \tau}} P_{i,\tau} \quad (i \in I_B \setminus \menge{\sigma}) \\
	\schlange{q}_j &\defeq q_j - \frac{P_{\sigma, j}}{P_{\sigma, \tau}} q_\tau \quad (j \in I_N \setminus \menge{\tau}) \\
\end{align*}
Vergleiche dazu auch das Merkblatt zum Simplex-Verfahren unter \vspace{-\parskip}
\begin{center}
	\url{https://www.math.tu-dresden.de/~martinovic/Zusammenfassung_Simplexverfahren.pdf}
\end{center}

\vspace{\parskip}

\begin{beispiel} %3.3
	Wir betrachten wie in \cref{beispiel: 3.2} die Optimierungsaufgabe $z = -x_1 - x_2 \to \min$ bei $x_1 + 2 x_2 \le 6$, $4x_1 + x_2 \le 10$, $x_1, x_2 \ge 0$. mit Simplex-Starttableau:
	\begin{indentpar}
		\begin{tabular}{R{1.8cm}|R{0.6cm}R{0.6cm}|R{0.6cm} l}
			$T_0$ & $x_1$ & $x_2$ & $1$ \\ \cline{1-4}
			$x_3 = $ & $-1$ & $\fbox{-2}$ & $6$ & {\footnotesize $\quer{t} = 3$} \\ 
			$x_4 = $ & $-4$ & $-1$ & $10$ & {\footnotesize $\quer{t} = 10$} \\ \cline{1-4}
			$z =$ & $-1$ & $-1$    & $0$ \\ \cline{1-4}
			Kellerzeile & $-\frac{1}{2}$ & $\ast$ & $3$ & {\footnotesize = neue Pivotzeile}
		\end{tabular}
	\end{indentpar}
	Nun wählen wir aber $\tau = 2$, woraus sich $\sigma =  3$ ergibt.
	Zur besseren Übersicht haben wir eine Kellerzeile eingeführt. Diese entspricht genau der neu berechneten Pivotzeile.
	\begin{indentpar}
		\begin{tabular}{R{1.8cm}|R{0.6cm}R{0.6cm}|R{0.6cm} l}
			$T_1$ & $x_1$ & \textcolor{cdpurple}{$x_3$} & $1$ \\ \cline{1-4}
			\textcolor{cdpurple}{$x_2 = $} & $- \sfrac{1}{2}$ & $-\sfrac{1}{2}$ & $3$ & {\footnotesize (Division durch -1 * Pivot)} \\ 
			$x_4 = $ & $-\sfrac{7}{2}$ & $\sfrac{1}{2}$ & $7$ \\ \cline{1-4}
			$z =$ & $-\sfrac{1}{2}$ & $\sfrac{1}{2}$  & $-3$ \\ \cline{1-4}
			Kellerzeile & $-\sfrac{2}{7}$ & $\sfrac{1}{7}$ & $2$
		\end{tabular}
	\end{indentpar}
	Nebenrechnung: z.B. $ 7 = 10 + 3 * (-1)$
	
	Im nächsten Schritt wählen wir nun $\tau = 1$ und $\sigma = 4$.
	\begin{indentpar}
		\begin{tabular}{R{1.8cm}|R{0.6cm}R{0.6cm}|R{0.6cm} l}
			$T_2$ & \textcolor{cdpurple}{$x_4$} & $x_3$ & $1$ \\ \hline
			$x_2 = $ & $\sfrac{1}{7}$ & $-\sfrac{4}{7}$ & $2$ & \\ 
			\textcolor{cdpurple}{$x_1 = $} & $-\sfrac{2}{7}$ & $\sfrac{1}{7}$ & $2$ \\ \hline
			$z =$ & $\sfrac{1}{7}$ & $\sfrac{3}{7}$  & $-4$ 
		\end{tabular}
	\end{indentpar}
	Da $\schlange{p} = \brackets{\begin{smallmatrix} 2 \\ 2 \end{smallmatrix}} \ge 0$ ist, ist die Lösung zulässig. Außerdem wissen wir wegen $\trans{\schlange{q}} = \brackets{\sfrac{1}{7}, \sfrac{3}{7}} \ge 0$, dass die Lösung optimal ist. Somit ergibt sich
	\begin{equation*}
		x^\ast = (x_1^\ast, x_2^\ast, x_3^\ast, x_4^\ast) = (2,2,0,0) \mit z^\ast = -4
	\end{equation*}
\end{beispiel}

\subsection{Phase 1 (Hilfsfunktionsmethode)}

Wir betrachten das Problem
\begin{equation}
	z = \trans{c} x \to \min \bei Ax = b , x \ge 0
	\label{eq: 3.7}
\end{equation}
Ohne Einschränkung sei $b \ge 0$. Durch folgendes Hilfsproblem lässt sich eine Startecke ermitteln (sofern eine solche überhaupt existiert). 
\begin{equation}
	h = \trans{e} y \to \min \bei y + Ax = b, x \in \Rn_+, y \in \Rm_+
	\label{eq: 3.8}
\end{equation}
mit $e = \transpose{1, \dots , 1} \in \Rm$.
Eine erste Basislösung für \eqref{eq: 3.8} ist gegeben durch 
\begin{equation}
	\begin{array}{@{}r|c|c@{}}
		T_0 & x & 1 \\ \hline
		y =  & -A & b \\ \hline
		h = & -\trans{e}A & \trans{e} b
	\end{array}
	\label{eq: 3.9}
\end{equation}

\begin{satz} %3.8
	Das Problem \eqref{eq: 3.7} besitzt genau dann eine zulässige Lösung, wenn $h_{\min} = 0$ den Optimalwert von \eqref{eq: 3.8} darstellt.
\end{satz}
\begin{proof}
	Offenbar gilt $h_{\min} = 0 \equivalent y = 0$.
	\begin{proof-equivalence}
		\hinrichtung Besitzt \eqref{eq: 3.7} eine zulässige Lösung $\schlange{x}$, dann ist $\left( \begin{smallmatrix} \schlange{x} \\ 0 \end{smallmatrix} \right)$ zulässig für \eqref{eq: 3.8}. Wegen $0 \le h = \trans{e} \schlange{y} = 0$ folgt $h_{\min} = 0$.
		\rueckrichtung Hat man umgekehrt $h_{\min} = 0$, so gilt $\schlange{y} = 0$ für jede optimale Lösung $\left( \begin{smallmatrix} \schlange{x} \\ \schlange{y} \end{smallmatrix} \right)$ von \eqref{eq: 3.8}. Aus der Zulässigkeit von $\left( \begin{smallmatrix} \schlange{x} \\ \schlange{y} \end{smallmatrix} \right)$ für \eqref{eq: 3.8} folgt dann die Zulässigkeit von $\schlange{x}$ für \eqref{eq: 3.7}.
	\end{proof-equivalence}
\end{proof}

\subsection{Der Simplexalgorithmus}

Mit den zuvor beschriebenen Vorgehensweisen lässt sich das Simplexverfahren zur Lösung der Optimierungsaufgabe \eqref{eq: 3.1} wie folgt algorithmisch formulieren:
\begin{itemize}
	\item \textbf{Schritt 1} (\textit{Initialisierung}): Ermittle eine erste zulässige Basislösung $x = \transpose{\trans{x_B}, \trans{x_N}} = \transpose{\trans{p}, \trans{0}}$ mit $p = (A_B)^{-1} b \ge 0$, wobei $I_B$ die Menge der Basisindizes ist und stelle ein erstes Simplextableau auf.
	\item \textbf{Schritt 2} (\textit{Optimalitätstest}): Berechne entsprechend \cref{aussage: 3.4}
	\begin{equation}
		\quer{q} \defeq \min_{j \in I_N} q_j \quad \mit \quad q_J \defeq c_j \trans{d} A^j \enskip (j \in I_N)
		\label{eq: 3.10}
	\end{equation}
	wobei $\trans{d} \defeq \trans{c_B} (A_B)^{-1}$ ist. Gilt $\quer{q} \ge 0$, dann ist $x$ Lösung von \eqref{eq: 3.1}. Andernfalls sei $q_\tau = \quer{q} < 0$.
	\item \textbf{Schritt 3} (\textit{Test auf Unbeschränktheit}): Gilt $P_{i \tau} \ge 0$ für alle $I \in I_B$, so ist die Aufgabe nicht lösbar ($f^\ast = -\infty$).
	\item \textbf{Schritt 4} (\textit{Austauschschritt}): Bestimme die Pivotzeile $\sigma$ gemäß
	\begin{equation*}
		- \frac{p_\sigma}{P_{\sigma \tau}} = \min \menge{- \frac{p_i}{P_{i \tau}} \colon \quer{P}_{i \tau} < 0, i \in I_B}
	\end{equation*}
	und führe den Austauschschritt $\sigma \leftrightarrow \tau$ (Aktualisierung Simplextableau) durch. Gehe zu Schritt 2.
\end{itemize}

\begin{bemerkung} %3.3
	\begin{enumerate}[label=(\roman*), nolistsep, topsep=-\parskip]
		\item Der Simplexalgorithmus löst Problem \eqref{eq: 3.1} nach endlich vielen Schritten exakt oder stellt dessen Unlösbarkeit fest.
		\item Pro Simplexschritt ist im Wesentlichen die Matrix $P$ (der Dimension $m \times (n-m)$ ) zu transformieren. Für $n \gg m$ kann das recht aufwendig sein, sodass ggf. alternative Varianten des Simplexalgorithmus' (z.B. das revidierte Simplexverfahren oder die Technik der Spaltengenerierung) effizienter sind.
		\item Der Test auf Unbeschränktheit der Zielfunktion kann auch für jede Spalte $j \in I_N$ mit $q_j < 0$ erfolgen, sofern dies nicht zu aufwendig ist.
	\end{enumerate}
\end{bemerkung}
\section{Das duale Simplexverfahren}

Nach \cref{aussage: 3.4} ist ein Tableau
\begin{center}
	\begin{tabular}{r|c|c}
		$T_0$ & $x_N$ & $1$ \\ \hline
		$x_B = $ & $P$ & $p$ \\ \hline
		$z =$ & $\trans{q}$ & $q_0$
	\end{tabular}
\end{center}
optimal, wenn $p \ge 0$ und $q \ge 0$ gelten. Nach Konstruktion gilt beim primalen Simplexverfahren stets $p \ge 0$.
Sei nun ein Tableau $T_0$ gegeben mit $q \ge 0$, aber \textit{nicht} $p \ge 0$, d.h. es gibt eine Zeile $\sigma \in I_B$ mit $p_\sigma < 0$.
Die zu $T_0$ gehörige Basislösung ist dann \textit{nicht} zulässig. Mithilfe des dualen Simplexverfahrens lässt sich jedoch (unter Beibehaltung von $q \ge 0$) eine zulässige Basislösung (d.h. mit "$p \ge 0$) erzeugen.
Entsprechend der bekannten Austauschregeln ergeben sich folgende Bedingungen:
\begin{equation*}
	\begin{alignedat}{2}
		\schlange{q}_j &\defeq q_j - \frac{P_{\sigma, j}}{P_{\sigma, \tau}} q_\tau &\overset{!}&{\ge} 0 \qquad \forall j \in I_N \setminus \menge{\tau} \\
		\schlange{q}_\tau &\defeq \frac{q_\tau}{P_{\sigma, \tau}} &\overset{!}&{\ge} 0  \\
		\schlange{p}_\sigma &\defeq - \frac{p_\sigma}{P_{\sigma, \tau}} &\overset{!}&{\ge} 0
	\end{alignedat}
\end{equation*}
Wegen $p_\sigma < 0$ und $q_\tau \ge 0$ ist somit ein Pivotelement mit $P_{\sigma, \tau} > 0$ zu wählen. Zur Sicherstellung von $\schlange{q}_j \ge 0$ für alle $j \in I_N \setminus \menge{\tau}$ muss ferner gelten
\begin{equation*}
	\frac{q_\tau}{P_{\sigma, \tau}} = \min \menge{\frac{q_j}{P_{\sigma, j}} \colon P_{\sigma, j} > 0, j \in I_N}
\end{equation*}
Die eigentlichen Austauschregeln sind analog zu denen des primalen Simplexverfahrens.

\begin{bemerkung} %3.4
	Da dieses Verfahren mit einem \textit{unzulässigen} Punkt startet, ist die Folge der Zielfunktionswerte (im Gegensatz zum primalen Simplexverfahren) nicht monoton fallend.
\end{bemerkung}

\begin{bemerkung} %3.5
	Falls eine zulässige Basislösung gefunden wird, so ist diese zwangsläufig optimal.
\end{bemerkung}

\begin{beispiel}
	Betrachten wir die Optimierungsaufgabe
	\begin{equation*}
		\begin{aligned}
			z = 6x_1 + 5x_2 + 12x_3 + 8x_4 + 9x_5 \to \min \quad \bei \quad 
			x_1 + x_3 + x_4 + x_5 &\ge 300, \\
			x_2 + 2x_3 + x_4 &\ge 400, \\
			x_i &\ge 0 \qquad \forall i = 1, \dots, 5
		\end{aligned}
	\end{equation*}
	Um daraus Gleichungsrestriktionen zu machen, führen wir Schlupfvariablen $x_6, x_7 \ge 0$ ein, d.h.
	\begin{equation*}
		\begin{aligned}
			x_1 + x_3 + x_4 + x_5 &\ge 300 + x_6 , \\
			x_2 + 2x_3 + x_4 &\ge 400 + x_7, \\
			x_i &\ge 0 \qquad \qquad \forall i = 1, \dots, 7
		\end{aligned}
	\end{equation*}
	Daraus ergibt sich nun folgendes Tableau
	\begin{indentpar}
		\begin{tabular}{R{1cm}|R{.5cm}R{.5cm}R{.5cm}R{.5cm}R{.5cm}|R{.9cm}l}
			$T_0$ & $x_1$ & $x_2$ & $x_3$ & $x_4$ & $x_5$ & $1$ \\ \cline{1-7}
			$x_6 =$ & $1$ & $0$ & $1$ & $1$ & $1$ & $-300$ \\
			$x_7 =$ & $0$ & \fbox{$1$} & $2$ & $2$ & $0$ & $-400$ \\ \cline{1-7}
			$z =$   & $6$ & $5$ & $12$ & $8$ & $9$ & $0$ & $\leftarrow \sigma = 7$ \\ \cline{1-7}
			Keller  & $0$ & $\ast$ & $-2$ & $-1$ & $0$ & $400$
		\end{tabular}
	\end{indentpar}
	
	Zur Wahl von $\tau = 2$: $\frac{5}{1}$, $\frac{12}{6} = 6$, $\frac{8}{1} = 8$. Dabei ist $5$ minimal, also $\tau = 2$.
	Fahren wir nun mit den weiteren Tableaus fort:
	
	\begin{indentpar}
		\begin{tabular}{R{1cm}|R{.5cm}R{.5cm}R{.5cm}R{.5cm}R{.5cm}|R{.9cm}l}
			$T_1$   & $x_1$ & $x_7$  & $x_3$      & $x_4$ & $x_5$ & $1$ \\ \cline{1-7}
			$x_6 =$ & $1$   & $0$    & \fbox{$1$} & $1$   & $1$   & $-300$ & $\leftarrow \sigma = 6$ \\
			$x_2 =$ & $0$   & $1$    & $-2$       & $-1$  & $0$   & $400$ \\ \cline{1-7}
			$z =$   & $6$   & $5$    & $2$        & $3$   & $9$   & $2000$ \\ \cline{1-7}
			Keller  & $0$   & $\ast$ & $\ast$ & $-1$  & $0$   & $400$
		\end{tabular}

		\begin{tabular}{R{1cm}|R{.5cm}R{.5cm}R{.5cm}R{.5cm}R{.5cm}|R{.9cm}l}
			$T_2$   & $x_1$ & $x_7$ & $x_6$ & $x_4$      & $x_5$ & $1$ \\ \cline{1-7}
			$x_3 =$ & $1$   & $0$   & $1$   & $-1$       & $1$   & $300$  \\
			$x_2 =$ & $2$   & $1$   & $-2$  & \fbox{$1$} & $2$   & $-200$ & $\leftarrow \sigma = 2$ \\ \cline{1-7}
			$z =$   & $4$   & $5$   & $2$   & $1$        & $7$   & $2600$ \\ \cline{1-7}
			Keller  & $-2$  & $-1$  & $2$   & $\tau = 4$ & $-2$  & $200$
		\end{tabular}
	
		\begin{tabular}{R{1cm}|R{.5cm}R{.5cm}R{.5cm}R{.5cm}R{.5cm}|R{.9cm}l}
			$T_3$   & $x_1$ & $x_7$ & $x_6$ & $x_2$ & $x_5$ & $1$ \\ \cline{1-7}
			$x_3 =$ & $1$   & $1$   & $-1$   & $-1$  & $1$   & $100$  \\
			$x_4 =$ & $-2$   & $-1$   & $2$  & $1$   & $-2$   & $200$ \\ \cline{1-7}
			$z =$   & $2$   & $4$   & $4$   & $1$   & $5$   & $2800$ \\
		\end{tabular}
	\end{indentpar}

	Somit ergibt sich die Lösung
	\begin{equation*}
		x^\ast = \transpose{0,0,100,200,0,0,0} \quad \und \quad z^\ast = 2800
	\end{equation*}
\end{beispiel}
\section{Dualität}

Wir betrachten nun die Optimierungsaufgabe
\begin{equation*}
	\begin{aligned}
		\qquad &\trans{c} x \to \min \bei Ax \le b \und x \in \Rn_+ \\
		&I \defeq \menge{1, \dots, m} \und J \defeq \menge{1, \dots, n}
	\end{aligned} \tag{P}
	% \label{eq: 3.11}
	\label{eq: P}
\end{equation*}

\begin{satz}[Charakterisierungssatz] %3.9
	Ein Punkt $x \in \Rn$ ist genau dann Lösung von (P), wenn ein $\quer{x} \in \Rm$ existiert, sodass insgesamt das folgende System gelöst wird:
	\begin{equation*}
		\begin{alignedat}{3}
			A \quer{x} - b &\le 0 \quad &\quer{x} &\ge 0 \qquad \qquad &(1) \\
			\trans{A} \quer{u} + c &\ge 0 \quad &\quer{u} &\ge 0 &(2) \\
			\trans{\quer{u}} \brackets{A \quer{x} - b} &= 0 \qquad \qquad &\trans{\quer{x}} \brackets{\trans{A}\quer{u} + c} &= 0 &(3)
		\end{alignedat}
	\end{equation*}
\end{satz}
\begin{proof}
	Die vorliegende Optimierungsaufgabe ist äquivalent zu
	\begin{equation*}
		f(x) = \trans{c} x \to \min \bei \underbrace{\begin{pmatrix} A \\ - \one_n \end{pmatrix}}_{= \schlange{A} \in \R^{(m + n) \times n}} x \le \underbrace{\begin{pmatrix} b \\ 0 \end{pmatrix}}_{= \schlange{b} \in \R^{m+n}}
		\tag{P'}
		\label{eq: P'}
	\end{equation*}
	Gemäß \cref{lemma: 2.8} ist $x$ genau dann Lösung von \eqref{eq: P'} (und \eqref{eq: P}), wenn ein Vektor $w = \left( \begin{smallmatrix} u \\ v \end{smallmatrix} \right) \in \R^{m + n}$ existiert mit 
	\begin{equation*}
		\begin{alignedat}{2}
			\nabla f(x) + \sum_{i \in I \cup J} w_i \schlange{a}_i &= 0 \\
			w_i &\ge 0 \qquad \qquad &&(i \in I \cup J) \\
			\trans{\schlange{a}_i} x - \schlange{b}_i &\le 0 &&(i \in I \cup J) \\
			w_i \brackets{\trans{\schlange{a}_i} x - b_i} &= 0 &&(i \in I \cup J)
		\end{alignedat}
		\tag{KKT}
	\end{equation*}
	Trennung von $I$ und $J$ führt zu
	\begin{equation*}
		c + \sum_{i \in I} u_i a_i + \sum_{j \in J} v_j (-e^j) = 0
		\tag{KKT}
	\end{equation*}
	\begin{equation*}
		\begin{alignedat}{4}
			u_i &\ge 0 \qquad &&(i \in I) 											& v_j &\ge 0 \qquad &&(j \in J) \\
			\trans{a_i} x - b_i &\le 0 &&(i \in I) 									& \transpose{-e^j} x - 0 &\le 0 &&(j \in J) \\
			u_i \brackets{\trans{a_i} x - b_i} &= 0 &&(i \in I) \qquad\quad	& v_j \brackets{\transpose{-e^j} x - 0} &= 0 &&(j \in J)  \\
		\end{alignedat}
	\end{equation*}
	Überführt man dieses System in eine Matrix-Vektor-Schreibweise, so ergibt sich
	\begin{equation*}
		\begin{aligned}
			c + \trans{A} u - v &= 0 \\
			u,v,x &\ge 0 \\
			Ax - b &\le 0 \\
			\trans{u} \brackets{Ax - b} &= 0 \\
			\trans{x} v &= 0
		\end{aligned}
	\end{equation*}
	Durch Umstellen der ersten Gleichung nach $v$ lässt sich diese Variable im System ''eliminieren`` und wir erhalten die Behauptung.
\end{proof}
\section{Transportoptimierung}

\subsection{Problemstellung}

\textbf{Zur Erinnerung:}
Es gebe Erzeuger $i \in I = \menge{0, \dots, r}$ und Verbraucher $k \in K = \menge{1, \dots, s}$. Weiterhin seien die Kosten $c_{ik}$ für den Transport einer Einheit von $i$ nach $k$ sowie der Vorrat $a_i > 0$ und der Bedarf $b_k > 0$ für alle $i \in I$ und $k \in K$ bekannt. Wie ist der gesamte Transport kostenminimal zu gestalten.

Als Variablen verwenden wir die Transportmenge $x_{ik}$ von $i$ nach $k$.
\begin{equation}
\begin{alignedat}{3}
	z = \sum_{i \in I} \sum_{k \in K} c_{ik} x_{ik} \to \min \quad \bei \quad 
	&\sum_{k \in K} x_{ik} &= a_i \quad &(i \in I) \\
	&\sum_{i \in I} x_{ik} &= b_k \quad &(k \in K) \\
	& x_{ik} &\ge 0 \quad &(i,k) \in I \times K \\
\end{alignedat}
\label{eq: 3.13}
\end{equation}

Mit   
\begin{equation*}
	\begin{aligned}
		x &= \transpose{x_{11}, x_{12}, \dots, x_{1s}, x_{21}, \dots, x_{rs}} \\
		c &= \transpose{c_{11}, c_{12}, \dots, c_{1s}, c_{21}, \dots, c_{rs}} \\
		\quer{b} &= \transpose{a_1, \dots, a_r, b_1, \dots, b_s}
	\end{aligned}
\end{equation*}
hat \eqref{eq: 3.13} die Form
\begin{equation*}
	z = \trans{c} x \to \min \bei Ax = \quer{b}, x \ge 0
\end{equation*}

\begin{bemerkung} %3.6
	Das Transportproblem ist eine sehr spezielle Optimierungsaufgabe. Dei Koeffizientenmatrix 
	\begin{equation*}
		A = 
		\left( \begin{array}{cccc|cccc|cccc}
		1 & 1 & \cdots & 1 &   &   &        &   &   &   &        &\\
		  &   &        &   & 1 & 1 & \cdots & 1 &   &   &        &\\
		  &   &        &   &   &   &        &   & 1 & 1 & \cdots & 1  \\
		\hline
		1 &   &        &   & 1 &   &        &   & 1 &   &        &  \\
		  & \multicolumn{2}{c}{\ddots} & & & \multicolumn{2}{c}{\ddots} & & & \multicolumn{2}{c}{\ddots} & \\
		  &   &        & 1 &   &   &        & 1 &   &   &        & 1  \\	  
		\end{array} \right) \in \R^{(r+s) \times (r*s)}
	\end{equation*}
	ist schwach besetzt. Insbesondere hat die Spalte von $A$, die zur Variablen $x_{ik}$ gehört, die Gestalt $A^{ik} = \begin{psmallmatrix} e^i \\ e^k \end{psmallmatrix} \in \R^{r+s}$.
\end{bemerkung}

\begin{satz} %3.13
	Das Transportproblem ist genau dann lösbar, wenn die Sättigungsbedingung
	\begin{equation}
		\sum_{i \in I} a_i = \sum_{k \in K} b_k
		\label{eq: 3.15}
	\end{equation}
	gilt.
\end{satz}
\begin{proof}
	Wir zeigen zuerst, dass \eqref{eq: 3.15}  äquivalent zu $G \neq \emptyset$ ist.
	\begin{itemize}
		\item Einerseits folgt aus $x \in G \neq \emptyset$ durch Summation der Gleichungsnebenbedingungen
		\begin{equation*}
			\sum_{i \in I} a_i = \sum_{i \in I} \sum_{k \in K} x_{ik} = \sum_{k \in K} \sum_{i \in I} x_{ik} = \sum_{k \in K} b_k
		\end{equation*}
		\item  Gilt hingegen \eqref{eq: 3.15}, so ist mit $\sigma \defeq \sum_{i \in I} a_i = \sum_{k \in K} b_k$ ein zulässiger Punkt $x = (x_{ik})$ wie folgt definiert:
		\begin{equation*}
			x_{ik} \defeq \frac{a_i b_k}{\sigma}
		\end{equation*}
	\end{itemize}
	Der zulässige Bereich $G$ ist polyedrisch (und damit abgeschlossen) und ferner wegen $0 \le x_{ik} \le \min\menge{a_i, b_k}$ beschränkt und somit kompakt.
	Mit dem Satz von Weierstraß\footnote{Die Zielfunktion ist linear, d.h. stetig, auf einer komapkten Menge $G$.} folgt dann die Lösbarkeit des Transportproblems.
\end{proof}

Die Systemmatrix $A$ besitzt für praxisrelevante Problemgrößen eine sehr große Anzahl an Einträgen, sodass die Anwendung des Simplexverfahrens im Allgemeinen nicht empfehlenswert ist; insbesondere deshalb, weil dieses die Struktur von $A$ nicht mit einbezieht.

Zur Lösung des Transportproblems hat sich daher ein Verfahren etabiliert, das auch die duale Aufgabe 
\begin{equation}
	w \defeq \trans{a} u + \trans{b} v \to \max \bei \trans{A} \begin{psmallmatrix} u \\ v \end{psmallmatrix} \le c, u \in \R^r, v \in \R^s
	\label{eq: 3.16}
\end{equation}
bzw.
\begin{equation}
	w \defeq \sum_{i \in I} a_i u_i + \sum_{k \in K} b_k v_k \to \max \bei u_i + v_k \le c_{ik} \quad (u_i, v_k \in \R, i \in I, k \in K)
	\label{eq: 3.17}
\end{equation}

\begin{satz}[Optimalitätskriterium] %3.14
	\label{satz: 3.14}
	Sei $x \in G$, d.h. $x$ ist zulässiger Transportplan, dann gilt
	\begin{equation*}
		\begin{aligned}
			x \text{ optimal} \equivalent \exists u \in \R^r, v \in \R^s \mit u_i + v_k \le c_{ik}, x_{ik} * (c_{ik} - u_i - v_k) = 0 \quad \forall i \in I, k \in K
		\end{aligned}
	\end{equation*}
\end{satz}
\begin{proof}
	Nach dem Charakterisierungssatz gilt: $x \in G$ ist genau dann optimal, wenn duale Variablen $u \in \R^r$ und $v \in \R^s$ existieren, sodass $(u,v)$ dual zulässig ist und die Komplementaritätsbedingungen gelten.
\end{proof}

\begin{aussage} %3.15
	Der Rang von $A$ ist $\card{I} + \card{K} - 1 = r+s-1$.
\end{aussage}
\begin{proof}
	Einerseits sind die Spalten $A^{11}, \dots, A^{1s}, A^{21}, A^{31}, \dots, A^{r1}$ von $A$ linear unabhängig, d.h. $\rg(A) \ge r + s - 1$.
	Andererseits ist die Summe der ersten $r$ Zeilen identisch mit der Summe der letzten $s$ Zeilen. Somit ist der Rang von oben beschränkt mit $\rg(A) \le r + s -1$.
\end{proof}

\begin{folgerung} %3.16
	Jede Ecke des zulässigen Bereichs $G$ hat höchstens $r+s-1$ positive Komponenten.
\end{folgerung}

\begin{definition} %3.4
	Eine Folge von Zellen (Indexpaaren) $(i_1, k_1), (i_2, k_1), (i_2, k_2) \dots (i_\ell, k_\ell), (i_1, k_\ell), (i_1, k_1)$ mit $i_\nu \neq i_\mu$, $k_\nu \neq k_\mu$ für $\nu \neq \mu$ heißt \begriff{Zyklus} (der Länge $2 \ell$).
\end{definition}

\begin{beispiel} %3.5
	Im folgenden Schema ist ein Zyklus der Länge $2 \ell = 8$ abgebildet:
	\begin{center}
		\begin{tabular}{l|ccccc}
			$i / k$ & 1 & 2 & 3 & 4 & 5 \\ \hline
			1       & $\ast$ & & $\ast$ & & \\
			2 & & $\ast$ & & & $\ast$ \\
			3 & $\ast$ & & & & $\ast$ \\
			4 & & $\ast$ & $\ast$ & & 
		\end{tabular}
	\end{center}
	Dieses Beispiel spielt eine wichtige Rolle bei der Feststellung, ob ein gegebener Transportplan eine Ecke von $G$ ist.
\end{beispiel}

\begin{aussage} %3.17
	\begin{enumerate}[label=(\roman*), nolistsep, topsep=-\parskip]
		\item Sei $J$ eine Menge von Zellen. Gilt $\card{J} \ge r+s$, so enthält $J$ mindestens einen Zyklus.
		\item Sei $x = (x_{ik})$ ein zulässiger Transportplan. $x$ ist genau dann eine Ecke von $G$, wenn $J_+ \defeq \menge{(i,k) : x_{ik} > 0}$ keinen Zyklus enthält.
	\end{enumerate}
\end{aussage}
\begin{proof}
	vielleicht in der Übung --- oder auch nicht.
\end{proof}

\subsection{Erzeugung eines ersten Transportplans}
Dieser Teil entspricht der ersten Phase des Simplexverfahrens, d.h. also der Bestimmung einer Startecke.
Gemäß der vorherigen Beobachtungen genügt es einen zyklenfreien zulässigen Transportplan zu finden. Hierfür können unterschiedliche Methoden genutzt werden.

\begin{description}
	\item[Nordwest-Ecken-Regel:] Die jeweilige noch nicht belegte Nordwest-Zelle wird mit maximaler Transportmenge belegt.
	\item[Regel der minimalen Kosten:] In jedem Schritt wird eine noch nicht belegte Zelle, die minimale Kosten hat, mit maximaler Transportmenge belegt.
	\item[Methode von Vogel:] Bestimme in jeder Zeile und Spalte die Differenz der zwei kleinsten Kostenkoeffizienten der noch freien Zellen. Wähle dann eine Zeile/Spalte mit maximaler Differenz und belege die Zelle mit kleinsten Kosten.
\end{description}

Darstellung der Inputdaten oder zulässigen Punkte in folgenden Schemata:

\begin{minipage}{\dimexpr0.5\linewidth-\fboxrule-\fboxsep}
	\centering
	\begin{tabular}{c|cccc}
		C & $b_1$ & $b_2$ & $\dots$ & $b_s$ \\ \hline
		$a_1$ & $c_{11}$ & $c_{12}$ & $\dots$ & $c_{1s}$ \\
		$\vdots$  & $\vdots$ & $\vdots$ & $\ddots$ & $\vdots$ \\
		$a_r$ & $c_{r1}$ & $c_{r2}$ & $\dots$ & $c_{rs}$
	\end{tabular}
	\captionof{table}{Inputdaten}
\end{minipage}
\begin{minipage}{\dimexpr0.5\linewidth-\fboxrule-\fboxsep}
	\centering
	\begin{tabular}{c|cccc}
		X & $b_1$ & $b_2$ & $\dots$ & $b_s$ \\ \hline
		$a_1$ & $x_{11}$ & $x_{12}$ & $\dots$ & $x_{1s}$ \\
		$\vdots$  & $\vdots$ & $\vdots$ & $\ddots$ & $\vdots$ \\
		$a_r$ & $x_{r1}$ & $x_{r2}$ & $\dots$ & $x_{rs}$
	\end{tabular}
	\captionof{table}{Transportplan}
\end{minipage}




\begin{beispiel} %3.6
	\label{beispie: 3.6}
	Gegeben Sei das folgende Transportproblem:
	\begin{center}
		\begin{tabular}{r|rrrrr}
			C & 12 &  5 &  6 &  7 &  7 \\ \hline
			4 & 12 &  6 & 10 &  9 &  5 \\
			19 & 10 & 16 & 17 &  3 &  7 \\
			14 &  4 & 11 &  5 &  8 & 10
		\end{tabular}
	\end{center}
	
	Man erhält folgende Startecken:
	\begin{center}
		\begin{tabular}{r|rrrrr}
			$\text{X}_{\text{NW}}$ & \cancel{12} \cancel{8} 0 &  \cancel{5} 0 &  \cancel{6} 0 &  \cancel{7} 0 &  \cancel{7} 9 \\ \hline
			0 &  4 &  0 &  0 &  0 &  0 \\
			0 \cancel{6} \cancel{11} \cancel{19} &  8 & 5 & 6 & 0 & 0 \\
			0 \cancel{7} \cancel{14} &  0 & 0 &  0 &  7 & 7
		\end{tabular}
	\end{center}

	\begin{center}
		\begin{tabular}{r|rrrrr}
			$\text{X}_{\text{NW}}$ & \cancel{12} \cancel{8} 0 &  \cancel{5} 0 &  \cancel{6} 0 &  \cancel{7} 0 &  \cancel{7} 9 \\ \hline
			0 &  4 &  0 &  0 &  0 &  0 \\
			0 \cancel{6} \cancel{11} \cancel{19} &  8 & 5 & 6 & 0 & 0 \\
			0 \cancel{7} \cancel{14} &  0 & 0 &  0 &  7 & 7
		\end{tabular}
	\end{center}

	Zielfunktionswert: $z(x_{\text{NW}}) = 4 * 12 + 8 * 10 + 5 * 16 + 6 * 17 + 7 * 8 + 7 * 10 = 436$
	
	\vspace{\parskip}
	
	\begin{center}
		\begin{tabular}{r|rrrrr}
			$\text{X}_{\text{NW}}$ & 12 &  5 &  6 & \cancel{7} 0 &  \cancel{7} 3 \\ \hline
			0 \cancel{4} & 0 &  0 & 0 &  0 &  $\fbox{4}^3$ \\
			0 \cancel{12} \cancel{19} & 0 & $\fbox{5}$ & $\fbox{4}$ &  $\fbox{7}^1$ &  $\fbox{3}$ \\
			0 \cancel{2} \cancel{14} &  $\fbox{12}^2$ & 0 &  $\fbox{2}^4$ &  0 & 0
		\end{tabular}
	\end{center}

	Zielfunktionswert: $z(x_{\text{MK}}) = 268$

	\vspace{\parskip}
	
	\begin{center}
		\begin{tabular}{r|rrrrr}
			X & 0 &  5 &  6 &  7 &  7 \\ \hline
			4 &  &  4 &  &   &   \\
			19 &  & 1 & 4 &  7 &  7 \\
			14 &  12 &  &  2 &   & 
		\end{tabular}
	\end{center}
	Zielfunktionswert: $z(x_{\text{V}}) = 236$
\end{beispiel}

Je nach Qualität der Startlösung können unterschiedlich viele Iterationen des Transportalgorithmus vonnöten sein.

\subsection{Der Transportalgorithmus}

Ausgehend von einer Startlösung berechnet der Algorithmus zunächst ein Paar $(u,v)$ dualer Variablen und prüft dann die Optimalität mit \cref{satz: 3.14}. Liegt keine Optimalität vor, wird ein neuer Plan erzeugt.

\textbf{Vorgehensweise:}

\begin{enumerate}
	\item Bestimme einen zulässigen, zyklenfreien Transportplan $X_0$ mit genau $r+s-1$ markierten Basiszellen. (Diese bilden dann die Menge $J_B = J_B(X_0)$.)
	\item Bestimme für den aktuellen Plan $X$ die zugehörigen dualen Variablen $u_i$ und $v_k$ aus dem Gleichungssystem 
	\begin{equation}
		u_i + v_k = c_{ik} \qquad (i,k) \in J_B = J_B(X)
		\label{eq: 3.18}
	\end{equation}
	\item Berechne für alle $(i,k) \notin J_B$ die Koeffizienten $w_{ik} = c_{ik} - u_i - v_k$. Falls $w_{ik} \ge 0$ für alle Zellen ist, dann ist $X$ optimal. Andernfalls wähle man eine Zelle $(p,q)$ mit $w_{pq} = \min\menge{w_{ik} : i \in I, k \in K} < 0$.
	\item Markiere $(p,q)$ im Schema von $X$, bestimme den  (eindeutigen) Zyklus $J_{pq}$ in $J_B \cup \menge{(p,q)}$ und markiere abwechselnd die Zellen in $J_{pq}$ mit ''+`` und ''-``. Sei $J_{pq}^-$ die Menge der mit ''-`` gekennzeichneten Zellen.
	\item Ermittle $\delta = x_{gh} \defeq \min\menge{x_{ik} : (i,k) \in J_{pq}^-}$ und aktualisiere den Plan $X$ gemäß
	\begin{equation*}
		X^{\text{neu}} \defeq \brackets{x_{ik}^{\text{neu}}} \quad \mit \quad  
		x_{ik}^{\text{neu}} \defeq \begin{cases}
		x_{ik} + \delta & \text{ falls }(i,k) \in \brackets{J_{pq} \cup \menge{(g,h)}} \setminus J_{pq}^-\\
		x_{ik} - \delta & \text{ falls }(i,k) \in J_{pq}^-\\
		x_{ik} & \text{ sonst}
		\end{cases}
	\end{equation*}
	Aktualisere die Menge der Basiszellen
	\begin{equation*}
		J_B^{\text{neu}} \defeq \brackets{J_B \cup \menge{(p,q)}} \setminus \menge{(g,h)}
	\end{equation*}
	und gehe zu Schritt 2.
\end{enumerate}

\begin{bemerkung} %3.7
	Aufgrund von $\card{J_B} = r + s - 1$ ist das in Schritt 2 zu lösende Gleichungssystem (zur Ermittlung von $u$ und $v$) unterbestimmt. Eine der Variablen kann also beliebig festgelegt werden. Die in Schritt 3 bestimmten Werte $w_{ik}$ sind jedoch unabhängig von dieser Wahl.
\end{bemerkung}

Im gesamten Algorithmus gilt stets $w_{ik} = 0$ für die aktuellen Basiszellen $(i,k) \in J_B$ (per Konstruktion in Schritt 2), diese beeinflussen den Optimalitätstest in Schritt 3 also nicht. Daher kann zur Darstellung das folgende komprimierte Schema genutzt werden:

\begin{center}
	\begin{tabular}{c|cccc}
		T & $v_1$ & $v_2$ & $\dots$ & $v_s$ \\ \hline
		$u_1$ & \fbox{$x_{11}$} & $w_{12}$  & $\dots$ & $w_{1s}$ \\
		$u_2$ & $w_{21}$ & $w_{22}$ & \fbox{$x_{23}$} & $w_{2s}$ \\
		$\vdots$ &  $\vdots$ & $\vdots$ & $\vdots$ & $\vdots$ \\
		$u_r$ & $w_{r1}$ & \fbox{$x_{r2}$} & $\dots$ & $w_{rs}$ 
	\end{tabular}
\end{center}


\begin{beispiel}[Fortsetzung von \cref{beispie: 3.6}] %3.7
	Gegeben Sei das Problem
	\begin{center}
		\begin{tabular}{r|rrrrr}
			C & 12 &  5 &  6 &  7 &  7 \\ \hline
			4 & 12 &  6 & 10 &  9 &  5 \\
			19 & 10 & 16 & 17 &  3 &  7 \\
			14 &  4 & 11 &  5 &  8 & 10
		\end{tabular}
	\end{center}
	Mit der Minimale-Kosten-Regel haben wir bereits den Plan $X_0$ bestimmt:
	\begin{center}
		\begin{tabular}{r|rrrrr}
			$X_0$ & 12 &  5 &  6 & 7 & 7 \\ \hline
			4 & 0 &  0 & 0 &  0 &  \fbox{4} \\
			19 & 0 & \fbox{5} & \fbox{4} &  \fbox{7} &  \fbox{3} \\
			14 &  \fbox{12} & 0 &  \fbox{2} &  0 & 0
		\end{tabular}
	\end{center}
	Bestimmung der dualen Variablen (Potenziale) $u_i$ und $v_k$
	\begin{center}
		\begin{tabular}{l|ccccc}
			$T_0$ & $v_1=16$ & $v_2 = 16$ & $v_3=17$ & $v_4=3$ & $v_5=7$ \\ \hline
			$u_1 = -2$ & $-2$ & $-8$  & $-5$ & $8$ & \fbox{$4$} \\
			$u_2 = 0$ & $-6$ & \fcolorbox{black}{cdgray!10}{$5$} & \fbox{$4$} &\fbox{$7$} & \fbox{$3$} \\
			$u_3 = \fcolorbox{cdgreen!10}{cdgreen!10}{-12}$ &  \fbox{$12$} & $7$ & \fbox{$2$} & $17$ & $15$ \\
		\end{tabular}
	\end{center}
	\fcolorbox{black}{cdgray!10}{$5$} : $\underbrace{u_2}_{=0} + v_2 = c_{22} = 16 \follows v_2 = 16$
	
	\fcolorbox{cdgreen!10}{cdgreen!10}{$-12$} : $u_3 + \underbrace{v_3}_{=17} = c_{33} = 5 \follows u_3 = c_{33} - 17 = -12$
	
	Dieser Plan ist nicht optimal, da negative Einträge $w_{ik}$ existieren.
	Wir wählen also den eindeutig bestimmten Zyklus $-8 \to 4 \to 3 \to 5 \to 8$ und geben alternierende ''Vorzeichen``, d.h. $-8^+ \to 4^- \to 3^+ \to 5^- \to 8^+$. 	
	Wir wählen das kleinste mit einem ''$-$`` markierte Zellenelement des Zyklus: $\delta = \min\menge{x_{ik} : (i,k) \in J_{pq}^-} = 4$.
	
	Ein neuer Plan ergibt sich nun mit
	\begin{center}
		\begin{tabular}{l|ccccc}
			$X_1$ & \\ \hline
			&  & \fcolorbox{black}{cdgray!10}{$4$}  & & & $0$ \\
			&  & \fbox{$1$} & \fbox{$4$} &\fbox{$7$} & \fbox{$7$} \\
			&  \fbox{$12$} &  & \fbox{$2$} & &  \\
		\end{tabular}
	\end{center}

	Achtung: $x_{21} = \fcolorbox{black}{cdgray!10}{4}$ war vorher nicht in der Basis (also $x_{21} = 0$) und somit $x_{21}^{\text{neu}} = x_{21} + \delta = 4$.
	
	Bestimmung der Potenziale $u_i$ und $v_k$ für $X_1$:
	\begin{center}
		\begin{tabular}{l|ccccc}
			$T_1$ & $v_1=16$ & $v_2 = 16$ & $v_3=17$ & $v_4=3$ & $v_5=7$ \\ \hline
			$u_1 = -10$ & $6$ & \fbox{$4$}  & $4$ & $16$ & $8$ \\
			$u_2 = 0$ & $-6$ & \fbox{$1$} & \fbox{$4$} &\fbox{$7$} & \fbox{$7$} \\
			$u_3 = -12$ &  \fbox{$12$} & $7$ & \fbox{$2$} & $17$ & $15$ \\
		\end{tabular}
	\end{center}
	Auch dieses Tableau ist noch nicht optimal. Wir erkennen den Zyklus $-6^+ \to 12^- \to 2^+ \to 4^-$.
	
	Ein neuer Plan ergibt sich zu
	\begin{center}
		\begin{tabular}{l|ccccc}
			$X_2$ & \\ \hline
			&  & \fbox{$4$}  & & &  \\
			& \fbox{$4$} & \fbox{$1$} & &\fbox{$7$} & \fbox{$7$} \\
			&  \fbox{$8$} &  & \fbox{$6$} & &  \\
		\end{tabular}
	\end{center}
	mit den Potenzialen
	\begin{center}
		\begin{tabular}{l|ccccc}
			$T_2$ & $v_1=10$ & $v_2 = 16$ & $v_3=11$ & $v_4=3$ & $v_5=7$ \\ \hline
			$u_1 = -10$ & $12$ & \fbox{$4$}  & $9$ & $16$ & $8$ \\
			$u_2 = 0$ & \fbox{$4$} & \fbox{$1$} & $6$ &\fbox{$7$} & \fbox{$7$} \\
			$u_3 = -6$ & \fbox{$8$} & $1$ & \fbox{$6$} & $11$ & $9$ \\
		\end{tabular}
	\end{center}

	Alle $w_{ik} \ge 0$, d.h. der das Tableau ist optimal und $X_2$ ist eine Lösung der gegebenen Optimierungsaufgabe. Es gilt $z(X_2) = 212$.
\end{beispiel}

\chapter{Diskrete Optimierung}
\label{chapter_4_diskreteOptimierung}
In diesem Kapitel befassen wir uns mit Techniken zur Lösung ganzzahliger Optimierungsaufgaben. Dabei dürfen einige oder gar alle Variablen diskrete Werte annehmen.
Somit entfällt eine Argumentation über Ableitung, zulässige Richtungen etc. Ganzzahlige Optimierungsaufgaben sind also ''schwieriger`` als die zugehörige stetige Relaxation. Dennoch kann, in einigen Fällen, das Lösen diskreter Aufgaben auch zur effizienten Lösung stetiger Aufgaben beitragen, wie folgendes Beispiel verdeutlicht.
\section{Spaltengenerierung}

Wir betrachten die stetige Relaxation des Bin-Packing-Problems (vgl. \cref{chapter_1_einfuehrung}). Zur Erinnerung: Es sind $b_i$ Teile der Länge $\ell_i$ ($i = 1, \dots, m$) in möglichst wenige Behälter der Kapazität $L$ zu packen. 
\begin{itemize}[nolistsep, topsep=-\parskip]
	\item Packungsvarianten: $a^j = \transpose{a_1^j , \dots , a_m^j} \in \Z_+^m$ mit $\trans{\ell}a^j \le L$ ($j \in J$)
	\item Variablen: $x_j$ beschreibt Häufigkeit, wie oft Variante $a^j$ genutzt wird.
\end{itemize}
\begin{equation*}
	z = \sum_{j \in  J} x_j \to \min \quad \bei \quad \sum_{j \in J} a_i^j * x_j = b_i \quad (i \in I) \quad \und \quad x_j \ge 0 \quad (j \in J)
\end{equation*}
Grundsätzlich ist diese Aufgabe mit dem Simplexverfahren lösbar, jedoch gibt es im Allgemeinen exponentiell viele Variablen, sodass pro Austauschschritt ein großer Aufwand entstünde.

Wir können uns hierbei zu Nutze machen, dass alle Spalten der Systemmatrix $A$ eine gemeinsame Struktur aufweisen:
\begin{equation*}
	a^j \text{ ist Spalte von } A \equivalent a^j \in \Z_+^m \und \trans{\ell} a^j \le L
\end{equation*}

Offenbar gilt hie $c = e = \transpose{1, \dots, 1}$, sodass für eine gewählte Basismatrix $A_B$ in Schritt 2 des Simplexalgorithmus folgendes zu bestimmen wäre:
\begin{equation*}
	\quer{q} \defeq \min_{j \in J_N} q_j \quad \mit \quad q_j = c_j - \trans{d} a^j = 1 - \trans{d}a^j, \quad a^j \in \Z_+^m, \trans{\ell} a^j \le L
\end{equation*}
wobei $\trans{d} \defeq \trans{c_B} A_B^{-1}$. In Schritt 2 wäre folglich die Aufgabe 
\begin{equation*}
	1 - \trans{d} a^j = q_j \to \min \bei \trans{\ell} a^j \le L \und a^j \in \Z_+^m
\end{equation*}
bzw. 
\begin{equation*}
	\trans{d} a^j \to \max \bei \trans{\ell} a^j \le L, a^j \in \Z_+^m
\end{equation*}
zu lösen.
Gilt $q_j^\ast < 0$, so liegt keine Optimalität vor und eine zugehörige Lösung $a^{j, \ast}$ wäre in die Basismatrix aufzunehmen. Gilt $g_j^\ast \ge 0$, so sind wir fertig.
\section{Die Methode Branch \& Bound}

Branch \& Bound (B\&B) ist eine sehr flexible Technik, um exakte Lösungsverfahren für Probleme der diskreten Optimierung zu entwickeln. Anschaulich betrachtet wird dabei eine schwierige Optimierungsaufgabe sukzessiv in Teilprobleme zerlegt, die wiederum ''leicht`` (näherungsweise) gelöst werden können und somit zur Lösung des Gesamtproblems beitragen. Näherungslösungen erhält man dabei oftmals mithilfe geeigneter Relaxationen.

\subsection{Grundlagen}

Wir betrachten das Anfangsproblem
\begin{equation*}
	f(x) \to \min \bei x \in E \cap D
	\tag{$P_0$}
	\label{eq: p_0}
\end{equation*}
und eine zugehörige Relaxation
\begin{equation*}
	g(x) \to \min \bei x \in E
	\tag{Q}
	\label{eq: q}
\end{equation*}
wobei $g(x) \le f(x)$ auf $D \cap E$ gilt.

\vspace{\parskip}
\fbox{\textbf{Prinzip der B\&B-Methode}}

Die Menge $E$ wird durch Separation in Teilmengen $E_i$ mit $i \in I$ zerlegt. Dadurch entstehen \begriff{Teilprobleme} 
\begin{equation*}
	f(x) \to \min \bei x \in D \cap E_i
	\tag{$P_i$}
	\label{eq: p_i}
\end{equation*}
Jedem dieser Teilprobleme \eqref{eq: p_i} soll nun eine Zahl $b(P_i)$, genannt \begriff{untere Schranke}, zugeordnet werden, sodass gilt
\begin{enumerate}[label=(\alph*), nolistsep]
	\item $b(P_i) \le \min\menge{f(x) : x \in D \cap E_i}$
	\item $b(P_i) = f(\dach{x})$ falls $D \cap E = \menge{\dach{x}}$
	\item $b(P_i) \le b(P_j)$ falls $E_j \subset E_i$
\end{enumerate}

Eine geeignete Möglichkeit besteht darin, z.B. die stetige Relaxation der Teilprobleme \eqref{eq: p_i} zu betrachten, d.h. 
\begin{equation*}
	b(P_i) \defeq \begin{cases}
	\min\menge{g(x) : x \in E_i} & \falls \card{E_i \cap D} > 1 \\
	f(\dach{x}) & \falls \card{E_i \cap D} = 1 \\
	+ \infty & \falls E_i \cap D = \emptyset
	\end{cases}
\end{equation*}

\pagebreak

\subsection{Allgemeiner B\&B-Algorithmus}
Bezeichne mit $R$ die Menge der noch zu bearbeitenden Teilprobleme (''Restmenge``) und mit $\quer{z}$ den Zielfunktionswert der bisher besten gefundenen zulässigen Lösung $\quer{x} \in D \cap E$.

\begin{enumerate}[label=\underline{\textbf{Schritt \arabic*:}}, leftmargin=*]
	\setcounter{enumi}{-1}
	\item \textbf{Initialisierung} --- Bestimme $b(P_0)$.
	\begin{enumerate}[label=(\alph*), noitemsep]
		\item Falls $\quer{x} \in D \cap E$ bekannt ist mit $f(\quer{x}) = b(P_0)$, dann \texttt{STOP}.
		\item Setze $R \defeq \menge{P_0}$ und $\quer{z} \defeq + \infty$ oder $\quer{z} = f(x)$, wenn ein $x \in D \cap E$ bekannt ist.
	\end{enumerate}	
	\item \textbf{Abbruchtest} --- Falls $R \neq \emptyset$, dann \texttt{STOP}. Falls $\quer{z} = +\infty$, dann ist \eqref{eq: p_0} nicht lösbar (leerer zulässiger Bereich), andernfalls ist $\quer{x}$ Lösung von \eqref{eq: p_0}
	\item \textbf{Strategie} --- Wähle entsprechend einer Auswahlstrategie ein $P_i \in R$ und setze $R \defeq R \setminus \menge{P_i}$.
	\item \textbf{Zerlegung (''branch``)} --- Zerlege $P_i$ durch Separation in endlich viele Teilprobleme $P_{i,1}, \dots, P_{i,k_i}$. Setze $j \defeq 1$.
	\item \textbf{Schranken- und Dominanztests (''bound``)}
	\begin{enumerate}[label=(\alph*), noitemsep]
		\item Berechne $b(P_{i,j})$. Falls dabei ein $\schlange{x} \in D \cap E$ gefunden wurde mit $f(\schlange{x}) < \quer{z}$, setze $\quer{x} \defeq \schlange{x}$ und $\quer{z} \defeq f(\schlange{x})$.
		\item Falls $b(P_{i,j}) < \quer{z}$, dann setze $R \defeq R \cup \menge{P_{i,j}}$. Falls $j < k_i$, setze $j \defeq j + 1$ und gehe zu (a).
		\item Setzte $R \defeq R \setminus \menge{P_k}$ für alle $P_k \in R$ mit $b(P_k) \ge \quer{z}$. 
	\end{enumerate}	
	Gehe zu Schritt 1.
\end{enumerate}

\begin{*bemerkung}
	\begin{enumerate}
		\item Die Endlichkeit des Verfahrens ist zu sichern, z.B. durch $\card{E_{i,j} \cap D} \le \card{E_i \cap D}$ für alle $j$ (falls $E_i \cap D$ endlich ist) oder durch $b(P_{i,j}) > b(P_i) + \epsilon$ mit $\epsilon > 0$ für alle $j$ und $i$.
		\item Das B\&B-Verfahren kann mithilfe eines Verzweigungsbaumes veranschaulicht werden.
		\item In Schritt 2 können verschiedene Auswahlstrategien gewählt werden, z.B. 
		\begin{itemize}[noitemsep]
			\item \begriff{Minimalsuche} (best bound search): wähle $P_i \in R$ mit $b(P_i) \le b(P_k)$ für alle $P_k \in R$.
			\item \begriff{Tiefensuche} (depth-first search, LIFO): wähle $P_i \in R$ mit kleinstem Schrankenwert unter allen Teilproblemen mit maximaler Verzweigungstiefe.
			\item \begriff{Breitensuche} (breadth-first-search, FIFO): wähle $P_i \in R$ mit kleinstem Schrankenwert unter allen Teilproblemen mit minimaler Verzweigungstiefe.
		\end{itemize}
	\end{enumerate}
\end{*bemerkung}

\subsection{Beispiele für B\&B-Verfahren}

\subsubsection{Das 0/1-Rucksackproblem}

Zur Erinnerung: Gegeben seien ganze Zahlen $c_i > 0$, $0 < a_i < b$ für $i \in I \defeq \menge{1, \dots, n}$
\begin{align*}
	f(x) &= \trans{c} x \to \max \bei \trans{a} x \le b, \enskip x \in \menge{0,1}^n 
	\tag{P} \\
	f(x) &= \trans{c} x \to \max \bei \trans{a} x \le b, \enskip c \in [0,1]^n 
	\tag{Q} \label{eq: rucksack-q}
\end{align*}
Hier gilt also $E = \menge{x \in \Rn : \trans{a} x \le b, 0 \le x \le 1, i \in I}$ und $D = \mathbb{B}^n = \menge{0,1}^n$.

\begin{bemerkung}
	Die stetige Relaxation \eqref{eq: rucksack-q} besitzt unter der Voraussetzung
	\begin{equation*}
		\frac{c_i}{a_i} \ge \frac{c_{i+1}}{a_{i+1}} \quad \text{ für } 1 \le i \le n
	\end{equation*}
	die Lösung
	\begin{equation*}
		\dach{x_i} = 1 \quad (i = 1, \dots, k) \qquad \dach{x}_{k+1} = \frac{b - \sum_{i=1}^k a_i}{a_{k+1}} \qquad \dach{x}_i = 0 \enskip i = k+2, \dots, n 
		\tag{R} \label{eq: rucksack-r}
	\end{equation*}
	mit $k \defeq \max\menge{j \in I : \sum_{i=1}^j a_i \le b}$.
\end{bemerkung}

\begin{*bemerkung_inline}
	Wir sortieren also abfallend nach Nutzen pro Volumen und packen dann soviel wie möglich in den Rucksack ($k$ Elemente). Dann füllen wir den Restplatz noch mit (einem Anteil von) dem nächsten Element auf ($k+1$-tes Element), alles danach können wir nicht mehr mitnehmen.
\end{*bemerkung_inline}

Innerhalb des Verzweigungsbaums ergeben sich die folgenden Teilprobleme (für bereits fixierte Variablen $\quer{x}_i$, $i \in I_k \subseteq I$).
\begin{align*}
	\sum_{i \in I_k} c_i \quer{x}_i + \sum_{i \notin I_k} c_i x_i &\to \max \bei \sum_{i \notin I_k} a_i x_i \le b - \sum_{i \in I_k} a_i \quer{x}_i, \quad i \notin I_k, x_i \in \mathbb{B}
	\tag{$P_k(\quer{x})$} \label{eq: rucksack-pk}\\
	\sum_{i \in I_k} c_i \quer{x}_i + \sum_{i \notin I_k} c_i x_i &\to \max \bei \sum_{i \notin I_k} a_i x_i \le b - \sum_{i \in I_k} a_i \quer{x}_i, \quad i \notin I_k, x_i \in [0,1]  \tag{$Q_k(\quer{x})$} \label{eq: rucksack-qk}
\end{align*}

\textbf{Beachte:} Da es sich um eine Maximierungsaufgabe handelt, werden im B\&B-Algorithmus \textit{obere} Schranken $b(P_k(\quer{x}))$ benötigt. Diese gewinnen wir aus den Optimalwerten der stetigen Relaxation $Q_k(\quer{x})$.

\begin{beispiel}
	Gegeben sei das binäre Rucksackproblem 
	\begin{equation*}
		\begin{alignedat}{8}
		z = \trans{c} x &= &8x_1 &+ &16x_2 &+ &20x_3 &+ &12x_4 &+ &6x_5 &+ &10x_6 &+ &4x_7 &\to \max \\
		\bei \trans{a} x &= &3x_1 &+ &7x_2 &+ &9x_3 &+ &6x_4 &+ &3x_5 &+ &5x_6 &+ &2x_7 &\le 17 = b, \quad x_i \in \menge{0,1}, i = 1, \dots, 7
		\end{alignedat}
		\tag{P} \label{eq: rucksack-bsp-p}
	\end{equation*}
	
	\begin{itemize}[leftmargin=*]
		\item Die korrekte Sortierung liegt bereits vor.
		\item Wurzelknoten \eqref{eq: rucksack-bsp-p}$=(\text{P}_0)$: Die stetige Relaxation besitzt die Lösung $x_1 = x_2 = 1$ und $x_3 = \frac{7}{9}$ sowie $x_4 = \dots = x_7 = 0$. Somit ist $b(\text{P}_0) = \lfloor 8 + 16 + 20 * \frac{7}{9} \rfloor = 39$. Die Abrundung ist dabei erlaubt, da $\trans{c} x \in \Z$ für alle zulässigen Punkte von \eqref{eq: rucksack-bsp-p}. Ein zulässiger Punkt für \eqref{eq: rucksack-bsp-p} ist gegeben durch $\quer{x}_1 = \quer{x}_2 = \quer{x}_4 = 1$, $\quer{x}_i = 0$ sonst. Dabei ist $z(\quer{x}) = 8 + 16 + 12 = 36 \defqe \quer{z}$. Wegen $36 < 39$ muss weiter verzweigt werden.
		\item Verzweigung: $x_3 = 0$ vs. $x_3 = 1$.
		\begin{itemize}
			\item Teilproblem: $(P_1) = (P_{0,1})$. Setzt man $x_3 = 0$, so hat die stetige Relaxation die Lösung $x_1 = x_2 = x_4 = 1$, $x_5 = \frac{1}{3}$ und $x_6 = x_7 = 0$. Somit ist $b(P_1) = \lfloor 8 + 16 + 12 + 6 * \frac{1}{3} = 38$. Ein daraus ableitbarer Punkt ist gegeben durch $\quer{x}_1 = \quer{x}_2 = \quer{x}_4 = 1$ mit $z = 36$, d.h. $\quer{z}$ muss nicht aktualisiert werden.
			\item $(P_2) = (P_{0,2})$. Setzt man $x_3 = 1$, so erhält man $x_1 = 1$, $x_2 = \frac{5}{7}$ mit $b = \lfloor 8 + 20 + 16 * \frac{5}{7} \rfloor = 39$ und einen zulässigen Punkt für \eqref{eq: rucksack-bsp-p} durch $\quer{x}_1 = \quer{x}_3 = \quer{x}_5 = \quer{x}_7 = 1$ mit $\quer{z} = 38$, d.h. $\quer{z}$ wurde verbessert.
			\item Damit kann ($P_1$) abgeschlossen werden, da maximal noch der Zielfunktionswert $38$ möglich ist, der in ($P_2$) bereits erreicht worden ist.
		\end{itemize}
		%
		\item Verzweigung: $x_2 = 0$ vs. $x_2 = 1$.
		\begin{itemize}
			\item Teilproblem $(P_3) = (P_{2,1})$. Setzt man $x_2 = 0$ (und $x_3 = 1$ von oben), dann hat die stetige Relaxation die Lösung $x_1 = 1$, $x_4 = \frac{5}{6}$ mit $b(P_3) = 38 = \quer{z}$, also sind wir hier fertig.
			\item Teilproblem $(P_4) = (P_{2,2})$. Setzt man $x_2 = 1$ (und $x_3 = 1$) so folgt $x_1 = \frac{1}{3}$, also $b(P_4) = 38 = \quer{z}$, d.h. wir sind hier wieder fertig.
		\end{itemize}
		%
		\item Nun ist $R = \emptyset$ und wir haben eine Lösung gefunden:
		\begin{equation*}
			x_1 = x_3 = x_5 = x_7 = 1 \quad \mit \quad z = 38
		\end{equation*}
	\end{itemize}
\end{beispiel}


%%%%%%%%%%%%%%%%%%%%%%%%%%%%%%%%%%%%%%%%%%%%%%%%%%%%%%%%%%%%%%%%%%%%%%%%%%%%%%%%%%
%%%%%%%%%%%%%%%%%%%%%%%%%%%%%%%%%%%%%%%%%%%%%%%%%%%%%%%%%%%%%%%%%%%%%%%%%%%%%%%%%%


\subsubsection{Ganzzahlige lineare Optimierung nach Land/Doig/Dahin}

Wir betrachten die ganzzahlige Optimierungsaufgabe 
\begin{equation*}
	\trans{c}x  \to \min \bei x \in D \cap E
	\tag{P} \label{eq: ldl-p}
\end{equation*}
mit $D = \Z^n$ und $E = E_0 = \menge{x \in \Rn: Ax = b, x \ge 0}$, wobei alle Inputdaten $(A,b,c)$ ganzzahlig sind. Die im Verlauf des Verfahrens zu betrachtenden Teilprobleme \eqref{eq: ldl-pi} haben die Form
\begin{equation*}
	\trans{c} x \to \min \bei x \in D \cap E_i
	\tag{P${}_\text{i}$} \label{eq: ldl-pi}
\end{equation*}
wobei $E_i$ durch eine oder mehrere zusätzliche Ungleichungen aus $E_0$ entsteht. Sei $x^{\text{LP}}$ eine Lösung der zu \eqref{eq: ldl-pi} gehörenden stetigen Relaxation
\begin{equation*}
	z(Q_i) = \min \menge{\trans{c} x : x \in E_i} 
	\tag{Q${}_\text{i}$} \label{eq: ldl-qi}
\end{equation*}
mit $x^{\text{LP}}_j \notin \Z$ für mindestens einen Index $j$. Dann kann der (gerundete) Optimalwert als Schranke $b(P_i)$ genutzt werden.
\begin{align*}
	E_{i,1} &= \menge{x \in E_i : x_j \le \left\lfloor x^{\text{LP}} \right\rfloor}
	\tag{P${}_{i,1}$} \\
	E_{i,2} &= \menge{x \in E_i : x_j \ge \left\lfloor x^{\text{LP}} \right\rfloor + 1 = \left\lceil x^{\text{LP}} \right\rceil} 
	\tag{P${}_{i,2}$}	
\end{align*}

\begin{bemerkung}
	Das Runden des Optimalwertes einer Relaxation zur Schrankenbestimmung ist nur für $c \in \Z^n$ zulässig.
\end{bemerkung}

\begin{beispiel}
	Betrachte 
	\begin{equation*}
	\begin{alignedat}{4}
		z = -7x_1 - 2x_2 \to \min \bei &&-x_1 &+ &2x_2 &+ &x_4 &= 4 \\
		&&5x_1 &+ &x_2 &+ &x_4 &= 20, \quad x_1, \dots, x_n \in \Z_+
	\end{alignedat}
	\end{equation*}
	Die Relaxation ergibt sich durch $x_1, \dots, x_n \ge 0$.
	\begin{itemize}[leftmargin=*]
		\item Wurzelknoten:
		\begin{center}
			\begin{tabular}{r|rr|r}
				$T_1$ & $x_1$ & $x_2$ & $1$ \\ \hline
				$x_3 = $ & $1$ & $-2$ & $4$ \\
				$x_4 = $ & $-5$ & $-1$ & $20$ \\ \hline
				$z = $   & $-7$ & $-2$ & $0$
			\end{tabular}
			$\qquad \overset{x_1 \leftrightarrow x_4}{\longrightarrow} \quad \dots \quad \overset{x_2 \leftrightarrow x_3}{\longrightarrow} \qquad$
			\begin{tabular}{r|rr|r}
				$T_3$ & $x_4$ & $x_3$ & $1$ \\ \hline
				$x_2 = $ & $-\frac{1}{11}$ & $-\frac{5}{11}$ & $\frac{40}{11}$ \\
				$x_1 = $ & $\frac{-2}{11}$ & $\frac{1}{11}$ & $\frac{26}{11}$ \\ \hline
				$z = $   & $\frac{16}{11}$ & $\frac{3}{11}$ & $-\frac{332}{11}$
			\end{tabular}
		\end{center}	
		%
		\item Verzweigung: nach $x_2$, da $\min \menge{4-\frac{40}{11}, \frac{40}{11} - 3} > \min \menge{4 - \frac{36}{11}, \frac{36}{11} - 3}$. 
		\begin{itemize}
			\item 1. Teilproblem: $x_2 \ge 4$ (führe $s_2 \ge 0$ ein)
			\begin{equation*}
				\begin{aligned}
					\follows s_2 &= x_2 - 4 
					\overset{\text{aus T}_3}{=} \brackets{-\frac{1}{11}x_4 - \frac{5}{11} x_3 + \frac{40}{11}} - 4 \\
					&= -\frac{1}{11} x_4 - \frac{5}{11} x_3 - \frac{4}{11} < 0
				\end{aligned}
			\end{equation*}
			Damit ist $x_2 \ge 4$ nicht möglich (leerer zulässiger Bereich) und dieser Fall muss nicht betrachtet werden.
			%
			\item 2. Teilproblem: $x_2 \le 3$ (führe $s_2 \ge 0$ ein)
			\begin{equation*}
				\begin{aligned}
					s_2 &= 3 - x_2 = 3 - (-\frac{1}{11} x_4 - \frac{5}{11} x_3 + \frac{40}{11}) \\
					&= \frac{1}{11} x_4 + \frac{5}{11} x_3 - \frac{7}{11}
				\end{aligned}
			\end{equation*}
			Füge dies ist $T_3$ ein:
			
			\begin{center}
				\begin{tabular}{r|rr|r}
					$T_3'$ & $x_4$ & $x_3$ & $1$ \\ \hline
					$x_2 = $ & $-\frac{1}{11}$ & $-\frac{5}{11}$ & $\frac{40}{11}$ \\
					$x_1 = $ & $\frac{-2}{11}$ & $\frac{1}{11}$ & $\frac{26}{11}$ \\
					$s_2 = $ & $\frac{1}{11}$ & \fbox{$\frac{5}{11}$} & $-\frac{7}{11}$ \\ \hline
					$z = $   & $\frac{16}{11}$ & $\frac{3}{11}$ & $-\frac{332}{11}$ \\ \hline
					Keller & $-\frac{1}{5}$ & $\ast$ & $\frac{7}{5}$
				\end{tabular}	
				$\quad \overset{s_2 \leftrightarrow x_3}{\longrightarrow} \quad$		
				\begin{tabular}{r|rr|r}
					$T_4$ & $x_4$ & $s_2$ & $1$ \\ \hline
					$x_2 = $ & $0$ & $-1$ & $3$ \\
					$x_1 = $ & $-\frac{1}{5}$ & $\frac{1}{5}$ & $\frac{17}{5}$ \\
					$x_3 = $ & $-\frac{1}{5}$ & $\frac{11}{5}$ & $\frac{7}{5}$ \\ \hline
					$z = $   & $\frac{7}{5}$ & $\frac{3}{5}$ & $-\frac{149}{5}$
				\end{tabular}
			\end{center}
		\end{itemize}
		%
		\item Verzweigung: $x_1$ ($x_3$ ebenso möglich)
		\begin{itemize}
			\item 3. Teilproblem: $x_1 \ge 4 \follows s_1 = x_1 - 4 = -\frac{1}{5}x_4 + \frac{1}{5} s_2 - \frac{3}{5}$
			
			\begin{center}
				\begin{tabular}{r|rr|r}
					$T_4'$ & $x_4$ & $s_2$ & $1$ \\ \hline
					$x_2 = $ & $0$ & $-1$ & $3$ \\
					$x_1 = $ & $-\frac{1}{5}$ & $\frac{1}{5}$ & $\frac{17}{5}$ \\
					$x_3 = $ & $-\frac{1}{5}$ & $\frac{11}{5}$ & $\frac{7}{5}$ \\
					$s_1 = $ & $-\frac{1}{5}$ & \fbox{$\frac{1}{5}$} & $\frac{3}{5}$ \\ \hline
					$z = $   & $\frac{7}{5}$ & $\frac{3}{5}$ & $-\frac{149}{5}$
				\end{tabular}
				$\quad \overset{s_1 \leftrightarrow s_2}{\longrightarrow} \quad$	
				\begin{tabular}{r|rr|r}
					$T_5$ & $x_4$ & $s_1$ & $1$ \\ \hline
					$x_2 = $ &  &  & $0$ \\
					$x_1 = $ &  &  & $4$ \\
					$x_3 = $ &  &  & $3$ \\
					$s_2 = $ &  &  & $8$ \\ \hline
					$z = $   & $2$ & $3$ & $-28$
				\end{tabular}
			\end{center}
			Damit haben wir zumindest \textit{eine} ganzzahlige Lösung $z = -28$. Wir wissen jedoch noch nicht, ob es tatsächlich die optimale Lösung ist.
			%
			\item 4. Teilproblem: $x_1 \le 3 \follows s_1 = 3-x_1 = \frac{1}{5} x_4 - \frac{1}{5} s_2 - \frac{2}{5}$. Dies fügt man zu $T_4$ hinzu. Ein dualer Simplexschritt führt dann zu
			\begin{center}
				\begin{tabular}{r|rr|r}
					$T_6$ & $s_1$ & $s_2$ & $1$ \\ \hline
					$x_2 = $ &  &  & $3$ \\
					$x_1 = $ &  &  & $3$ \\
					$x_3 = $ &  &  & $2$ \\
					$x_3 = $ &  &  & $1$ \\ \hline
					$z = $   & $2$ & $3$ & $-27$
				\end{tabular}
			\end{center}
			Dabei haben wir also eine ganzzahlige Lösung mit $z = -27$.
		\end{itemize}
	\end{itemize}
	Insgesamt wissen wir also $x^\ast = (4,0,3,0)$ mit $z^\ast = -28$ aus dem dritten Teilproblem.
\end{beispiel}


%%%%%%%%%%%%%%%%%%%%%%%%%%%%%%%%%%%%%%%%%%%%%%%%%%%%%%%%%%%%%%%%%%%%%%%%%%%%%%%%%%
%%%%%%%%%%%%%%%%%%%%%%%%%%%%%%%%%%%%%%%%%%%%%%%%%%%%%%%%%%%%%%%%%%%%%%%%%%%%%%%%%%


\subsubsection{Das Rundreiseproblem (Traveling Salesman Problem, TSP)}

Gegeben seien $n$ Orte, eine Kostenmatrix $C = (c_{ik}) \in \R^{n \times n}$, wobei $c_{ik}$ die Entfernung (Zeit, Kosten, ...) von $i$ nach $k$ beschreibt. \\
\textit{Annahme:} $c_{ii} = + \infty$ für alle $i \in I = \menge{1, \dots, n}$. In einigen Anwendungen ist $C$ auch symmetrisch.

\textbf{Variablen:} $x_{ik} \in \menge{0,1}$, $(i,k) \in I \times I$ mit $x_{ik} = 1 \equivalent $ man reist von $i$ nach $k$

\textbf{Optimierungsproblem:}
\begin{align}
	z = \sum_{i \in I} \sum_{k \in I} c_{ik} x_{ik} \to \min 
	\label{eq: 4.1} \\
	\bei \sum_{i \in I} x_{ik} &= 1 \quad (k \in I) 
	\label{eq: 4.2} \\
	\sum_{k \in I} x_{ik} &= 1 \quad (i \in I)
	\label{eq: 4.3} \\
	x_{ik} &\in \menge{0,1} \quad (i,k) \in I \times I 
	\label{eq: 4.4} \\
	\sum_{i,k} x_{ik} \in S &\le \card{S} - 1 \quad (S \subset I, 0 < \card{S})
	\label{eq: 4.5} 
\end{align}

Die Bedingung \eqref{eq: 4.5} wird auch \begriff{Subtoureliminationsbedingung} (SEB) genannt.
Die Bedingungen \eqref{eq: 4.1} bis \eqref{eq: 4.4} modellieren ein spezielles Transportproblem (das \begriff{Zuordnungsproblem}).
Für das Problem \eqref{eq: 4.1} -- \eqref{eq: 4.5} kann man auf folgende Weise Schranken erhalten:
\begin{itemize}
	\item Lösung des (ganzzahligen) Zuordnungsproblems
	\item stetige Relaxation von \eqref{eq: 4.1} -- \eqref{eq: 4.5}: löse zunächst das Zuordnungsproblem. Falls dabei Subtouren entstehen, verbiete man diese mit geeigneten Bedingungen vom Typ \eqref{eq: 4.5} und löse danach das um diese Bedingung erweiterte Zuordnungsproblem.
	\item Zeilen- und Spaltenreduktion: Ermittlung einer zulässigen Lösung $(u,v)$ des dualen Problems (der stetigen Relaxation des Zuordnungsproblems) mithilfe der folgenden Idee:
	\begin{equation*}
		\begin{alignedat}{3}
			&u_i &&\defeq \min\menge{c_{ik} : k \in I} \qquad &&(i \in I) \\
			&v_i &&\defeq \min\menge{c_{ik} - u_i : i \in I} \qquad &&(k \in I) \\
			\follows &w_{ik} &&\phantom{:}= c_{ik} - u_i - v_k \ge 0 \qquad &&\forall (i,k) \in I \times I \text{ (duale Zulässigkeit)}
		\end{alignedat}
	\end{equation*}
	Somit gilt
	\begin{equation*}
		\begin{aligned}
			z &= \sum_{i \in I} \sum_{k \in I} c_{ik} x_{ik} 
			\overset{(\star)}{=} \sum_{i \in I} \sum_{k \in I} \underbrace{w_{ik}}_{\ge 0} \underbrace{x_{ik}}_{\ge 0} + \sum_{i \in I} u_i + \sum_{k \in I} v_k \\
			&\ge \sum_{i \in I} u_i + \sum_{k \in I} v_k \qquad \text{ (untere Schranke für $z$) }
		\end{aligned}
	\end{equation*}
	Im Schritt ($\star$) haben wir dabei $w_{ik} = c_{ik} + u_i + v_i$ und die Bedingungen \eqref{eq: 4.2} sowie \eqref{eq: 4.3} verwendet.
	
	Die oben genannten Schrankenwerte unterscheiden sich (mitunter stark) hinsichtlich des numerischen Aufwands und der Güte der erhaltenen Näherungen. In dieser Vorlesung betrachten wir die dritte Variante.
\end{itemize}

\begin{beispiel}
	Wir betrachten die Kostenmatrix
	
	\begin{center}
		\begin{tabular}{r|ccccc|r}
			$C = (c_{ik})$ &          &    &    &    &    & $u_i$ \\ \hline 
			& $\infty$ & 32 & \underline{22} & 30 & 24 & 22 \\
			& 10       & $\infty$ & \underline{3} & 18 & $\infty$ & 3\\
			& $\infty$ & \underline{9} & $\infty$ & 14 & 12 & 9\\
			& 16 & 10 & 7 & $\infty$ & \underline{6} & 6 \\
			& 15 & 19 & 15 & \underline{12} & $\infty$ & 12 \\ \hline
			$v_k$ & 3 & 0 & 0 & 0 & 0 & $b = 55$ \\ 
		\end{tabular}
	\end{center}

	Damit ist $b = \sum_i u_i + \sum_k v_k = 55$ eine untere Schranke für den Optimalwert.
\end{beispiel}

\begin{bemerkung}
	Die Schranke, die aus Zeilen- und anschließender Spaltenreduktion erhalten wird, weicht im Allgemeinen von der Schranke ab, die aus Spalten- und anschließender Zeilenreduktion gewonnen wird.
\end{bemerkung}

Unter Einbeziehung der reduzierten Kostenmatrix $D = (w_{ik})$ mit $w_{ik} = c_{ik} - u_i - v_i$ kann eine Verzweigungs- und Auswahlstrategie formuliert werden:
Dabei betrachten wir die Elemente mit $w_{ik} = 0$ und verzweigen gemäß $x_{ik} = 0$ vs. $x_{ik} = 1$.
\begin{itemize}
	\item Die Belegung $x_{ik} = 0$ ist gleichbedeutend mit der Änderung des aktuellen Kostenwertes auf $+\infty$.
	\item Eingedenk\footnote{Alternativer Vorschlag: im Lichte der Bedingungen \dots} der Bedingungen \eqref{eq: 4.2} und \eqref{eq: 4.3} impliziert die Wahl $x_{ik} = 1$, dass $x_{jk} = 0$ für alle $j \neq i$ und $x_{i\ell} = 0$ für alle $\ell \neq k$ gelten muss. Für diese Indexpaare kann man die Kosten auf $+\infty$ erhöhen. Weiterhin muss die Subtour $i \to k \to i$ verhindert werden, d.h. es gilt $x_{ki} = 0$ und damit $w_{ki} = +\infty$. (Analog verfahre man gegebenenfalls mit längeren Subtouren.)
	Infolge dieser Änderung der Kostenmatrix können im Anschluss weitere Zeilen- und Spaltenreduktionen ermöglicht werden, die zu verbesserte Schranken (in den entsprechenden Teilproblemen) führen können.
\end{itemize}


\end{document}