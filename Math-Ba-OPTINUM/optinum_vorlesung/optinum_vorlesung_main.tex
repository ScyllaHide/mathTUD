% This work is licensed under the Creative Commons
% Attribution-NonCommercial-ShareAlike 4.0 International License. To view a copy
% of this license, visit http://creativecommons.org/licenses/by-nc-sa/4.0/ or
% send a letter to Creative Commons, PO Box 1866, Mountain View, CA 94042, USA.

% (c) Eric Kunze, 2019

%%%%%%%%%%%%%%%%%%%%%%%%%%%%%%%%%%%%%%%%%%%%%%%%%%%%%%%%%%%%%%%%%%%%%%%%%%%%
% Template for lecture notes and exercises at TU Dresden.
%%%%%%%%%%%%%%%%%%%%%%%%%%%%%%%%%%%%%%%%%%%%%%%%%%%%%%%%%%%%%%%%%%%%%%%%%%%%

\documentclass[ngerman, a4paper, 11pt]{report}

\usepackage[ngerman]{babel}
\usepackage{../../texmf/tex/latex/layoutMathTUD}
\usepackage[smallequationskip]{../../texmf/tex/latex/mathworkMathTUD}

\usepackage{../../texmf/tex/latex/mathoperatorsMathTUD}
\usepackage[includechapter]{../../texmf/tex/latex/maththeorems3MathTUD}

\usepackage{../../texmf/tex/latex/titlepageMathTUD}
\usepackage{../../texmf/tex/latex/graphicsMathTUD}

%%%%%%%%%%%%%%%%%%%%%%%%%%%%%%%%%%%%%%%%%%%%%%%%%%%%%%%%%%%%%%%%%%%
%                        TITLE STYLES                             %
%%%%%%%%%%%%%%%%%%%%%%%%%%%%%%%%%%%%%%%%%%%%%%%%%%%%%%%%%%%%%%%%%%%

\usepackage{titlesec}   % change title headings look
\usepackage{chngcntr}   % modify counters
\usepackage{relsize}    % relative font size (smaller[i], larger[i], ...)

\makeatletter
\@ifpackageloaded{opensans}{}{\usepackage[scale=1]{opensans}}
\ifx\osfamily\undefined
    \newcommand*{\osfamily}{\fontfamily{fos}\selectfont}
    \DeclareTextFontCommand{\textos}{\osfamily}
\fi
\makeatother

\newcommand{\titlefont}{\osfamily}
\newcommand{\chaptersize}{\huge}
\newcommand{\sectionsize}{\LARGE}

\renewcommand{\thepart}{\Alph{part}}

% \titleformat{<command>}[<shape>]{<format>}{<label>}{<sep>}{<before-code>}[<after-code>]
% \titlespacing*{<command>}{<left>}{<before-sep>}{<after-sep>}[<right-sep>]

%%%%%%%%% Kapitel * \\ Titel
%\titleformat{\chapter}[display]{\bfseries\titlefont\color{cddarkblue}}{\chaptersize\smaller \chaptername\;\thechapter}{-10pt}{\chaptersize\MakeUppercase}%
%\titlespacing{\chapter}{10pt}{0pt}{10pt}%

%%%%%%%%% like break but additionally framed
\titleformat{\chapter}[frame]{\bfseries\titlefont\color{cddarkblue}}{\enskip \chaptersize \smaller \chaptername\;\thechapter \enskip}{10pt}{\chaptersize\centering\MakeUppercase}%
\titlespacing{\chapter}{0pt}{0pt}{10pt}%


%%%%%%%%% chapter.section
\counterwithin{section}{chapter}%
\titleformat*{\section}{\bfseries\titlefont\sectionsize}%{\thesection}{8pt}{}%
\titlespacing{\section}{0pt}{10pt}{5pt}
\titleformat*{\subsection}{\bfseries\titlefont\sectionsize\smaller}

%%%%%%%%% section.
%\renewcommand{\thechapter}{\Roman{chapter}}
%\titlelabel{\thetitle.\quad} % "." behind section/sub... (3. instead of 3)
%\counterwithout{section}{chapter}%
%\titleformat*{\section}{\bfseries\titlefont\sectionsize}%{\thesection}{8pt}{}%
%\titleformat*{\subsection}{\bfseries\titlefont\sectionsize\smaller}

%%%%%%%%% section
%\titlelabel{\thetitle \quad} % no "." behind section/sub... (3 instead of 3.)
%\titleformat{\section}[hang]{\bfseries\titlefont\sectionsize}{\thesection}{8pt}{}%
%\titleformat*{\section}{\bfseries\titlefont\sectionsize}
%\titleformat*{\subsection}{\bfseries\titlefont\sectionsize\smaller}

%%%%%%%%%%%%%%%%%%%%%%%%%%%%%%%%%%%%%%%%%%%%%%%%%%%%%%%%%%%%%%%%%%%
%                          HIGHLIGHTING                           %
%%%%%%%%%%%%%%%%%%%%%%%%%%%%%%%%%%%%%%%%%%%%%%%%%%%%%%%%%%%%%%%%%%%
\newcommand{\begriff}[1]{\textbf{#1}}
\newcommand{\person}[1]{\textsc{#1}}

%%%%%%%%%%%%%%%%%%%%%%%%%%%%%%%%%%%%%%%%%%%%%%%%%%%%%%%%%%%%%%%%%%%
%                             COUNTER                             %
%%%%%%%%%%%%%%%%%%%%%%%%%%%%%%%%%%%%%%%%%%%%%%%%%%%%%%%%%%%%%%%%%%%
\usepackage{chngcntr}

% automatic reset of section after chapter ended 
\pretocmd{\chapter}{\setcounter{section}{0}}{}{}

% automatic reset of equation counter in each section
\pretocmd{\chapter}{\setcounter{equation}{0}}{}{}

%\counterwithin{theorem}{chapter}
%\counterwithin{definition}{chapter}
%\counterwithin{satz}{chapter}
%\counterwithin{lemma}{chapter}
%\counterwithin{proposition}{chapter}
%\counterwithin{folgerung}{chapter}
%\counterwithin{korollar}{chapter}
%\counterwithin{beispiel}{chapter}
%\counterwithin{erinnerung}{chapter}
%\counterwithin{wiederholung}{chapter}
%\counterwithin{bemerkung}{chapter}
%\counterwithin{anmerkung}{chapter}
%\counterwithin{algorithmus}{chapter}

%%%%%%%%%%%%%%%%%%%%%%%%%%%%%%%%%%%%%%%%%%%%%%%%%%%%%%%%%%%%%%%%%%%
%                          ENUMERATIONS                           %
%%%%%%%%%%%%%%%%%%%%%%%%%%%%%%%%%%%%%%%%%%%%%%%%%%%%%%%%%%%%%%%%%%%
\usepackage{enumerate}
\usepackage[inline]{enumitem} 		%customize label

\renewcommand{\labelitemi}{\raisebox{1pt}{\scalebox{.6}{$\blacksquare$}}}
\renewcommand{\labelitemii}{$\vartriangleright$}
\renewcommand{\labelitemiii}{--}
% Variantionen des Dreiecks als Aufzählungszeichen $\blacktriangleright$ / $\vartriangleright$ / $\triangleright$

\renewcommand{\labelenumi}{(\arabic{enumi})}
\renewcommand{\labelenumii}{\alph{enumii}.}
\renewcommand{\labelenumiii}{\roman{enumiii}.}
%%%%%%%%%%%%%%%%%%%%%%%%%%%%%%%%%%%%%%%%%%%%%%%%%%%%%%%%%%%%%%%%%%%


%%%%%%%%%%%%%%%%%%%%%%%%%%%%%%%%%%%%%%%%%%%%%%%%%%%%%%%%%%%%%%%%%%%
%                         HEADER & FOOTER                         %
%%%%%%%%%%%%%%%%%%%%%%%%%%%%%%%%%%%%%%%%%%%%%%%%%%%%%%%%%%%%%%%%%%%
\newcommand*{\rightinfo}{Vorlesung ''Optimierung`` bei Dr. Martinovic im Wintersemester 2019/20}

\usepackage{tikz}       % needed for right info
\usetikzlibrary{calc}

\usepackage{fancyhdr} 	% customize header / footer
% Add new page-style (just footer), patch \chapter command to use this page style

\fancypagestyle{myStyle}{%
    \fancyhf{} %
    \fancyfoot[C]{\thepage} %
    \renewcommand{\headrulewidth}{0pt}     % Line at the header invisible
    \renewcommand{\footrulewidth}{0pt}     % Line at the footer visible
    \fancyhead[C]{\textcolor{gray}\leftmark} %
    \fancyhead[R]{%
        \begin{tikzpicture}[overlay,remember picture]
        \node [
        fill=none,  % Farbe des Randstreifens
        text=gray,  % Textfarbe
        font=\osfamily\normalsize,  % Einstellungen für die Schrift
        inner xsep=\footskip,       % Abstand des Textes von unten
        % maximale Textbreite = Papierhöhe - 2*Abstand des Textes von unten:
        text width={\dimexpr\paperheight-2\footskip\relax},
        align=center,
        minimum height=7mm,% Breite des Randstreifens
        anchor=south west,
        rotate=90
        ]
        at ($(current page.south east)+(-10mm,0mm)$)
        {\rightinfo};
        \end{tikzpicture}%
     }
}

\fancypagestyle{rightinfo}{%
    \fancyhf{} %
    \fancyfoot[C]{\thepage} %
    \renewcommand{\headrulewidth}{0pt}     % Line at the header invisible
    \renewcommand{\footrulewidth}{0pt}     % Line at the footer visible
    \fancyhead[R]{%
        \begin{tikzpicture}[overlay,remember picture]
        \node [
        fill=none,  % Farbe des Randstreifens
        text=gray,  % Textfarbe
        font=\sffamily\normalsize,  % Einstellungen für die Schrift
        inner xsep=\footskip,       % Abstand des Textes von unten
        % maximale Textbreite = Papierhöhe - 2*Abstand des Textes von unten:
        text width={\dimexpr\paperheight-2\footskip\relax},
        align=center,
        minimum height=7mm,% Breite des Randstreifens
        anchor=south west,
        rotate=90
        ]
        at ($(current page.south east)+(-10mm,0mm)$)
        {\rightinfo};
        \end{tikzpicture}%
     }
}

%% changes pagestyle on first page of each chapter; instead of empty page the normal footer is printed
\patchcmd{\chapter}{\thispagestyle{plain}}{\thispagestyle{rightinfo}}{}{}

\pagestyle{myStyle}
\pagenumbering{arabic}

%% remember chapter-title in \leftmark and \rightmark
%\renewcommand{\chaptermark}[1]{%
%    \markboth{\chaptername
%        \ \thechapter:\ #1}{}}
%
%% remember section title in \leftmark
%\renewcommand{\sectionmark}[1]{%
%    \markright{\thesection.\ #1}{}}
%
%%change header:
%\renewcommand{\headrulewidth}{0.75pt}
%\renewcommand{\footrulewidth}{0.3pt}
%\lhead{\rightmark}%left: section-number. section-title
%\rhead{\leftmark}%right: chapter chapternumber: chapter-title

% remove page number from part{}-pages
%\let\sv@endpart\@endpart
%\def\@endpart{\thispagestyle{empty}\sv@endpart}
%%%%%%%%%%%%%%%%%%%%%%%%%%%%%%%%%%%%%%%%%%%%%%%%%%%%%%%%%%%%%%%%%%%


%%%%%%%%%%%%%%%%%%%%%%%%%%%%%%%%%%%%%%%%%%%%%%%%%%%%%%%%%%%%%%%%%%%
%                        TABLE OF CONTENTS                        %
%%%%%%%%%%%%%%%%%%%%%%%%%%%%%%%%%%%%%%%%%%%%%%%%%%%%%%%%%%%%%%%%%%%
\usepackage{tocloft}

\renewcommand{\cfttoctitlefont}{\titlefont\Huge\bfseries}
%%%%%%%%%%%%%%%%%%%%%%%%%%%%%%%%%%%%%%%%%%%%%%%%%%%%%%%%%%%%%%%%%%%

%%%%%%%%%%%%%%%%%%%%%%%%%%%%%%%%%%%%%%%%%%%%%%%%%%%%%%%%%%%%%%%%%%%
%                            LISTINGS                             %
%%%%%%%%%%%%%%%%%%%%%%%%%%%%%%%%%%%%%%%%%%%%%%%%%%%%%%%%%%%%%%%%%%%
\usepackage{listings}

%%%%%%%%%%%%%%%%%%%%%%%%%%%%%%%%%%%%%%%%%%%%%%%%%%%%%%%%%%%%%%%%%%%
%                           REFERENCES                            %
%%%%%%%%%%%%%%%%%%%%%%%%%%%%%%%%%%%%%%%%%%%%%%%%%%%%%%%%%%%%%%%%%%%
\RequirePackage[unicode,bookmarks=true]{hyperref}
\hypersetup{
    % pdfborder={0 0 0}			% no boxed around links
    pdfborderstyle={/S/U/W 1},	% underlining insteas of boxes
    linkbordercolor=cdblue,
    urlbordercolor=cdblue
%	colorlinks,
%	citecolor=black,
%	filecolor=cddarkblue!80,
%	linkcolor=black,
%	urlcolor=cddarkblue!80
}

\RequirePackage{cleveref}
\crefname{theorem}{Theorem}{Theoreme}
\crefname{satz}{Satz}{Sätze}
\crefname{lemma}{Lemma}{Lemmata}
\crefname{aussage}{Aussage}{Aussagen}
\crefname{proposition}{Proposition}{Propositionen}
\crefname{folgerung}{Folgerung}{Folgerungen}
\crefname{korollar}{Korollar}{Korollare}
\crefname{definition}{Definition}{Definitionen}
\crefname{bemerkung}{Bemerkung}{Bemerkungen}
\crefname{beispiel}{Beispiel}{Beispiele}
\crefname{erinnerung}{Erinnerung}{Erinnerungen}
\crefname{algorithmus}{Algorithmus}{Algorithmen}

\RequirePackage{bookmark}		% pdf-bookmarks


\usepackage{../../texmf/tex/latex/referencesMathTUD}

%%%%%%%%%%%%%%%%%%%%%%%%%%%%%%%%%%%%%%%%%%%%%%%%%%%%%%%%%%%%%%%%%%%%%%%%%%%%

%---------------------------------------
% additional packages
%---------------------------------------

% none

%---------------------------------------
% general settings
%---------------------------------------

\name{Eric Kunze}
\matnr{Nummer}
\email{\href{mailto:eric.kunze@mailbox.tu-dresden.de}{\ttfamily eric.kunze@mailbox.tu-dresden.de}}

\modul{Optimierung und Numerik}
\period{Wintersemester 2019/20}

%\renewcommand{\tutor}{Dr. Legrand}
%\renewcommand{\group}{Tag x. DS, (un)gerade Woche}

\lecturer{Dr. John Martinovic}
\faculty{Mathematik}
\institute{Numerik}
\professorship{Numerik der Optimalen Steuerung}

%%%%%%%%%%%%%%%%%%%%%%%%%%%%%%%%%%%%%%%%%%%%%%%%%%%%%%%%%%%%%%%%%%%%%%%%%%%%



\undef\folge
\NewDocumentCommand{\folge}{m O{n \in \N}}{\left( #1 \right)_{#2}}
\renewcommand{\complement}{\mathsf{C}}

\newcommand{\widesim}[1][2.5]{
	\mathrel{\scalebox{#1}[1]{\ensuremath{\sim}}}
}


%%%%%%%%%%%%%%%%%%%%%%%%%%%%%%%%%%%%%%%%%%%%%%%%%%%%%%%%%%%%%%%%%%%%%%%%%%%%


\begin{document}

\makeTUtitle
    
\tableofcontents

\chapter{Einführung}
\label{chapter_1_einfuehrung}
\section{Aufgabenstellung und Grundbegriffe}

Es seien $G \subseteq \Rn$ und $\abb{f}{G}{\R}$ gegeben. In dieser Vorlesung betrachten wir Optimierungsaufgaben (OA) der Form
\begin{equation}\label{eq: oa}
	f(x) \to \min \quad \bei x \in G
\end{equation}
Man nennt
\begin{itemize}[nolistsep, topsep=-\parskip]
	\item $f$ die \begriff{Zielfunktion},
	\item $G$ den \begriff{zulässigen Bereich} und
	\item ein $x \in  G$ \begriff{zulässigen Punkt} (oder zulässige Lösung).
\end{itemize}
Ein zulässiger Punkt $x^\ast \in G$ heißt \begriff{optimal} (oder Lösung oder optimale Lösung), wenn für alle $x \in G$ die Ungleichung
\begin{equation}
	f(x^\ast) \le f(x)
\end{equation}
gilt. Falls das Problem \eqref{eq: oa} lösbar ist, so wird mit $f^\ast = f(x^\ast)$ der \begriff{Optimalwert} bezeichnet. Das Problem \eqref{eq: oa} ist ein
\begin{itemize}[nolistsep, topsep=-\parskip]
	\item \begriff{unrestringiertes} (oder freies) Optimierungsproblem, wenn $G = \Rn$ gilt,
	\item andernfalls (d.h. für $G \neq \Rn$) ein \begriff{restringiertes} Problem
\end{itemize}
und außerdem eine
\begin{itemize}[nolistsep, topsep=-\parskip]
	\item \begriff{diskrete} (oder ganzzahlige) OA (engl. integer program), falls jede Variable eine diskreten Menge angehört
	\item \begriff{kontinuierliche} (oder stetige) OA, falls alle Variablen stetige Werte annehmen
	\item gemischt ganzzahlige OA, wenn sowohl stetige als auch diskrete Variablen vorkommen.
\end{itemize}

Gilt in \eqref{eq: oa} $f(x) = \trans{c} x$ für ein $c \in \Rn$ und ist $G$ durch lineare Bedingungen beschreibbar, so heißt \eqref{eq: oa} \begriff{linear}. In diesem Fall lässt sich \eqref{eq: oa} schreiben als
\begin{equation}
	\trans{c} x \to \min \quad \bei Ax = a, Bx \le b
\end{equation}
mit geeigneten Matrizen $A$ und $B$ sowie Vektoren $a$ und $b$.

Gerade für (gemischt) ganzzahlige OA kann die Lösung der Originalaufgabe schwierig sein. Eine verwandte, jedoch im Allgemeinen leichter zu lösende Aufgabe kann  in diesen Fällen wie folgt erhalten werden:

\begin{definition}
	Wir betrachten die Optimierungsaufgaben
	\begin{itemize}[nolistsep, topsep=-\parskip]
		\item[(P)] $f(x) \to \min \quad \bei x \in D \cap E$
		\item[(Q)] $g(x) \to \min \quad \bei x \in E$
	\end{itemize}
	(Q) heißt \begriff{Relaxation} zu (P) falls $g(x) \le f(x)$ für alle $x \in D \cap E$ gilt. In vielen Fällen wird dabei $g = f$ gewählt.
\end{definition}

Der Optimalwert der Relaxation kann als Näherung (bzw. untere Schranke) für den tatsächlichen Optimalwert von (P) genutzt werden. Meistens liefert die Lösung von (Q) jedoch keinen zulässigen Puntk für (P).

\begin{satz}
	Ist $\quer{x}$ eine Lösung von (Q) und gilt $\quer{x} \in D$ sowie $f(\quer{x}) = g(\quer{x})$, dann löst $\quer{x}$ auch (P).
\end{satz}
\begin{proof}
	siehe Übung
\end{proof}

\begin{definition}
	Seien (Q1) und (Q2) Relaxationen zu (P). (Q1) heißt \begriff{stärker} (oder strenger) als (Q2), wenn die Schranke (d.h. der Optimalwert) von (Q1) größer oder gleich der Schranke (Optimalwert) von (Q2) für jede Instanz von (P) ist.
\end{definition}

\begin{*anmerkung}
	Zur Erklärung des Begriffes ''Instanz`` betrachte das folgende Beispiel.
	\begin{itemize}[nolistsep, topsep=-\parskip]
		\item Problemklasse: $\trans{c} x \to \min$
		\item Instanz der Problemklasse: $x_1 + 2x_2 - 3x_3 \to \min$
	\end{itemize}
	Eine Instanz ist also eine konkrete Belegung.
\end{*anmerkung}
\section{Beispiele zur kontinuierlichen Optimierung}

\subsection{Transportoptimierung}

Es gebe Erzeuger $i \in I = \menge{0, \dots , n}$ und Verbraucher $j \in J = \menge{1, \dots , n}$. Weiterhin seien die Kosten $c_{ij}$ für den Transport einer Einheit von $i$ nach $j$ sowie der Vorrat $a_i > 0$ und der Bedarf $b_j > 0$ für alle $i$ und $j$ gegeben. Wie muss der Transport organisiert werden, damit die Gesamtkosten minimal sind?

Für jedes mathematische Modell einer OA braucht man
\begin{itemize}[nolistsep, topsep=-\parskip]
	\item geeignete Variablen ($\to x$)
	\item Zielfuntkion ($\to f$)
	\item Nebenbedingungen ($\to G$)
\end{itemize}

\begin{description}
	\item[Variablen] $x_{ij} \ge 0$ für alle $i \in I$ und $j \in J$ beschreibe die Einheiten, die von $i$ nach $j$ transportiert werden.
	\item[Zielfunktion] $f(x) = \sum\limits_{i \in I} \sum\limits_{j \in J} c_{ij} x_{ij} \to \min$
	\item[Nebenbedingungen] \leavevmode
	\begin{itemize}[nolistsep, topsep=-\parskip]
		\item Kapazitätsbeschränkung der Erzeuger $i \in I$: $\sum\limits_{j \in J} x_{ij} \le a_i \quad (i \in I)$
		\item Bedarfserfüllung von Verbrauchern $j \in J$: $\sum\limits_{i \in I} x_{ij} \ge b_j \quad (j \in J) $
	\end{itemize}
\end{description}

Somit können wir als Modell formulieren:
\begin{equation*}
	\begin{aligned}
	f(x) = \sum_{i \in I} \sum_{j \in J} c_{ij} x_{ij} \to \min \quad \bei &\sum_{j \in J} x_{ij} \le a_i \enskip (i \in I), \\
	&\sum_{i \in I} x_{ij} \ge b_j \enskip (j \in J), \\
	& x_{ij} \ge 0 \enskip ((i,j) \in I \times J)
	\end{aligned}
\end{equation*}
\section{Beispiele zur diskreten Optimierung}

\subsection{Das Rucksackproblem}

Gegeben seien ein Behälter (''Rucksack``) mit Kapazität $b \in \Z_+ \defeq \menge{0, 1, \dots}$ sowie $m$ Teile, die jeweils durch ein Gewicht $a_i \in \Z_+$ und einen Nutzen $c_i \in \Z_+$ beschrieben werden ($i = 1, \dots , m$). Aus dieser Menge von Objekten ist eine nutzenmaximale Teilmenge auszuwählen.


\begin{description}
	\item[Variablen] \begin{equation*}
		x_i \defeq \begin{cases}
		1 & \text{wenn Teil $i$ eingepackt wird} \\ 0 & \text{sonst}
		\end{cases} \quad (i = 1, \dots , m)
	\end{equation*}
	\item[Zielfunktion] $f(x) = \sum\limits_{i=1}^{m} c_i x_i \to \max$
	\item[Nebenbedingungen] Kapazitätsbedingung: $\sum\limits_{i=1}^{m} a_i x_i \le b$
\end{description}

Als Modell können wir somit formulieren:
\begin{equation*}
	\begin{aligned}
	f(x) = \sum_{i=1}^{m} c_i x_i \to \max \quad \bei \sum_{i=1}^m a_i x_i \le b \und x_i \in \menge{0,1} \enskip (i = 1, \dots , m)
	\end{aligned}
\end{equation*}

Aufgrund der binären Gestalt der Variablen wird das Problem auch als $0/1$-Rucksackproblem bezeichnet. Im Gegensatz dazu ist beim klassischen Rucksackproblem jedes Teil mehrfach nutzbar. In diesem Fall ist $x_i \in \Z_+$ zu fordern.

\subsection{Das Bin-Packing-Problem}

Gegeben seien (sehr große) Anzahl an Behältern der Kapazität $L$ sowie $b_i$ Teile des Gewichts oder Volumens $\ell_i$ mit $i \in I = \menge{1, \dots , m}$. Man ermittle die minimale Anzahl an Behältern, die benötigt wird, um alle Objekte zu verstauen.
Jede Packung (eines Behälters) kann als Vektor $a = (a_1 , \dots , a_m) \in \Z_+^m$ geschrieben werden, wobei $a_i$ angibt, wie oft das Teil $i$ benutzt wird. Ein solcher Vektor ist eine zulässige Packung, wenn 
\begin{equation*}
	\sum_{i=1}^m \ell_i a_i \le L
\end{equation*} 
ist.

\begin{description}
	\item[Modell nach \person{Kantorovich}] Wir benötigen 
	\begin{itemize}[nolistsep]
		\item eine obere Schranke $u \in \Z_+$ für die maximal benötigte Anzahl an Behältern
		\item $y_k = \begin{cases}
		1 & \text{wenn Rucksack } k \text{ benutzt wird} \\ 0 & \text{sonst}
		\end{cases} \quad (k = 1 , \dots , u)$
		\item $x_{ik} \in \Z_+$, die angeben, wieviele Objekte vom Typ $i$ in Rucksack $k$ gepackt werden ($(i,k) \in \menge{1, \dots , m} \times \menge{1, \dots , u}$)
	\end{itemize}
	Daraus ergibt sich nun folgendes Modell:
	\begin{equation*}
		\begin{aligned}
		f^\text{Kant}(x,y) = \sum_{k=1}^u y_k \to \min \bei \quad & \sum_{k=1}^u x_{ik} = b_i \quad &&(i = 1, \dots , m) \\
		& \sum_{i=1}^m x_{ik} \ell_i \le L * y_k \quad &&(k = 1 , \dots , u) \\
		& y_k \in \menge{0,1} \quad &&(k = 1 , \dots , u) \\
		& x_{ik} \in \Z_+ \quad &&((i,k) \in \menge{1, \dots , m} \times \menge{1, \dots , u})
		\end{aligned}
	\end{equation*}
	Die erste Nebenbedingung sorgt dafür, dass alle Teile gepackt werden; die zweite Nebenbedingung liefert die Einhaltung der Kapazität unter Berücksichtigung, dass nur bepackte Behälter gezählt werden.

	Es kann stets $u = \sum_{i=1}^m b_i$ gewählt werden. Das Auffinden besserer Schranken ist im Allgemeinen schwierig.
	Eine Relaxation kann z.B. durch $y_k \in [0,1]$ und $x_{ik} \in \R_+$ erhalten werden. Diese liefert jedoch keine guten Näherungen.
	%
	\item[Modell von Gilmore \& Gomory] Es seien $J$ eine Indexmenge aller zulässigen Packungen und $x_j \in \Z_+$ ($j \in J$) die Häufigkeit, wie oft ein Behälter nach dem durch $j$ angegebenen Schema $a^j = (a_1^j , \dots , a_m^j)$ mit $\trans{\ell} a^j \le L$ gefüllt wird.
	Daraus ergibt sich folgendes Modell:
	\begin{equation*}
		\begin{aligned}
		f^{GG}(x) = \sum_{j \in J} x_j \to \min \quad \bei \quad 
		& \sum_{j \in J} a_i^j * x_j = b_i \quad && (i = 1, \dots , m) \\
		& x_j \in \Z_+ && (j \in J)
		\end{aligned}
	\end{equation*}
	Die Nebenbedingung sorgt dafür, dass alle Teile gepackt werden.
	
	Es gibt im Allgemeinen exponentiell viele zulässige Packungen $a^j$ ($j \in J$), deren Koeffizienten allesamt in den Nebenbedingungen benötigt werden.
	
	Eine Relaxation erhält man zum Beispiel durch $x_j \in \R_+$. Diese stetige Relaxation ist sehr gut; man vermutet, dass 
	\begin{equation*}
		f^{GG, \ast} - f^{GG, \ast}_\text{relax} < 2
	\end{equation*}
	gilt.
\end{description}

Erfreulicherweise gibt es zum Gilmore-Gomory-Modell äquivalente Formulierungen, die mit einer polynomiellen Zahl von Variablen arbeiten und eine ebenso gute stetige Relaxation besitzen (z.B. Flussmodelle).


\end{document}