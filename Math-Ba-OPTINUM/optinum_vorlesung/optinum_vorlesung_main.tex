% This work is licensed under the Creative Commons
% Attribution-NonCommercial-ShareAlike 4.0 International License. To view a copy
% of this license, visit http://creativecommons.org/licenses/by-nc-sa/4.0/ or
% send a letter to Creative Commons, PO Box 1866, Mountain View, CA 94042, USA.

% (c) Eric Kunze, 2019

%%%%%%%%%%%%%%%%%%%%%%%%%%%%%%%%%%%%%%%%%%%%%%%%%%%%%%%%%%%%%%%%%%%%%%%%%%%%
% Template for lecture notes and exercises at TU Dresden.
%%%%%%%%%%%%%%%%%%%%%%%%%%%%%%%%%%%%%%%%%%%%%%%%%%%%%%%%%%%%%%%%%%%%%%%%%%%%

\documentclass[ngerman, a4paper, 11pt]{report}

\usepackage[ngerman]{babel}
\usepackage{../../texmf/tex/latex/layoutMathTUD}
\usepackage[smallequationskip]{../../texmf/tex/latex/mathworkMathTUD}

\usepackage{../../texmf/tex/latex/mathoperatorsMathTUD}
\usepackage[includechapter]{../../texmf/tex/latex/maththeorems3MathTUD}

\usepackage{../../texmf/tex/latex/titlepageMathTUD}
\usepackage{../../texmf/tex/latex/graphicsMathTUD}

%%%%%%%%%%%%%%%%%%%%%%%%%%%%%%%%%%%%%%%%%%%%%%%%%%%%%%%%%%%%%%%%%%%
%                        TITLE STYLES                             %
%%%%%%%%%%%%%%%%%%%%%%%%%%%%%%%%%%%%%%%%%%%%%%%%%%%%%%%%%%%%%%%%%%%

\usepackage{titlesec}   % change title headings look
\usepackage{chngcntr}   % modify counters
\usepackage{relsize}    % relative font size (smaller[i], larger[i], ...)

\makeatletter
\@ifpackageloaded{opensans}{}{\usepackage[scale=1]{opensans}}
\ifx\osfamily\undefined
    \newcommand*{\osfamily}{\fontfamily{fos}\selectfont}
    \DeclareTextFontCommand{\textos}{\osfamily}
\fi
\makeatother

\newcommand{\titlefont}{\osfamily}
\newcommand{\chaptersize}{\huge}
\newcommand{\sectionsize}{\LARGE}

\renewcommand{\thepart}{\Alph{part}}

% \titleformat{<command>}[<shape>]{<format>}{<label>}{<sep>}{<before-code>}[<after-code>]
% \titlespacing*{<command>}{<left>}{<before-sep>}{<after-sep>}[<right-sep>]

%%%%%%%%% Kapitel * \\ Titel
%\titleformat{\chapter}[display]{\bfseries\titlefont\color{cddarkblue}}{\chaptersize\smaller \chaptername\;\thechapter}{-10pt}{\chaptersize\MakeUppercase}%
%\titlespacing{\chapter}{10pt}{0pt}{10pt}%

%%%%%%%%% like break but additionally framed
\titleformat{\chapter}[frame]{\bfseries\titlefont\color{cddarkblue}}{\enskip \chaptersize \smaller \chaptername\;\thechapter \enskip}{10pt}{\chaptersize\centering\MakeUppercase}%
\titlespacing{\chapter}{0pt}{0pt}{10pt}%


%%%%%%%%% chapter.section
\counterwithin{section}{chapter}%
\titleformat*{\section}{\bfseries\titlefont\sectionsize}%{\thesection}{8pt}{}%
\titlespacing{\section}{0pt}{10pt}{5pt}
\titleformat*{\subsection}{\bfseries\titlefont\sectionsize\smaller}

%%%%%%%%% section.
%\renewcommand{\thechapter}{\Roman{chapter}}
%\titlelabel{\thetitle.\quad} % "." behind section/sub... (3. instead of 3)
%\counterwithout{section}{chapter}%
%\titleformat*{\section}{\bfseries\titlefont\sectionsize}%{\thesection}{8pt}{}%
%\titleformat*{\subsection}{\bfseries\titlefont\sectionsize\smaller}

%%%%%%%%% section
%\titlelabel{\thetitle \quad} % no "." behind section/sub... (3 instead of 3.)
%\titleformat{\section}[hang]{\bfseries\titlefont\sectionsize}{\thesection}{8pt}{}%
%\titleformat*{\section}{\bfseries\titlefont\sectionsize}
%\titleformat*{\subsection}{\bfseries\titlefont\sectionsize\smaller}

%%%%%%%%%%%%%%%%%%%%%%%%%%%%%%%%%%%%%%%%%%%%%%%%%%%%%%%%%%%%%%%%%%%
%                          HIGHLIGHTING                           %
%%%%%%%%%%%%%%%%%%%%%%%%%%%%%%%%%%%%%%%%%%%%%%%%%%%%%%%%%%%%%%%%%%%
\newcommand{\begriff}[1]{\textbf{#1}}
\newcommand{\person}[1]{\textsc{#1}}

%%%%%%%%%%%%%%%%%%%%%%%%%%%%%%%%%%%%%%%%%%%%%%%%%%%%%%%%%%%%%%%%%%%
%                             COUNTER                             %
%%%%%%%%%%%%%%%%%%%%%%%%%%%%%%%%%%%%%%%%%%%%%%%%%%%%%%%%%%%%%%%%%%%
\usepackage{chngcntr}

% automatic reset of section after chapter ended 
\pretocmd{\chapter}{\setcounter{section}{0}}{}{}

% automatic reset of equation counter in each section
\pretocmd{\chapter}{\setcounter{equation}{0}}{}{}

%\counterwithin{theorem}{chapter}
%\counterwithin{definition}{chapter}
%\counterwithin{satz}{chapter}
%\counterwithin{lemma}{chapter}
%\counterwithin{proposition}{chapter}
%\counterwithin{folgerung}{chapter}
%\counterwithin{korollar}{chapter}
%\counterwithin{beispiel}{chapter}
%\counterwithin{erinnerung}{chapter}
%\counterwithin{wiederholung}{chapter}
%\counterwithin{bemerkung}{chapter}
%\counterwithin{anmerkung}{chapter}
%\counterwithin{algorithmus}{chapter}

%%%%%%%%%%%%%%%%%%%%%%%%%%%%%%%%%%%%%%%%%%%%%%%%%%%%%%%%%%%%%%%%%%%
%                          ENUMERATIONS                           %
%%%%%%%%%%%%%%%%%%%%%%%%%%%%%%%%%%%%%%%%%%%%%%%%%%%%%%%%%%%%%%%%%%%
\usepackage{enumerate}
\usepackage[inline]{enumitem} 		%customize label

\renewcommand{\labelitemi}{\raisebox{1pt}{\scalebox{.6}{$\blacksquare$}}}
\renewcommand{\labelitemii}{$\vartriangleright$}
\renewcommand{\labelitemiii}{--}
% Variantionen des Dreiecks als Aufzählungszeichen $\blacktriangleright$ / $\vartriangleright$ / $\triangleright$

\renewcommand{\labelenumi}{(\arabic{enumi})}
\renewcommand{\labelenumii}{\alph{enumii}.}
\renewcommand{\labelenumiii}{\roman{enumiii}.}
%%%%%%%%%%%%%%%%%%%%%%%%%%%%%%%%%%%%%%%%%%%%%%%%%%%%%%%%%%%%%%%%%%%


%%%%%%%%%%%%%%%%%%%%%%%%%%%%%%%%%%%%%%%%%%%%%%%%%%%%%%%%%%%%%%%%%%%
%                         HEADER & FOOTER                         %
%%%%%%%%%%%%%%%%%%%%%%%%%%%%%%%%%%%%%%%%%%%%%%%%%%%%%%%%%%%%%%%%%%%
\newcommand*{\rightinfo}{Vorlesung ''Optimierung`` bei Dr. Martinovic im Wintersemester 2019/20}

\usepackage{tikz}       % needed for right info
\usetikzlibrary{calc}

\usepackage{fancyhdr} 	% customize header / footer
% Add new page-style (just footer), patch \chapter command to use this page style

\fancypagestyle{myStyle}{%
    \fancyhf{} %
    \fancyfoot[C]{\thepage} %
    \renewcommand{\headrulewidth}{0pt}     % Line at the header invisible
    \renewcommand{\footrulewidth}{0pt}     % Line at the footer visible
    \fancyhead[C]{\textcolor{gray}\leftmark} %
    \fancyhead[R]{%
        \begin{tikzpicture}[overlay,remember picture]
        \node [
        fill=none,  % Farbe des Randstreifens
        text=gray,  % Textfarbe
        font=\osfamily\normalsize,  % Einstellungen für die Schrift
        inner xsep=\footskip,       % Abstand des Textes von unten
        % maximale Textbreite = Papierhöhe - 2*Abstand des Textes von unten:
        text width={\dimexpr\paperheight-2\footskip\relax},
        align=center,
        minimum height=7mm,% Breite des Randstreifens
        anchor=south west,
        rotate=90
        ]
        at ($(current page.south east)+(-10mm,0mm)$)
        {\rightinfo};
        \end{tikzpicture}%
     }
}

\fancypagestyle{rightinfo}{%
    \fancyhf{} %
    \fancyfoot[C]{\thepage} %
    \renewcommand{\headrulewidth}{0pt}     % Line at the header invisible
    \renewcommand{\footrulewidth}{0pt}     % Line at the footer visible
    \fancyhead[R]{%
        \begin{tikzpicture}[overlay,remember picture]
        \node [
        fill=none,  % Farbe des Randstreifens
        text=gray,  % Textfarbe
        font=\sffamily\normalsize,  % Einstellungen für die Schrift
        inner xsep=\footskip,       % Abstand des Textes von unten
        % maximale Textbreite = Papierhöhe - 2*Abstand des Textes von unten:
        text width={\dimexpr\paperheight-2\footskip\relax},
        align=center,
        minimum height=7mm,% Breite des Randstreifens
        anchor=south west,
        rotate=90
        ]
        at ($(current page.south east)+(-10mm,0mm)$)
        {\rightinfo};
        \end{tikzpicture}%
     }
}

%% changes pagestyle on first page of each chapter; instead of empty page the normal footer is printed
\patchcmd{\chapter}{\thispagestyle{plain}}{\thispagestyle{rightinfo}}{}{}

\pagestyle{myStyle}
\pagenumbering{arabic}

%% remember chapter-title in \leftmark and \rightmark
%\renewcommand{\chaptermark}[1]{%
%    \markboth{\chaptername
%        \ \thechapter:\ #1}{}}
%
%% remember section title in \leftmark
%\renewcommand{\sectionmark}[1]{%
%    \markright{\thesection.\ #1}{}}
%
%%change header:
%\renewcommand{\headrulewidth}{0.75pt}
%\renewcommand{\footrulewidth}{0.3pt}
%\lhead{\rightmark}%left: section-number. section-title
%\rhead{\leftmark}%right: chapter chapternumber: chapter-title

% remove page number from part{}-pages
%\let\sv@endpart\@endpart
%\def\@endpart{\thispagestyle{empty}\sv@endpart}
%%%%%%%%%%%%%%%%%%%%%%%%%%%%%%%%%%%%%%%%%%%%%%%%%%%%%%%%%%%%%%%%%%%


%%%%%%%%%%%%%%%%%%%%%%%%%%%%%%%%%%%%%%%%%%%%%%%%%%%%%%%%%%%%%%%%%%%
%                        TABLE OF CONTENTS                        %
%%%%%%%%%%%%%%%%%%%%%%%%%%%%%%%%%%%%%%%%%%%%%%%%%%%%%%%%%%%%%%%%%%%
\usepackage{tocloft}

\renewcommand{\cfttoctitlefont}{\titlefont\Huge\bfseries}
%%%%%%%%%%%%%%%%%%%%%%%%%%%%%%%%%%%%%%%%%%%%%%%%%%%%%%%%%%%%%%%%%%%

%%%%%%%%%%%%%%%%%%%%%%%%%%%%%%%%%%%%%%%%%%%%%%%%%%%%%%%%%%%%%%%%%%%
%                            LISTINGS                             %
%%%%%%%%%%%%%%%%%%%%%%%%%%%%%%%%%%%%%%%%%%%%%%%%%%%%%%%%%%%%%%%%%%%
\usepackage{listings}

%%%%%%%%%%%%%%%%%%%%%%%%%%%%%%%%%%%%%%%%%%%%%%%%%%%%%%%%%%%%%%%%%%%
%                           REFERENCES                            %
%%%%%%%%%%%%%%%%%%%%%%%%%%%%%%%%%%%%%%%%%%%%%%%%%%%%%%%%%%%%%%%%%%%
\RequirePackage[unicode,bookmarks=true]{hyperref}
\hypersetup{
    % pdfborder={0 0 0}			% no boxed around links
    pdfborderstyle={/S/U/W 1},	% underlining insteas of boxes
    linkbordercolor=cdblue,
    urlbordercolor=cdblue
%	colorlinks,
%	citecolor=black,
%	filecolor=cddarkblue!80,
%	linkcolor=black,
%	urlcolor=cddarkblue!80
}

\RequirePackage{cleveref}
\crefname{theorem}{Theorem}{Theoreme}
\crefname{satz}{Satz}{Sätze}
\crefname{lemma}{Lemma}{Lemmata}
\crefname{aussage}{Aussage}{Aussagen}
\crefname{proposition}{Proposition}{Propositionen}
\crefname{folgerung}{Folgerung}{Folgerungen}
\crefname{korollar}{Korollar}{Korollare}
\crefname{definition}{Definition}{Definitionen}
\crefname{bemerkung}{Bemerkung}{Bemerkungen}
\crefname{beispiel}{Beispiel}{Beispiele}
\crefname{erinnerung}{Erinnerung}{Erinnerungen}
\crefname{algorithmus}{Algorithmus}{Algorithmen}

\RequirePackage{bookmark}		% pdf-bookmarks


\usepackage{../../texmf/tex/latex/referencesMathTUD}

%%%%%%%%%%%%%%%%%%%%%%%%%%%%%%%%%%%%%%%%%%%%%%%%%%%%%%%%%%%%%%%%%%%%%%%%%%%%

%---------------------------------------
% additional packages
%---------------------------------------

% none

%---------------------------------------
% general settings
%---------------------------------------

\name{Eric Kunze}
\matnr{Nummer}
\email{\href{mailto:eric.kunze@mailbox.tu-dresden.de}{\ttfamily eric.kunze@mailbox.tu-dresden.de}}

\modul{Optimierung und Numerik}
\period{Wintersemester 2019/20}

%\renewcommand{\tutor}{Dr. Legrand}
%\renewcommand{\group}{Tag x. DS, (un)gerade Woche}

\lecturer{Dr. John Martinovic}
\faculty{Mathematik}
\institute{Numerik}
\professorship{Numerik der Optimalen Steuerung}

%%%%%%%%%%%%%%%%%%%%%%%%%%%%%%%%%%%%%%%%%%%%%%%%%%%%%%%%%%%%%%%%%%%%%%%%%%%%



\undef\folge
\NewDocumentCommand{\folge}{m O{n \in \N}}{\left( #1 \right)_{#2}}
\renewcommand{\complement}{\mathsf{C}}

\newcommand{\widesim}[1][2.5]{
	\mathrel{\scalebox{#1}[1]{\ensuremath{\sim}}}
}


%%%%%%%%%%%%%%%%%%%%%%%%%%%%%%%%%%%%%%%%%%%%%%%%%%%%%%%%%%%%%%%%%%%%%%%%%%%%


\begin{document}

\makeTUtitle
    
\tableofcontents

\chapter{Einführung}
\label{chapter_1_einfuehrung}
\section{Aufgabenstellung und Grundbegriffe}

Es seien $G \subseteq \Rn$ und $\abb{f}{G}{\R}$ gegeben. In dieser Vorlesung betrachten wir Optimierungsaufgaben (OA) der Form
\begin{equation}\label{eq: oa}
	f(x) \to \min \quad \bei x \in G
\end{equation}
Man nennt
\begin{itemize}[nolistsep, topsep=-\parskip]
	\item $f$ die \begriff{Zielfunktion},
	\item $G$ den \begriff{zulässigen Bereich} und
	\item ein $x \in  G$ \begriff{zulässigen Punkt} (oder zulässige Lösung).
\end{itemize}
Ein zulässiger Punkt $x^\ast \in G$ heißt \begriff{optimal} (oder Lösung oder optimale Lösung), wenn für alle $x \in G$ die Ungleichung
\begin{equation}
	f(x^\ast) \le f(x)
\end{equation}
gilt. Falls das Problem \eqref{eq: oa} lösbar ist, so wird mit $f^\ast = f(x^\ast)$ der \begriff{Optimalwert} bezeichnet. Das Problem \eqref{eq: oa} ist ein
\begin{itemize}[nolistsep, topsep=-\parskip]
	\item \begriff{unrestringiertes} (oder freies) Optimierungsproblem, wenn $G = \Rn$ gilt,
	\item andernfalls (d.h. für $G \neq \Rn$) ein \begriff{restringiertes} Problem
\end{itemize}
und außerdem eine
\begin{itemize}[nolistsep, topsep=-\parskip]
	\item \begriff{diskrete} (oder ganzzahlige) OA (engl. integer program), falls jede Variable eine diskreten Menge angehört
	\item \begriff{kontinuierliche} (oder stetige) OA, falls alle Variablen stetige Werte annehmen
	\item gemischt ganzzahlige OA, wenn sowohl stetige als auch diskrete Variablen vorkommen.
\end{itemize}

Gilt in \eqref{eq: oa} $f(x) = \trans{c} x$ für ein $c \in \Rn$ und ist $G$ durch lineare Bedingungen beschreibbar, so heißt \eqref{eq: oa} \begriff{linear}. In diesem Fall lässt sich \eqref{eq: oa} schreiben als
\begin{equation}
	\trans{c} x \to \min \quad \bei Ax = a, Bx \le b
\end{equation}
mit geeigneten Matrizen $A$ und $B$ sowie Vektoren $a$ und $b$.

Gerade für (gemischt) ganzzahlige OA kann die Lösung der Originalaufgabe schwierig sein. Eine verwandte, jedoch im Allgemeinen leichter zu lösende Aufgabe kann  in diesen Fällen wie folgt erhalten werden:

\begin{definition}
	Wir betrachten die Optimierungsaufgaben
	\begin{itemize}[nolistsep, topsep=-\parskip]
		\item[(P)] $f(x) \to \min \quad \bei x \in D \cap E$
		\item[(Q)] $g(x) \to \min \quad \bei x \in E$
	\end{itemize}
	(Q) heißt \begriff{Relaxation} zu (P) falls $g(x) \le f(x)$ für alle $x \in D \cap E$ gilt. In vielen Fällen wird dabei $g = f$ gewählt.
\end{definition}

Der Optimalwert der Relaxation kann als Näherung (bzw. untere Schranke) für den tatsächlichen Optimalwert von (P) genutzt werden. Meistens liefert die Lösung von (Q) jedoch keinen zulässigen Puntk für (P).

\begin{satz}
	Ist $\quer{x}$ eine Lösung von (Q) und gilt $\quer{x} \in D$ sowie $f(\quer{x}) = g(\quer{x})$, dann löst $\quer{x}$ auch (P).
\end{satz}
\begin{proof}
	siehe Übung
\end{proof}

\begin{definition}
	Seien (Q1) und (Q2) Relaxationen zu (P). (Q1) heißt \begriff{stärker} (oder strenger) als (Q2), wenn die Schranke (d.h. der Optimalwert) von (Q1) größer oder gleich der Schranke (Optimalwert) von (Q2) für jede Instanz von (P) ist.
\end{definition}

\begin{*anmerkung}
	Zur Erklärung des Begriffes ''Instanz`` betrachte das folgende Beispiel.
	\begin{itemize}[nolistsep, topsep=-\parskip]
		\item Problemklasse: $\trans{c} x \to \min$
		\item Instanz der Problemklasse: $x_1 + 2x_2 - 3x_3 \to \min$
	\end{itemize}
	Eine Instanz ist also eine konkrete Belegung.
\end{*anmerkung}
\section{Beispiele zur kontinuierlichen Optimierung}

\subsection{Transportoptimierung}

Es gebe Erzeuger $i \in I = \menge{0, \dots , n}$ und Verbraucher $j \in J = \menge{1, \dots , n}$. Weiterhin seien die Kosten $c_{ij}$ für den Transport einer Einheit von $i$ nach $j$ sowie der Vorrat $a_i > 0$ und der Bedarf $b_j > 0$ für alle $i$ und $j$ gegeben. Wie muss der Transport organisiert werden, damit die Gesamtkosten minimal sind?

Für jedes mathematische Modell einer OA braucht man
\begin{itemize}[nolistsep, topsep=-\parskip]
	\item geeignete Variablen ($\to x$)
	\item Zielfuntkion ($\to f$)
	\item Nebenbedingungen ($\to G$)
\end{itemize}

\begin{description}
	\item[Variablen] $x_{ij} \ge 0$ für alle $i \in I$ und $j \in J$ beschreibe die Einheiten, die von $i$ nach $j$ transportiert werden.
	\item[Zielfunktion] $f(x) = \sum\limits_{i \in I} \sum\limits_{j \in J} c_{ij} x_{ij} \to \min$
	\item[Nebenbedingungen] \leavevmode
	\begin{itemize}[nolistsep, topsep=-\parskip]
		\item Kapazitätsbeschränkung der Erzeuger $i \in I$: $\sum\limits_{j \in J} x_{ij} \le a_i \quad (i \in I)$
		\item Bedarfserfüllung von Verbrauchern $j \in J$: $\sum\limits_{i \in I} x_{ij} \ge b_j \quad (j \in J) $
	\end{itemize}
\end{description}

Somit können wir als Modell formulieren:
\begin{equation*}
	\begin{aligned}
	f(x) = \sum_{i \in I} \sum_{j \in J} c_{ij} x_{ij} \to \min \quad \bei &\sum_{j \in J} x_{ij} \le a_i \enskip (i \in I), \\
	&\sum_{i \in I} x_{ij} \ge b_j \enskip (j \in J), \\
	& x_{ij} \ge 0 \enskip ((i,j) \in I \times J)
	\end{aligned}
\end{equation*}


\end{document}