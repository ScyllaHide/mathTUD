\section{Das duale Simplexverfahren}

Nach \cref{aussage: 3.4} ist ein Tableau
\begin{center}
	\begin{tabular}{r|c|c}
		$T_0$ & $x_N$ & $1$ \\ \hline
		$x_B = $ & $P$ & $p$ \\ \hline
		$z =$ & $\trans{q}$ & $q_0$
	\end{tabular}
\end{center}
optimal, wenn $p \ge 0$ und $q \ge 0$ gelten. Nach Konstruktion gilt beim primalen Simplexverfahren stets $p \ge 0$.
Sei nun ein Tableau $T_0$ gegeben mit $q \ge 0$, aber \textit{nicht} $p \ge 0$, d.h. es gibt eine Zeile $\sigma \in I_B$ mit $p_\sigma < 0$.
Die zu $T_0$ gehörige Basislösung ist dann \textit{nicht} zulässig. Mithilfe des dualen Simplexverfahrens lässt sich jedoch (unter Beibehaltung von $q \ge 0$) eine zulässige Basislösung (d.h. mit "$p \ge 0$) erzeugen.
Entsprechend der bekannten Austauschregeln ergeben sich folgende Bedingungen:
\begin{equation*}
	\begin{alignedat}{2}
		\schlange{q}_j &\defeq q_j - \frac{P_{\sigma, j}}{P_{\sigma, \tau}} q_\tau &\overset{!}&{\ge} 0 \qquad \forall j \in I_N \setminus \menge{\tau} \\
		\schlange{q}_\tau &\defeq \frac{q_\tau}{P_{\sigma, \tau}} &\overset{!}&{\ge} 0  \\
		\schlange{p}_\sigma &\defeq - \frac{p_\sigma}{P_{\sigma, \tau}} &\overset{!}&{\ge} 0
	\end{aligned}
\end{equation*}
Wegen $p_\sigma < 0$ und $q_\tau \ge 0$ ist somit ein Pivotelement mit $P_{\sigma, \tau} > 0$ zu wählen. Zur Sicherstellung von $\schlange{q}_j \ge 0$ für alle $j \in I_N \setminus \menge{\tau}$ muss ferner gelten
\begin{equation*}
	\frac{q_\tau}{P_{\sigma, \tau}} = \min \menge{\frac{q_j}{P_{\sigma, j}} \colon P_{\sigma, j} > 0, j \in I_N}
\end{equation*}
 Die eigentlichen Austauschregeln sind analog zu denen des primalen Simplexverfahrens.