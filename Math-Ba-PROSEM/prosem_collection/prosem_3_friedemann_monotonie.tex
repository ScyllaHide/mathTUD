\chapter{Konvexität \& Monotonie}

\vortragender{Friedemann Krannich}

\section{Wiederholung}
	Wir erinnern uns an die folgenden Sätze und Definitionen:
	\begin{definition}[Konvexität]
		Sei $X\subseteq\mathbb{R}^n$ konvex, $f : X\to\mathbb{R}$ eine Funktion.
		\begin{enumerate}
			\item $f$ ist konvex (auf X), falls $\forall{}x,y\in{}X \ \forall{} \lambda{}\in{}(0,1):$ \\
			\begin{displaymath}
			f(\lambda{}x+(1-\lambda)y)\leq\lambda{}f(x)+(1-\lambda)f(y)
			\end{displaymath}
			\item $f$ ist strikt konvex (auf X): nutze in 1. \(<\) statt $\leq$
			\item $f$ ist gleichmäßig konvex (auf X), falls $\exists{}\mu{}>0: \ \forall{}x,y\in{}X  \ \forall{}\lambda{}\in(0,1):$ \\
			\begin{equation*}
			f(\lambda{}x+(1-\lambda{})y)+\mu{}\lambda{}(1-\lambda{})\|x-y\|\leq\lambda{}f(x)+(1-\lambda{})f(y)
			\end{equation*}
		\end{enumerate}
	\end{definition}
	
	\begin{satz}[Zusammenhang Konvexität und Ableitung einer Funktion]
		Sei $X\subseteq\mathbb{R}^n$ offen und konvex, $f : X\to\mathbb{R}$ stetig diffbar
		\begin{enumerate}
			\item $f$ ist konvex (auf X) $ \Leftrightarrow \forall x,y \in X: 		f(x)-f(y)\geq{}f'(y)^T(x-y)$
			\item $f$ ist strikt konvex (auf X) $ \Leftrightarrow\forall x \neq y \in X : f(x)-f(y) > f'(y)^T(x-y)$
			\item $f$ ist gleichmäßig konvex (auf X) $\Leftrightarrow \exists \mu > 0 : \forall x,y \in X:
			$\\$ f(x)-f(y) \geq f'(y)^T(x-y)+ \mu \| x-y \|^2 $
		\end{enumerate}
	\end{satz}
	
	\section{Monotonie einer Funktion}
	
	\begin{definition}[Monotonie]
		Sei $X\subseteq\mathbb{R}^n$. Eine Funktion $f : X \rightarrow \mathbb{R}$\textsuperscript{n} heißt
		\begin{enumerate}
			\item monoton (auf X), falls $\forall x,y \in X:$
			\begin{equation*}
			(x-y)^T(f(x)-f(y)) \geq 0
			\end{equation*}
			\item strikt monoton (auf X), falls $\forall x,y \in X \ \textnormal{mit} \ x \neq y:$
			\begin{equation*}
			(x-y)^T(f(x)-f(y)) > 0
			\end{equation*}
			\item gleichmäßig monoton (auf X) falls $\exists \mu > 0 : \forall x,y \in X:$
			\begin{equation*}
			(x-y)^T(f(x)-f(y)) \geq \mu \| x-y \|^2
			\end{equation*}
		\end{enumerate}
	\end{definition}
	
	\begin{bemerkung}
		$f$ gleichmäßig monoton \( \Rightarrow \) $f$ strikt monoton \\ 
		$f$ strikt monoton \( \Rightarrow \) $f$ monoton
	\end{bemerkung}
	
	\begin{satz}[Zusammenhang Monotonie und Konvexität einer Funktion]
		$X\subseteq$ $\mathbb{R}$\textsuperscript{n} offen und konvex, Funktion f : $X\to\mathbb{R}$ stetig differenzierbar
		\begin{enumerate}
			\item $f$ konvex \( \Leftrightarrow \) $f'$ monoton
			\item $f$ strikt konvex \( \Leftrightarrow \) $f'$ strikt monoton
			\item $f$ gleichmäßig konvex \( \Leftrightarrow \) $f'$ gleichmäßig monoton
		\end{enumerate}
	\end{satz}
	\begin{proof}
		zu 1. und 2.: \\
		"\( \Rightarrow \)" \ : \ Sei $f$ konvex. Satz 1.2 1. liefert: \begin{equation} \label{7}
		\forall x,y \in X: f(x)-f(y) \geq f'(y)^T(x-y) \end{equation} sowie
		\begin{equation} \label{8}
		\forall x,y \in X: f(y)-f(x) \geq f'(x)^T(y-x)
		\end{equation}
		Addition von (\ref{7}) und (\ref{8}) liefert
		\begin{equation*} \label{9}
		0 \geq (f'(y)-f'(x))^T(x-y)
		\end{equation*}
		äquivalent zu
		\begin{equation*} \label{10}
		0 \geq -(f'(x)-f'(y))^T(x-y)
		\end{equation*}
		was wiederum 
		\begin{equation*} \label{11}
		(f'(x)-f'(y))^T(x-y) \geq 0
		\end{equation*} impliziert, also ist $f'$ monoton.
		Für strenge Konvexität bzw. strenge Monotonie funktioniert der Beweis analog, nutze \(>\) statt  \( \geq \) . \\
		"\( \Leftarrow \)" \ : \ Seien $x,y \in X$ beliebig aber fest. Der Mittelwertsatz liefert:
		\begin{equation} \label{12}
		\exists \theta \in (0,1): f(x)-f(y)=f'(\xi)^T(x-y) \ \textnormal{mit} \ \xi =y+ \theta (x-y) \in X
		\end{equation}
		Da $f'$ monoton ist folgt mit der Definition von \( \xi\)
		\begin{equation} \label{13}
		0 \leq (\xi -y)^T(f'( \xi )-f'(y))=\theta (x-y)^T(f'(\xi)-f'(y))
		\end{equation}
		(\ref{12}) und (\ref{13}) zusammen liefern
		\begin{equation*}
		f(x)-f(y)=f'(\xi)^T(x-y)-f'(y)^T(x-y)+f'(y)^T(x-y)
		\end{equation*}
		\begin{equation*}
		=\theta^{-1}(f'(\xi)-f'(y))^T\theta(x-y)+f'(y)^T(x-y) \geq f'(y)^T(x-y)
		\end{equation*}
		mit Satz 1.2 1. folgt dann die Konvexität von f. \\
		Der Beweis, dass strikte Monotonie von $f'$ strikte Konvexität von $f$ impliziert erfolgt analog unter Nutzung von \(>\) statt  \( \geq \) . \\
		Zu 3.: 
		"$\Rightarrow$" \ : \ $f$ gleichmäßig konvex. Satz 1.2 3. liefert $\exists \mu > 0 : \forall x,y \in X:$
		\begin{equation*}
		f(x)-f(y) \geq f'(y)^T(x-y)+\mu \| x-y \|^2
		\end{equation*}
		sowie
		\begin{equation*}
		f(y)-f(x) \geq f'(x)^T(y-x)+ \mu \| x-y \|^2
		\end{equation*}
		Addition der beiden Ungleichungen liefert:
		\begin{equation*} \label{key}
		0 \geq f'(y)^T(x-y)+f'(x)^T(y-x)+2 \mu \| x-y \|^2
		\end{equation*}
		\begin{equation*}
		\Rightarrow 0 \geq f'(y)^T(x-y)-f'(x)^T(x-y)+2\mu \| x-y \|^2
		\end{equation*}
		\begin{equation*}
		\Rightarrow 0 \geq (f'(y)-f'(x)^T(x-y)+2\mu \| x-y \|^2
		\end{equation*}
		\begin{equation*}
		\Rightarrow (x-y)(f'(x)-f'(y))^T \geq 2\mu \| x-y \|^2
		\end{equation*}
		$\Rightarrow$ $f'$ ist gleichmäßig monoton\\
		"$\Leftarrow$" \ : \ $f'$ gleichmäßig monoton, d.h. $\exists \mu > 0 :\forall x,y \in X:$
		\begin{equation*}
		(x-y)^T(f'(x)-f'(y)) \geq \mu \| x-y\|^2 
		\end{equation*}
		Sei nun $x,y \in X$ fest, $m \in \mathbb{N}$ beliebig aber fest. \\
		Definiere $t_k:= \frac{k}{m+1} \  \ k \in \{0,...,m+1\}$ .\\
		Nach dem Mittelwertsatz existiert ein  $\theta_k \in (t_k,t_{k+1}) $ mit $ \xi_k=y+\theta_k(x-y) $ mit
		\begin{equation*}
		f(y+t_{k+1}(x-y))-f(y+t_k(x-y))=(t_{k+1}-t_k)f'(\xi_k)^T(x-y)
		\end{equation*}
		\begin{displaymath}
		\Rightarrow f(x)-f(y)=\sum_{k=0}^{m}[f(y+t_{k+1}(x-y))-f(y+t_k(x-y))]
		\end{displaymath}
		\begin{displaymath}
		=\sum_{k=0}^{m}(t_{k+1}-t_k)f'(\xi_k)^T(x-y)=f'(y)^T(x-y)+\sum_{k=0}^{m}(t_{k+1}-t_k)(f'(\xi_k)-f'(y))^T(x-y)
		\end{displaymath}
		\begin{displaymath}
		=f'(y)^T(x-y)+\sum_{k=0}^{m} \frac{t_{k+1}-t_k}{\theta_k}(f'(\xi_k)-f'(y))^T(\xi_k-y) \textnormal{ nach der Definition von $\xi_k$}
		\end{displaymath}
		\begin{displaymath}
		\geq f'(y)^T(x-y)+\mu \sum_{k=0}^{m} \frac{t_{k+1}-t_k}{\theta_k} \| \xi_k -y\|^2 \textnormal { da $f'$ gleichmäßig monoton ist}
		\end{displaymath}
		\begin{displaymath}
		= f'(y)^T(x-y)+\mu \|x-y \|^2 \sum_{k=0}^{m} \theta_k(t_{k+1}-t_k) \textnormal{ nach der Definition von $\xi_k$}.
		\end{displaymath}
		Da $\theta_k \in (t_k,t_{k+1})$ folgt
		\begin{displaymath}
		\sum_{k=0}^{m} \theta_k(t_{k+1}-t_k) \geq \sum_{k=0}^{m}t_k(t_{k+1}-t_k)
		\end{displaymath}
		\begin{displaymath}
		=\sum_{k=0}^{m}\frac{k}{m+1}\left(\frac{k+1}{m+1}-\frac{k}{m+1}\right)= \frac{1}{(m+1)^2} \sum_{k=0}^{m}k = \frac{1}{2}\frac{m}{m+1}
		\end{displaymath}
		\begin{displaymath}
		\Rightarrow f(x)-f(y) \geq f'(y)^T(x-y)+\frac{1}{2}\mu\frac {m}{m+1} \| x-y \|^2
		\end{displaymath}
		Mit $m\rightarrow\infty$ folgt
		\begin{displaymath}
		f(x)-f(y) \geq f'(y)^T(x-y)+\frac{1}{2}\mu \| x-y \|^2
		\end{displaymath}
		Mit Satz 1.2 3. folgt, dass $f$ gleichmäßig konvex ist.
	\end{proof}