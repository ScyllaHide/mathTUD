\section{Risikomaße: Axiomatischer Zugang}

Sei $\mathcal{M}$ die Menge der $\R$-wertigen Zufallsvariablen auf $(\Omega, \F, \P)$. Wir wollen untersuchen, welche Abbildungen 
\begin{equation*}
	\abb{\rho}{\mathcal{M}}{\R}, L \mapsto \rho(L)
\end{equation*}
als Risikomaße geeignet sind.

\begin{*definition}[Eigenschaften von Risikomaßen]
	Seien $L_1, L_2 \in \mathcal{M}$.
	\begin{enumerate}[label=(\alph*), nolistsep]
		\item Monotonie: $L_1 \le L_2$ fast sicher $\follows \rho(L_1) \le \rho(L_2)$
		\item Translationsinvarianz: $\rho(L + m) = \rho(L) + m$ für alle $m \in \R$ (Kapitalanforderung)
		\item Subadditivität: $\rho(L_1 + L_2) \le \rho(L_1) + \rho(L_2)$ (belohnt Risikostreuung/Diversifikation)
		\item positive Homogenität: $\rho(c * L) = c * \rho(L)$ für alle $c \ge 0$
		\item Konvexität: $\rho(\gamma L_1 + (1-\gamma) L_2) \le \gamma \rho(L_1) + (1-\gamma) \rho(L_2)$ für alle $\gamma \in [0,1]$
		\item Verteilungsinvarianz: $F_{L_1} = F_{L_2} \follows \rho(L_1) = \rho(L_2)$
	\end{enumerate}
\end{*definition}

\begin{*bemerkung_inline}
	Translationsinvarianz beschreibt eine Kapitalanforderung; Subadditvität belohnt Risikostreuung/Diversifikation
	\begin{itemize}[nolistsep]
		\item Es gilt (c) $+$ (d) $\follows$ (e), denn
		\begin{equation*}
			\rho(\gamma L_1 + (1-\gamma) L_2) \overset{(c)}{\le} \rho(\gamma L_1) + \rho((1-\gamma) L_2) \overset{(d)}{=} \gamma \rho(L_1) + (1-\gamma) \rho(L_2)
		\end{equation*}
		\item Es gilt (d) $+$ (e) $\follows$ (c), denn
		\begin{equation*}
			\rho(L_1 + L_2) \overset{(d)}{=} 2 * \rho(\frac{L_1}{2} + \frac{L_2}{2}) \overset{(e)}{=} \rho(L_1) + \rho(L_2)
		\end{equation*}
	\end{itemize}
\end{*bemerkung_inline}

\begin{*definition}
	\begin{enumerate}[label=(\arabic*), nolistsep]
		\item Ein Risikomaß heißt \begriff{monetär}, wenn (a) und (b) gelten.
		\item Ein Risikomaß heißt \begriff{konvex}, wenn (a), (b) und (e) gilt.
		\item Ein Risikomaß heißt \begriff{kohärent}, wenn (a) bis (d) gilt (und damit auch (e)).
	\end{enumerate}
\end{*definition}

\begin{*bemerkung_inline}
	Offensichtlich gilt:
	\begin{equation*}
		\text{kohärente Risikomaße } \subseteq \text{ konvexe Risikomaße } \subseteq \text{ monetäre Risikomaße }
	\end{equation*}
\end{*bemerkung_inline}

\begin{*beispiel}
	\begin{itemize}
		\item Der Value-at-Risk $\VaR_\alpha$ ist monetär, positiv homogen und verteilungsinvariant, aber nicht subadditiv oder konvex. \\
		Seien $X_1, X_2$ unabhängig und identisch verteilt mit 
		\begin{equation*}
		X_i = \begin{cases}
		+1 & \text{mit Wahrscheinlichkeit } p = \frac{1}{3} \\ -1 &\text{mit Wahrscheinlichkeit } 1-p = \frac{2}{3}
		\end{cases}
		\end{equation*}
		Somit ist $\VaR_{0.5}(X_i) = -1$. Also gilt
		\begin{equation*}
		X_1 + X_2 = \begin{cases}
		+2 & \text{mit Wahrscheinlichkeit } p^2 = \frac{1}{9} \\
		0 & \text{mit Wahrscheinlichkeit } p(1-p) = \frac{2}{9} \\
		-2 & \text{mit Wahrscheinlichkeit } (1-p)^1 = \frac{4}{9}
		\end{cases}
		\end{equation*}
		d.h. $\VaR_{0.5}(X_1 + X_2) = 0$.
		\begin{equation*}
		\follows \VaR_{0.5}(X_1 + X_2) > \VaR_{0.5}(X_1) + \VaR_{0.5}(X_2) \follows \text{ nicht subadditiv}
		\end{equation*}
		\item Der Expected-Shortfall ist ein kohärentes, verteilungsinvariantes Risikomaß (siehe Übung).
	\end{itemize}
\end{*beispiel}