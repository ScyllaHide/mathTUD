%%%%%%%%%%%%%%%%%%%%%%%%%%%%%%%%%%%%%%%%%%%%%%%%%%%%%%%%%%%%%%%%%%%
%                          HIGHLIGHTING                           %
%%%%%%%%%%%%%%%%%%%%%%%%%%%%%%%%%%%%%%%%%%%%%%%%%%%%%%%%%%%%%%%%%%%
\newcommand{\begriff}[1]{\textbf{#1}}
\newcommand{\person}[1]{\textsc{#1}}

%%%%%%%%%%%%%%%%%%%%%%%%%%%%%%%%%%%%%%%%%%%%%%%%%%%%%%%%%%%%%%%%%%%
%                             COUNTER                             %
%%%%%%%%%%%%%%%%%%%%%%%%%%%%%%%%%%%%%%%%%%%%%%%%%%%%%%%%%%%%%%%%%%%
\usepackage{chngcntr}

% automatic reset of section after chapter ended 
\pretocmd{\chapter}{\setcounter{section}{0}}{}{}

% automatic reset of equation counter in each section
\pretocmd{\section}{\setcounter{equation}{0}}{}{}

\counterwithin{themcount}{chapter}

%%%%%%%%%%%%%%%%%%%%%%%%%%%%%%%%%%%%%%%%%%%%%%%%%%%%%%%%%%%%%%%%%%%
%                          ENUMERATIONS                           %
%%%%%%%%%%%%%%%%%%%%%%%%%%%%%%%%%%%%%%%%%%%%%%%%%%%%%%%%%%%%%%%%%%%
\usepackage{enumerate}
\usepackage[inline]{enumitem} 		%customize label

\renewcommand{\labelitemi}{\raisebox{1pt}{\scalebox{.6}{$\blacksquare$}}}
\renewcommand{\labelitemii}{$\vartriangleright$}
\renewcommand{\labelitemiii}{--}
% Variantionen des Dreiecks als Aufzählungszeichen $\blacktriangleright$ / $\vartriangleright$ / $\triangleright$

\renewcommand{\labelenumi}{(\arabic{enumi})}
\renewcommand{\labelenumii}{\alph{enumii}.}
\renewcommand{\labelenumiii}{\roman{enumiii}.}
%%%%%%%%%%%%%%%%%%%%%%%%%%%%%%%%%%%%%%%%%%%%%%%%%%%%%%%%%%%%%%%%%%%


%%%%%%%%%%%%%%%%%%%%%%%%%%%%%%%%%%%%%%%%%%%%%%%%%%%%%%%%%%%%%%%%%%%
%                         HEADER & FOOTER                         %
%%%%%%%%%%%%%%%%%%%%%%%%%%%%%%%%%%%%%%%%%%%%%%%%%%%%%%%%%%%%%%%%%%%
\newcommand*{\rightinfo}{Vorlesung ''Stochastik -- Finanzmathematik`` bei Prof. Dr. Keller-Ressel im Wintersemester 2019/20}

\usepackage{tikz}       % needed for right info
\usetikzlibrary{calc}

\usepackage{fancyhdr} 	% customize header / footer
% Add new page-style (just footer), patch \chapter command to use this page style

\fancypagestyle{myStyle}{%
    \fancyhf{} %
    \fancyfoot[C]{\thepage} %
    \renewcommand{\headrulewidth}{0pt}     % Line at the header invisible
    \renewcommand{\footrulewidth}{0pt}     % Line at the footer visible
    \fancyhead[C]{\textcolor{gray}\leftmark} %
    \fancyhead[R]{%
        \begin{tikzpicture}[overlay,remember picture]
        \node [
        fill=none,  % Farbe des Randstreifens
        text=gray,  % Textfarbe
        font=\osfamily\normalsize,  % Einstellungen für die Schrift
        inner xsep=\footskip,       % Abstand des Textes von unten
        % maximale Textbreite = Papierhöhe - 2*Abstand des Textes von unten:
        text width={\dimexpr\paperheight-2\footskip\relax},
        align=center,
        minimum height=7mm,% Breite des Randstreifens
        anchor=south west,
        rotate=90
        ]
        at ($(current page.south east)+(-10mm,0mm)$)
        {\rightinfo};
        \end{tikzpicture}%
     }
}

\fancypagestyle{rightinfo}{%
    \fancyhf{} %
    \fancyfoot[C]{\thepage} %
    \renewcommand{\headrulewidth}{0pt}     % Line at the header invisible
    \renewcommand{\footrulewidth}{0pt}     % Line at the footer visible
    \fancyhead[R]{%
        \begin{tikzpicture}[overlay,remember picture]
        \node [
        fill=none,  % Farbe des Randstreifens
        text=gray,  % Textfarbe
        font=\sffamily\normalsize,  % Einstellungen für die Schrift
        inner xsep=\footskip,       % Abstand des Textes von unten
        % maximale Textbreite = Papierhöhe - 2*Abstand des Textes von unten:
        text width={\dimexpr\paperheight-2\footskip\relax},
        align=center,
        minimum height=7mm,% Breite des Randstreifens
        anchor=south west,
        rotate=90
        ]
        at ($(current page.south east)+(-10mm,0mm)$)
        {\rightinfo};
        \end{tikzpicture}%
     }
}

%% changes pagestyle on first page of each chapter; instead of empty page the normal footer is printed
\patchcmd{\chapter}{\thispagestyle{plain}}{\thispagestyle{rightinfo}}{}{}

\pagestyle{myStyle}
\pagenumbering{arabic}

%% remember chapter-title in \leftmark and \rightmark
%\renewcommand{\chaptermark}[1]{%
%    \markboth{\chaptername
%        \ \thechapter:\ #1}{}}
%
%% remember section title in \leftmark
%\renewcommand{\sectionmark}[1]{%
%    \markright{\thesection.\ #1}{}}
%
%%change header:
%\renewcommand{\headrulewidth}{0.75pt}
%\renewcommand{\footrulewidth}{0.3pt}
%\lhead{\rightmark}%left: section-number. section-title
%\rhead{\leftmark}%right: chapter chapternumber: chapter-title

% remove page number from part{}-pages
%\let\sv@endpart\@endpart
%\def\@endpart{\thispagestyle{empty}\sv@endpart}
%%%%%%%%%%%%%%%%%%%%%%%%%%%%%%%%%%%%%%%%%%%%%%%%%%%%%%%%%%%%%%%%%%%


%%%%%%%%%%%%%%%%%%%%%%%%%%%%%%%%%%%%%%%%%%%%%%%%%%%%%%%%%%%%%%%%%%%
%                        TABLE OF CONTENTS                        %
%%%%%%%%%%%%%%%%%%%%%%%%%%%%%%%%%%%%%%%%%%%%%%%%%%%%%%%%%%%%%%%%%%%
\usepackage{tocloft}

\renewcommand{\cfttoctitlefont}{\titlefont\Huge\bfseries}
\setcounter{tocdepth}{1}
%%%%%%%%%%%%%%%%%%%%%%%%%%%%%%%%%%%%%%%%%%%%%%%%%%%%%%%%%%%%%%%%%%%

%%%%%%%%%%%%%%%%%%%%%%%%%%%%%%%%%%%%%%%%%%%%%%%%%%%%%%%%%%%%%%%%%%%
%                            LISTINGS                             %
%%%%%%%%%%%%%%%%%%%%%%%%%%%%%%%%%%%%%%%%%%%%%%%%%%%%%%%%%%%%%%%%%%%
\usepackage{listings}
