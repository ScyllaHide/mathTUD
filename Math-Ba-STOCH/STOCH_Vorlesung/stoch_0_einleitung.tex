\chapter{Einleitung}

\section*{Literatur}
\begin{itemize}[nolistsep]
    \item \textit{Georgii} : Stochastik (5. Auflage)
    \item \textit{Schilling} : Wahrscheinlichkeit
    \item \textit{Bauer} : Wahrscheinlichkeitstheorie (5. Auflage)
    \item \textit{Krengel} : Einführung in die W-Theorie und Statistik
    \item \textit{Dehling \& Haupt} : Einführung in die W-Theorie und Statistik
\end{itemize}

\section*{Was ist Stochastik ?}

Altgriechisch "Stochastikos" ($\sigma \tau  \chi \alpha \tau \iota \kappa  \zeta$) $\leadsto$ ``scharfsinnig im Vermuten'' % TODO

Fragestellungen stammen insbesondere aus dem Glücksspiel, heute vielmehr auch aus der Ver"-sicherungs- und Finanzmathematik - überall da, wo Zufall / Risiko / Chance auftaucht.

\subsection*{Was ist mathematische Stochastik ?}
\begin{itemize}[leftmargin=*]
    \item Beschreibt zufällige Phänomene in einer exakten Sprache. \\
    Bsp.: "Beim Würfeln erscheint jedes sechste Mal (im Schnitt) die Augenzahl 6" $\leadsto$ Gesetz der großen Zahlen
    \item lässt sich in zwei Teilgebiete unterteilen: Wahrscheinlichkeitstheorie \& Statistik \\
    Die W-Theorie beschreibt und untersucht konkret gegebene Zufallssituationen. Dagegen zieht die Statistik Schlussfolgerungen aus Beobachtungen. Dabei benötigt sie die Modelle der W-Theorie - umgekehrt benötigt auch die W-Theorie die Statistik zur Bestätigung der Modelle.
    \item In diesem Semester konzentrieren wir uns auf die Wahrscheinlichkeitstheorie.
\end{itemize}

