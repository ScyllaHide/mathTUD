\chapter{Wahrscheinlichkeitsräume}
\section{Grundbegriffe der Wahrscheinlichkeitstheorie}
\subsection*{Ergebnisraum}
Welche möglichen Ausgänge eines zufälligen Geschehens interessieren uns?
\begin{*beispiel}
    Würfeln: Augenzahl, aber nicht Lage, Fallhöhe, usw.
\end{*beispiel}

\begin{definition}[Ergebnisraum]
    Die Menge der relevanten Ergebnisse eines Zufallgeschehens nennen wir \begriff{Ergebnisraum} und bezeichnen diesen mit $\Omega$.
\end{definition}

\begin{*beispiel}
    \begin{itemize}
        \item Würfeln: $\Omega = \menge{1,2, \dots , 6}$
        \item Wartezeiten: $\Omega = \R_+ = [0,\infty)$ (also überabzählbar)
    \end{itemize}
\end{*beispiel}

\subsection*{Ereignisse}
Oft interessiert man sich gar nicht für das konkrete Ergebnis des Zufallsexperiments, sondern nur für das Eintreten gewisser Ereignisse.

\begin{*beispiel}
    Würfeln: Zahl ist $> 3$ \\
    Wartezeiten: Wartezeit ist $\leq 5$ Minuten
\end{*beispiel}

Wir wollen also Teilmengen des Ergebnisraums betrachten, d.h. Elemente von $\pows{\Omega}$ (Potenzmenge), denen eine Wahrscheinlichkeit zugeordnet werden kann d.h. welche \textit{messbar} sind.

\begin{definition}[Ereignisraum]
    Sei $\Omega \neq \emptyset$ ein Ergebnisraum und $\mathcal{F}$ eine $\sigma$-Algebra auf $\Omega$, d.h. eine Familie von Teilmengen von $\Omega$, sodass 
    \begin{enumerate}
        \item $\Omega \in \mathcal F$
        \item $A \in \mathcal{F} \follows A^\complement \in \mathcal{F}$
        \item $A_1, A_2, \dots \in \mathcal{F} \follows \bigcup_{i \geq 1} A_i \in \mathcal{F}$
    \end{enumerate}
    Dann heißt $(\Omega, \mathcal{F})$ \begriff{Ereignisraum} oder messbarer Raum.
\end{definition}

\subsection*{Wahrscheinlichkeit}
Wir ordnen nun den Ereignissen Wahrscheinlichkeiten mittels einer Abbildung $\abb{\mathbb{P}}{\mathcal{F}}{[0,1]}$
zu, sodass
\begin{enumerate}
    \item[(N)] Normierung: $\mathbb{P}(\Omega) = 1$
    \item[(A)] Additivität: Für paarweise disjunkte Ereignisse $A_1, A_2, \dots \in \mathcal{F}$ ist $\mathbb{P}\left(\bigcup_{i \geq 1} A_i\right) = \sum_{i \geq 0} \mathbb{P}(A_i)$.
\end{enumerate}

(N), (A) und die Nichtnegativität von $\mathbb{P}$ werden als Kolmogorov-Axiome bezeichnet (nach Kolmogorov: Grundbegriffe der Wahrscheinlichkeitstheorie, 1933).

\begin{definition}[Wahrscheinlichkeit]
    Sei $(\Omega, \mathcal{F})$ ein Ereignisraum und $\abb{\mathbb{P}}{\mathcal{F}}{[0,1]}$ eine Abbildung mit den Eigenschaften (N) und (A). Dann heißt $\mathbb{P}$ \begriff{Wahrscheinlichkeitsmaß} oder auch \begriff{Wahrscheinlichkeitsverteilung}.
\end{definition}

Aus der Definition folgen direkt die folgenden Eigenschaften:

\begin{satz}[Rechenregelen für Wahrscheinlichkeitsmaße]
    Sei $\mathbb{P}$ ein W-Maß auf einem Ereignisraum $(\Omega, \mathcal{F})$ und $A,B,A_1,A_2,\dots \in \mathcal{F}$. Dann gilt:
    \begin{enumerate}[leftmargin=*]
        \item $\mathbb{P}(\emptyset) = 0$
        \item Endliche Additivität: $\mathbb{P} (A \cup B) + \mathbb{P} (A \cap B) = \mathbb{P}(A) + \mathbb{P}(B)$ und $\mathbb{P}(A) + \mathbb{P}(A^\complement) = 1$
        \item Monotonie: $A \subseteq B \follows \mathbb{P}(A) \leq \mathbb{P}(B)$
        \item $\sigma$-Subadditivität: $\mathbb{P}\left(\bigcup_{i \geq 1} A_i \right)  \leq \sum_{i \geq 1} \mathbb{P}(A_i)$
        \item $\sigma$-Stetigkeit: Wenn $A_n \nearrow A$ (d.h. $A_1 \subseteq A_2 \subseteq \cdots$ und $A = \bigcup_{i=1}^{\infty}A_i$ ) oder $A_n \searrow A$, so gilt $\mathbb{P}(A_n) \to \mathbb{P}(A)$ für $n \to \infty$
    \end{enumerate}
\end{satz}
\begin{proof}
    siehe MINT oder Schillings Lehrbuch
\end{proof}

\begin{beispiel}
    Für einen beliebigen Ereignisraum $(\Omega, \mathcal{F})$ und ein beliebiges Element $\xi \in \Omega$ definiert
    \begin{align*}
        \delta_\xi(A) := \begin{cases} 1 & \xi \in A \\ 0 & \text{sonst} \end{cases}
    \end{align*}
    ein (degeneriertes) W-Maß auf $(\Omega, \mathcal{F})$, welches wir als \begriff{Dirac-Maß} oder Dirac-Verteilung bezeichnen.
\end{beispiel}

\begin{beispiel}
    Wir betrachten das Zufallsexperiment "Würfeln mit einem fairen, 6-seitigen Würfel" mit der Ergebnismenge $\Omega = \menge{1, \dots, 6}$ und Ereignisraum $\mathcal{F} = \pows{\Omega}$. Setzen wir aus Symmetriegründen
    \begin{align*}
        \mathbb{P}(A) = \frac{\# A}{6}
    \end{align*}
    mit $\# A = \card{A} = \text{Kardinalität}$. Dies definert ein W-Maß.
\end{beispiel}

\begin{beispiel}[Wartezeiten an der Bushaltestelle]
    Ergebnisraum $\Omega = \R_+$ und Ereignisraum Borel'sche $\sigma$-Algebra $\mathcal{F} = \mathcal{B}(\R_+)$. Ein mögliches W-Maß können wir durch
    \begin{align*}
        \mathbb{P}(A) \defeq \int_A \lambda e^{-\lambda x} \dx
    \end{align*}
    für einen Parameter $\lambda > 0$ festlegen. (offensichtlich gelten $\mathbb{P}(\Omega) = 1$ und die $\sigma$-Additivität aufgrund der $\sigma$-Additivität des Integrals). Wir bezeichnen dieses Maß als \begriff{Exponentialverteilung}. (Warum gerade dieses Maß für Wartezeiten gut geeigent ist, sehen wir später.)
\end{beispiel}

