\begin{exercisePage}[Wahrscheinlichkeit, Zufallsvariablen, Verteilungsfunktionen]
    \begin{lemma} \label{lemma: 1_1.1_schnitt}
        Sei $(\Omega, \ereignisF)$ ein messbarer Raum. Dann gilt
        \begin{equation*}
            \folge{A_n} \subseteq \ereignisF \follows \bigcap_{n \in \N} A_n \in \ereignisF
        \end{equation*}
    \end{lemma}
    \begin{proof}
        Da $\folge{A_n} \subseteq \ereignisF$ ist auch $\folge{A_n^\complement} \subseteq \ereignisF$. Dann ist $\bigcap_{n \in \N} A_k = \left( \bigcup_{n \in \N} A_k^\complement \right)^\complement$ und somit weil $\bigcup_{n \in \N} A_k^\complement \in \ereignisF$ und nach Definition einer $\sigma$-Algebra auch $\bigcap_{n \in \N} A_n \in \ereignisF$. Insbesondere gilt dies auch falls endliche viele Mengen weggelassen werden, d.h. für alle $n \in \N$ ist auch $\bigcap_{k \geq n} A_k \in \ereignisF$.
    \end{proof}

    \begin{exercise}
        \begin{enumerate}[leftmargin=*, label=(\alph*)]
            \item Sei ($\Omega, \ereignisF$) ein messbarer Raum. Zeigen Sie, dass für jede Menge $B \subseteq \Omega$ das Mengensystem $\quer{F} \defeq \menge{A \cap B \colon A \in \ereignisF}$ eine $\sigma$-Algebra über dem Grundraum $B$ ist.
            \item Es sei ($\Omega, \ereignisF$) ein messbarer Raum und $A_1, A_2, \dots \in \ereignisF$ eine Folge von Ereignissen. Zeigen Sie:
            \begin{equation*}
                \underline{A} \defeq \liminf_{n \to \infty} A_n = \bigcup_{n=1}^\infty \bigcap_{k=n}^\infty A_k \in \ereignisF \qquad \overline{A} \defeq \limsup_{n \to \infty} A_n = \bigcap_{n=1}^\infty \bigcup_{k=n}^\infty A_k \in \ereignisF \qquad \underline{A} \subseteq \overline{A}
            \end{equation*}
        \end{enumerate}
    \end{exercise}

    \begin{enumerate}[leftmargin=*, label=(\alph*)]
        \item Wir zeigen die drei Eigenschaften einer $\sigma$-Algebra.
            \begin{itemize}[leftmargin=*]
                \item Da $\ereignisF$ eine $\sigma$-Algebra ist, ist $\Omega \in \ereignisF$. Wählen wir also $A = \Omega \in \ereignisF$, so ist $B = \Omega \cap B$, da $B \subseteq \Omega$ und somit $\Omega \in \quer{\ereignisF}$.
                \item Sei $C \in \quer{\ereignisF}$, d.h. es existiert $A \in \ereignisF$ mit $C = A \cap B$. Dann ist $C^\complement = B \setminus C = B \setminus (A \cap B) = (B \setminus A) \cap B = A^\complement \cap B \in \quer{\ereignisF}$.
                \item Seien $\folge{C_n} \subseteq \quer{\ereignisF}$, d.h. es existiert $\folge{A_n} \subseteq \ereignisF$, sodass für alle $n \in \N$ gilt, dass $C_n = A_n \cap B$. Dann ist
                \begin{equation*}
                    \bigcup_{n \in \N} C_n = \bigcup_{n \in \N} (A_n \cap B) =  \underbrace{ \left( \bigcup_{n \in \N} A_n \right)}_{\in \ereignisF}  \cap \, B \in \quer{\ereignisF}
                \end{equation*}
            \end{itemize}
        \item Sei $\folge{A_n} \subseteq \ereignisF$. 
            \begin{itemize}[leftmargin=*]
                \item \cref{lemma: 1_1.1_schnitt} sagt uns, dass für alle $n \in \N$ auch $\bigcap_{k \geq n} A_k \in \ereignisF$ gilt. Mit der Definition von $\ereignisF$ folgt schließlich, dass auch $\bigcup_{n \in \N} \bigcap_{k \geq n} A_k \in \ereignisF$ ist.
                \item Analog zum ersten Punkt ist $\bigcap_{n \in \N} \bigcup_{k \geq n} A_k \in \ereignisF$, da sowohl die (abzählbare) Vereinigung als auch der (abzählbare) Schnitt in $\ereignisF$ liegen.
                \item Für alle $n,m \in \N$ gilt $\bigcap_{k \geq n} A_k \subseteq A_{\max \menge{n,m}} \subseteq \bigcup_{k \geq m} A_k$. Somit ist für alle $m \in \N$ auch $\bigcup_{n \in \N} \bigcap_{k \geq n} A_k \subseteq A_k \subseteq \bigcup_{k \geq m} A_k$ und dann $\bigcup_{n \in \N} \bigcap_{k \geq n} A_k \subseteq \bigcap_{m \in \N} \bigcup_{k \geq m} A_k$, was bereits $\underline{A} \subseteq \overline{A}$ zeigt.
            \end{itemize}
    \end{enumerate}

%%%% AUFGABE 1.2 %%%%%%%%%%%%%%%%%%%%%%%%%%%%%%%%%%%%%%%%%%%%%%%%%%%%%%%%%%%%%%%%%
    \begin{exercise}
        Es sei $X$ eine stetige, reelle Zufallsvariable mit Dichtefunktion
        \begin{equation*}
            f(x) := \begin{cases} c * (4x - 2x^2) & \text{falls } 0 < x < 2 \\ 0 & \text{sonst} \end{cases}
        \end{equation*}
        \begin{enumerate}[leftmargin=*, label=(\alph*), nolistsep]
            \item Bestimmen Sie $c$.
            \item Berechnen Sie $\P(X > 1)$ und $\P(X=1)$
            \item Bestimmen Sie die zugehörige Verteilungsfunktion.
        \end{enumerate}
    \end{exercise}

    \begin{enumerate}[leftmargin=*, label=(\alph*)]
        \item Sei $\P$ das zur Dichte $f$ gehörende Wahrscheinlichkeitsmaß. Dann gilt nach Satz 1.8
        \begin{equation*}
        \P(A) = \int_A f(x) \dx \quad \text{für alle } A \in \borel\R
        \end{equation*}
        Somit muss auch insbesondere die Normierung erfüllt sein, d.h.
        \begin{equation*}
        \P(\R) = \int_\R f(x) \dx \overset{!}{=} 1
        \end{equation*}
        Die Additivität bleibt durch die Integraleigenschaften erhalten. Also:
        \begin{equation*}
        \begin{aligned}
        \int_\R f(x) \dx &= \int_{-\infty}^0 f(x) \dx + \int_0^2 f(x) \dx + \int_2^\infty f(x) \dx \\
        &= \int_{-\infty}^0 0 \dx + \int_0^2 f(x) \dx + \int_2^\infty 0 \dx \\
        &= \int_0^2 c * (4x - 2x^2) \dx 
        \enskip = \enskip 2c \left[ x^2 - \frac{1}{3} x^3 \right]_0^2
        \enskip = \enskip 2c \left( 4 - \frac{8}{3} \right) \\
        &= \frac{8}{3} c \enskip \overset{!}{=} \enskip 1 
        \follows c = \frac{3}{8}
        \end{aligned}
        \end{equation*}
        %
        \item 
        \begin{equation*}
            \begin{aligned}
            \P(X > 1) &= \int_1^\infty f(x) \dx 
            = \int_1^2 f(x) \dx + \int_2^\infty 0 \dx \\
            &= \int_1^2 \frac{3}{2} x - \frac{3}{4} x^2 \dx
            = \left[ \frac{3}{4} x^2 - \frac{3}{12} x^3 \right]_1^2 \\
            &= \left( 3-2 \right) - \left( \frac{3}{4} - \frac{3}{12} \right) 
            = 1 - \frac{1}{2} = \frac{1}{2} \\
            %
            \P(X=1) &= \int_1^1 f(x) \dx = 0              
            \end{aligned}
        \end{equation*}
        %
        \item Die Verteilungsfunktion einer \ZV ist definiert als 
        \begin{equation*}
            F_X(x) \defeq \P(X \leq x) = \P( (-\infty, x]) = \int_{-\infty}^x f(\xi) \diff{\xi}
        \end{equation*}
        Ist $x \leq 0$, so ist $F_X(x) = 0$, da $f(x) = 0$ für alle $x \leq 0$ ist und somit $\int_{-\infty}^x f(\xi) \diff{\xi} = \int_{-\infty}^x 0 \diff{\xi} = 0$. Für $x \in (0,2)$ ist 
        \begin{equation*}
            F_X(x) = \int_{-\infty}^x f(\xi) \diff{\xi} = \int_0^x f(\xi) \diff{\xi} = \left[ \frac{3}{4} \xi^2 - \frac{3}{12} \xi^3 \right]_0^x = \frac{3}{4} x^2 - \frac{3}{12} x^3
        \end{equation*}
        Für $x \geq 2$ ist stets $(0,2) \subset (-\infty, x)$ und daher ist
        \begin{equation*}
            \begin{aligned}
            F_X(x) = \int_{-\infty}^x f(\xi) \diff{\xi} 
            &= \int_{-\infty}^0 f(\xi) \diff{\xi} + \int_0^2 f(\xi) \diff{\xi} + \int_2^\infty f(\xi) \diff{\xi} \\
            &= \int_0^2 f(\xi) \diff{\xi} \\
            &= \left[ \frac{3}{4} \xi^2 - \frac{3}{12} \xi^3 \right]_0^2 \\
            &= 3-2 = 1
            \end{aligned}
        \end{equation*}
        Schlussendlich ist
        \begin{equation*}
            F_X(x) = \begin{cases}
            0 & \text{für } x \leq 0 \\
            \frac{3}{4} x^2 - \frac{3}{12} x^3 & \text{für } x \in (0,2) \\
            1 & \text{für } x \geq 2
            \end{cases}
        \end{equation*}   
        die zugehörige Verteilungsfunktion.     
    \end{enumerate}

%%%% AUFGABE 1.3 %%%%%%%%%%%%%%%%%%%%%%%%%%%%%%%%%%%%%%%%%%%%%%%%%%%%%%%%%%%%%%%%%
    \begin{exercise}
        Sei $\abb{F}{\R}{\R} \colon x \mapsto a + b * \arctan\left( \frac{x-t}{s} \right)$ für Parameter $a,b,t \in \R$ und $s > 0$.
        \begin{enumerate}[leftmargin=*, label=(\alph*)]
            \item Bestimmen Sie $a$ und $b$ so, dass $F$ eine Verteilungsfunktion ist.
            \item Bestimmen Sie die zugehörige Dichtefunktion.
        \end{enumerate}
    \end{exercise}
    
    \begin{enumerate}[leftmargin=*, label=(\alph*)]
        \item Nach Satz 1.19 gilt für jede Verteilungsfunktion
        \begin{equation*}
        \lim_{x \to -\infty} F(x) = 0 \quad \text{und} \quad \lim_{x \to \infty} F(x) = 1
        \end{equation*}
        Wenden wir dies auf die obige Funktion $F$ an unter Berücksichtigung, dass $\lim_{x \to \pm \infty} \arctan(x) = \pm \frac{\pi}{2}$. Dann gilt
        \begin{equation*}
        \lim_{x \to \pm \infty} a + b* \arctan \left(\frac{x-t}{s} \right) = a + b * \arctan \left(\frac{x-t}{s} \right)
        \end{equation*}
        Unterscheiden wir nun den Grenzprozess in die jeweilige Richtung, so erhalten wir das folgende lineare Gleichungssystem
        \begin{equation*}
        \sysdelim[]
        \systeme[ab]{a - b * \frac{\pi}{2} \overset{!}{=} 0 , a + b * \frac{\pi}{2} \overset{!}{=} 1} 
        \follows \sysdelim[]\systeme[b]{a = \frac{\pi}{2} b , a = 1 - \frac{\pi}{2} b} 
        \follows \systeme[c]{b = \pi^{-1} , a = 0.5}
        \end{equation*}
        Somit ist also $F(x) \defeq 0.5 + \frac{1}{\pi} \arctan \left( \frac{x-t}{s} \right)$. 
        
        \paragraph{Monotonie.} Seien $x,y \in \R$ mit $x < y$. Dann ist auch $x -t < y - t$ und $\frac{x-t}{s} < \frac{y-t}{s}$ für alle $s,t \in \R, s > 0$. Da der Arkustangens strikt monoton wachsend ist, ist auch $\arctan \left( \frac{x-t}{s} \right) < \arctan \left( \frac{y-t}{s} \right)$ und da sich durch die Parameter $a$ und $b$ nichts mehr ändert, gilt
        \begin{equation*}
        0.5 + \frac{1}{\pi} \arctan \left( \frac{x-t}{s} \right) < 0.5 + \frac{1}{\pi} \arctan \left( \frac{y-t}{s} \right) \follows F(x) < F(y)
        \end{equation*}
        d.h. die Funktion $F$ ist (streng) monoton wachsend.
        
        \paragraph{Rechtsstetigkeit.} Sei $\folge{x_n} \subseteq \R$ mit $x_n \searrow x$ für $n \to \infty$. Dann gilt nach den Rechenregeln für Folgen auch $\frac{x_n - t}{s} \searrow \frac{x - t}{s}$. Da der Arkustangens stetig auf $\R$ ist, ist also $\arctan(y_n) \to \arctan(y)$ für $\folge{y_n} \subseteq \R$ mit $y_n \to y$, jeweils $n \to \infty$, insbesondere also auch $y_n \searrow y$. Also folgt mit den Rechenregeln für Folgen wiederum
        \begin{equation*}
        0.5 + \frac{1}{\pi} \arctan \left( \frac{x_n-t}{s} \right) \enskip \longrightarrow \enskip 0.5 + \frac{1}{\pi} \arctan \left( \frac{x-t}{s} \right) \quad (n \to \infty)
        \end{equation*}
        was die insbesondere auch die Rechtsstetigkeit von $F$ zeigt. 
        
        Damit definiert $F$ nach Satz 1.19 eine Verteilungsfunktion.
        %
        \item Für die Verteilungsfunktion $\abb{f}{\R}{\R}$ gilt $f(x) = F'(x)$. Wegen
        \begin{equation*}
            \ableitung{x} \arctan(x) = \frac{1}{x^2 + 1}
        \end{equation*}
        ist
        \begin{equation*}
            \ableitung{x} \arctan\left(\frac{x-t}{s}\right) = \frac{1}{\left(\frac{x-t}{s}\right)^2 + 1} * \frac{1}{s}
        \end{equation*}
        Also ist schließlich
        \begin{equation*}
            f(x) = F'(x) = \frac{b}{s \left( \left( \frac{x-t}{s} \right)^2 + 1 \right)}
            = \frac{b}{\frac{(x-t)^2}{s} + \frac{s^2}{s}} 
            = \frac{sb}{s^2 + (x-t)^2} 
            = \frac{s}{\pi \left( s^2 + (x-t)^2 \right)}
        \end{equation*}
        die zugehörige Dichtefunktion.
    \end{enumerate}

%%%% AUFGABE 1.4 %%%%%%%%%%%%%%%%%%%%%%%%

    \begin{exercise}
        \begin{enumerate}[leftmargin=*, label=(\alph*)]
            \item Es sei $(\Omega , \ereignisF , \P)$ ein \WRaum und $A,B \in \ereignisF$ Ereignisse mit $\P(A) = \lfrac{3}{4}$ und $\P(B) = \lfrac{1}{3}$. Zeigen Sie: $\lfrac{1}{12} \leq \P(A \cap B) \leq \lfrac{1}{3}$.
        \item Es sei $(\Omega , \ereignisF , \P)$ ein \WRaum und $A,B,C \in \ereignisF$ Ereignisse mit
        \begin{equation*}
            \begin{aligned}
                \P(A) &= 0.7 \qquad & \P(A \cap B) &= 0.4  \\
                \P(B) &= 0.6 & \P(A \cap C) &= 0.3 \qquad & \P(A \cap B \cap C) &= 0.1 \\
                \P(C) &= 0.5 & \P(B \cap C) &= 0.2 \\
            \end{aligned}
        \end{equation*}
        Berechnen Sie $\P(A \cup B)$, $\P(A \cup B \cup C)$ und $\P(A^\complement \cap B^\complement \cap C)$.
        \end{enumerate}
    \end{exercise}
    
    \begin{enumerate}[leftmargin=*, label=(\alph*)]
        \item Für die Abschätzung nach unten betrachten wir $(A \cap B)^\complement = A^\complement \cup B^\complement$. Dafür gilt die $\sigma$-Subadditivität:
        \begin{equation*}
        \P(A^\complement \cup B^\complement) \leq \P(A^\complement) + \P(B^\complement) = 1 - \P(A) + 1 - \P(B) = 2 - \frac{3}{4} - \frac{1}{3} = \frac{11}{12}
        \end{equation*} 
        Somit ist $\P(A \cap B) = 1 - \P((A \cap B)^\complement) \geq \frac{1}{12}$. Für die Abschätzung nach oben verwenden wir die Monotonie des (Wahrscheinlichkeits-)Maßes $\P$. Es ist klar, dass $A \cap B \subseteq A$ und $A \cap B \subseteq B$ gilt, d.h. somit ist $\P(A \cap B) \leq \P(A)$ \textit{und} $\P(A \cap B) \leq \P(B)$. Damit folgt nun auch schon $\P(A \cap B) \leq \min \menge{\P(A) , \P(B)} = \P(B) = \frac{1}{3}$. Somit ist also die Ungleichungskette $\lfrac{1}{12} \leq \P(A \cap B) \leq \lfrac{1}{3}$ gezeigt.
        \item 
        \begin{align*}
            \P(A \cup B) = \P(A) + \P(B) - \P(A \cap B) = 0.7 + 0.6 - 0.4 = 0.9
        \end{align*}
        \begin{align*}
            \P(A \cup B \cup C) &= \P(A \cup B) + \P(C) - \P((A \cup  B) \cap C) \\
            &= \P(A \cup B) + \P(C) - \P((A \cap C) \cup  (B \cap C)) \\
            &= \P(A \cup B) + \P(C) - \P(A \cap C) - \P(B \cap C) + \P(A \cap B \cap C) \\
            &= 0.9 + 0.5 - 0.3 - 0.2 + 0.1 \\
            &= 1
        \end{align*}
        \begin{align*}
            \P \left( A^\complement \cap B^\complement \cap C \right)
            &= \P \left( \left( A \cap B \cap C^\complement \right)^\complement \right) \\
            &= 1 - \P \left( A \cup B \cup C^\complement \right) \\
            &= 1 - \P(A \cup B) - \P(C^\complement) + \P((A \cup B) \cap C^\complement) \\
            &= 1 - \P(A \cup B) - \P(C^\complement) + \P((A \cap C^\complement) \cup (B \cap C^\complement)) \\
            &= 1 - \P(A \cup B) - \P(C^\complement) + \P(A \cap C^\complement) + \P(B \cap C^\complement) - \P(A \cap B \cap C^\complement) \\
            &= 1 - \P(A \cup B) - \P(C^\complement) + \P(A \setminus (A \cap C)) + \P(B \setminus (B \cap C)) \\
            &\phantom{=} - \P((A \cap B) \setminus (A \cap B \cap C)) \\
            &= 1 - \P(A \cup B) - 1 + \P(C) + \P(A) - \P(A \cap C)) \\
            &\phantom{=} + \P(B) - \P(B \cap C) - \P(A \cap B) + \P(A \cap B \cap C) \\
            &= \P(A) + \P(B) + \P(C) - \P(A \cap B) - \P(A \cap C) - \P(B \cap C) + \P(A \cap B \cap C) \\
            &= 0.7 + 0.6 + 0.5 - 0.9 - 0.4 - 0.3 - 0.2 + 0.1 \\
            &= 0.1
        \end{align*}
    \end{enumerate} 


%%%% AUFGABE 1.5 %%%%%%%%%%%%%%%%%%%%%%%%
    
    \begin{exercise}
        Es sei $(\Omega , \ereignisF , \P)$ ein Wahrscheinlichkeitsraum und $A_1 , \dots , A_n \in \ereignisF$ seien Ereignisse. Zeigen Sie
        \begin{equation*}
            \P \left( \bigcap_{k=1}^n \right) \geq \sum_{k=1}^n \P(A_k) - n + 1
        \end{equation*}
    \end{exercise}

    \begin{equation*}
        \begin{aligned}
            \P \left( \bigcap_{k=1}^n \right) 
            = \P \left( \left( \bigcup_{k=1}^n A_k^\complement \right)^\complement \right)
            &= 1 - \P \left( \bigcup_{k=1}^n A_k^\complement \right) \\
            \overset{\sigma\text{-Subadd.}}&{\geq} 1 - \sum_{k=1}^n \P(A_k^\complement) \\
            &= 1 - \sum_{k=1}^n \left( 1 - \P(A_k) \right) \\
            &= 1 - n + \sum_{k=1}^n \P(A_k) \\
            &= \sum_{k=1}^n \P(A_k) - n + 1
        \end{aligned}
    \end{equation*}
    
\pagebreak
    
%%%%% AUFGABE 1.6 %%%%%%%%%%%%%%%%%%%%%%%
    
    \begin{exercise}
        Es sei $(\Omega , \ereignisF)$ ein messbarer Raum und $\abb{f}{\Omega}{\R}$ eine messbare Funktion. Zeigen Sie, dass dann auch $\abs{f}$ messbar ist. Beweisen oder widerlegen Sie die Umkehrung.
    \end{exercise}

    Wir zeigen zuerst die Hinrichtung. 
    \begin{itemize}[leftmargin=*]
        \item Sei $\folge{f_n} \subseteq \mathcal{M}(\ereignisF)$. Dann ist auch $\sup_{n \in \N} f_n \in \mathcal{M}(\ereignisF)$, da aus
        \begin{equation*}
            \begin{aligned}
                x \in \menge{\sup_{n \in \N} f_n > a} 
                &\equivalent \exists n_0 \in \N \colon a < f_{n_0}(x) \leq \sup_{n \in \N} f_n(x) \\
                &\equivalent x \in \bigcup_{n \in \N} \menge{f_n > a}
            \end{aligned}
        \end{equation*}
        folgt, dass $\menge{\sup_{n \in \N} f_n(x) < a} = \bigcup_{n \in \N} \menge{f_n(x) < a} \in \ereignisF$. Und schließlich ist $\sup_{n \in \N} f_n(x)$ genau dann messbar, wenn $\menge{\sup_{n \in \N} f_n(x) < a}$ es für alle $a \in \Omega$ ist. Analog zeigt man dies auch für das Infimum und erhält damit die Messbarkeit von Limes inferior und Limes superior. Existiert schließlich auch der Limes, so ist auch dieser messbar.
        \item Sei $f \in \mathcal{M}(\ereignisF)$. Dann existiert nach dem Sombrero-Lemma eine Folge $\folge{f_n} \in \mathcal{E}(\ereignisF)$ mit $f_n \longrightarrow f$ für $n \to \infty$. Es ist klar, dass alle einfachen Treppenfunktionen $f_n$ messbar sind. Aufgrund des ersten Punktes ist dann auch der Grenzprozess $\lim_{n \to \infty} f_n = f$ messbar.
        \item Damit ist dann auch schon $f^+ \defeq \max(f , 0)$ und $f^- \defeq - \min(f, 0)$ messbar, da sich $f$ jeweils durch eine Folge einfacher Funktionen annähern lasst und die Null durch die konstante Nullfolge beschreibbar ist. Insbesondere sind dann auch die jeweiligen Extrema von $f_n$ und $0$ wieder einfach und somit messbar. Schließlich ist $f^+$ und $f^-$ messbar.
        \item Sei $f$ eine messbare Funktion. Dann sind auch $f^+$ und $f^-$ messbar, was bereits die Messbarkeit von $\abs{f}$ impliziert, da $\abs{f} = f^+ + f^-$.
    \end{itemize}

    Die Umkehrung ist im Allgemeinen falsch. Zur Konstruktion eines Gegenbeispiels bezeichne $V$ die aus der Übung bekannte Vitali-Menge, welche nicht messbar ist. Betrachten wir dazu die Funktion
    \begin{equation*}
        f(x) \defeq \left( \one_V - \one_{V^\complement} \right) (x)       
        =   \begin{cases}
                1 &\text{falls } x \in V \\
                -1 &\text{falls } x \notin V
            \end{cases}
    \end{equation*}
    Offensichtlich ist $\abs{f} \equiv 1$ und damit messbar. Jedoch ist $f$ nicht messbar, weil beispielsweise $f^{-1}(\menge{1}) = V \notin \ereignisF$ nicht messbar ist.
\end{exercisePage}